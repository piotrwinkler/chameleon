\section*{A. INSTRUKCJA DLA UŻYTKOWNIKA}

  Uruchomienie zawartych w projekcie programów wymaga interpretera
  języka Python w wersji 3.6.8 lub wyższej!

  UWAGA, poniższa instrukcja zawiera komendy przeznaczone dla systemu operacyjnego
  Linux!

  Aby poprawnie skonfigurować środowisko należy utworzyć folder docelowy, np.
  wpisując w konsoli:

  \textit{mkdir <nazwa folderu z projektem>}

  Następnie należy umieścić w folderze pliki projektu. Przed wykonaniem
  kolejnego kroku zalecane jest utworzenie wirtualnego środowiska za pomocą
  polecenia:

  \textit{python3 -m venv <nazwa folderu z wirtualnym środowiskiem>},

  oraz aktywowanie go poprzez komendę:

  \textit{source <nazwa folderu z wirtualnym środowiskiem>/bin/activate}

  Teraz należy wejść do folderu głównego poprzez polecenie:

  \textit{cd <nazwa folderu z projektem>}

  Wewnątrz tego folderu należy wykonać polecenia umożliwiające instalację niezbędnych
  pakietów oraz zbudowanie projektu:

  \textit{pip install -U setuptools}

  \textit{python setup.py install}

  \textit{pip install -r requirements.txt}

  Po wykonaniu tych operacji środowisko jest gotowe do uruchomienia. W ramach
  przygotowanego projektu udostępnione zostały skrypty umożliwiające trenowanie
  i testowanie sieci neuronowych filtrujących oraz kolorujących czarno-białe
  obrazy. W celu uruchomienia odpowiedniego programu należy wejść do folderu
  docelowego:

  \textit{cd <nazwa folderu docelowego> (np. sepia)}

  Aby rozpocząć proces uczenia sieci należy zastosować komendę:

  \textit{python train <odpowiednia nazwa> (np. train sepia)}

  Aby rozpocząć testy działania sieci należy zastosować komendę:

  \textit{python test <odpowiednia nazwa> (np. test sepia)}

  Poprawne działanie przedstawionych skryptów wymaga dostarczenia odpowiednich
  danych wejściowych. Aby skorzystać z własnego zbioru treningowego należy podać
  ścieżkę do folderu z obrazami w skrypcie \textit{data/consts.py}, w folderze docelowym,
  pod nazwą \textit{TRAINING DATASET DIRECTORY}. W tym samym pliku należy podać
  również ścieżkę do zbioru testowego pod nazwą \textit{TEST DATASET DIRECTORY}.
  Aby poprawnie wykonać testy konieczne jest również zdefiniowanie ścieżki do
  wytrenowanego modelu pod nazwą \textit{NET LOADING DIRECTORY}.

  Rozwiązanie kolorowania czarno-białych obrazów korzysta ze zbioru uczącego
  dostępnego poprzez API języka Python, dlatego nie jest wymagane dostarczenie
  własnych danych w tym przypadku.

  Aby dokonać konfiguracji w systemie Windows możliwe jest wykorzystanie
  podanych poleceń za pośrednictwem środowiska deweloperskiego \textit{MinGW}.

\section*{B. INSTRUKCJA DLA DEWELOPERA}

  W celu początkowego skonfigurowania środowiska oraz uruchomienia procesu
  treningowego lub testów sieci z wykorzystaniem frameworka \textit{TorchFrame}
  należy kierować się instrukcjami zawartymi w dodatku \textit{A}.

  W ramach pracy deweloperskiej możliwe jest konfigurowanie parametrów systemu
  zgodnie z opisem zawartym w rozdziale \ref{TorchFrame}. Ponadto zalecane jest
  napisanie dedykowanych wersji następujących komponentów:

  \begin{itemize}
  \item \textit{Dataset} - klasa dziedzicząca po \textit{BaseDataset}. W ramach
  metody magicznej \textit{getitem} należy zdefiniować sposób pobierania i przetwarzania
  pojedynczej próbki danych treningowych przekazywanych na wejście sieci.

  \item \textit{Tester} - klasa dziedzicząca po \textit{BaseTester}. W jej obrębie
  należy zdefiniować sposób testowania pożądanego rozwiązania. Można wykorzystać
  do tego predefiniowane metody statyczne klasy bazowej służące między innymi
  do wyświetlania obrazów oraz implementowania przekazanych konwersji zdefiniowanych
  w pliku \textit{conversions.py}.

  \item \textit{conversions.py} - w tym skrypcie można zdefiniować własne metody
  konwersji danych. Każda z konwersji jest klasą. W ramach metody magicznej \textit{call}
  należy zdefiniować sposób przetwarzania pojedynczej próbki danych (w większości
  przygotowanych przykładów jest to pojedynczy obraz). Podczas inicjalizacji
  do klasy przekazane zostaną parametry zdefiniowane w pliku konfiguracyjnym
  \textit{JSON}.

  \item \textit{entrypoint} - W dostarczonych przykładach są to wszystkie
  skrypty typu \textit{train.py} oraz \textit{test.py}. Istotne jest
  zawarcie w nich odczytu danych z pliku \textit{JSON} oraz konfiguracji środowiska tak,
  jak zostało to zrobione w przykładowych rozwiązaniach. Można również dowolnie
  modyfikować ilość i rodzaj ścieżek systemowych podawanych podczas inicjalizacji
  klas \textit{Trainer} oraz \textit{Tester}.
  \end{itemize}

  Bardziej zaawansowani użytkownicy mogą pokusić się o modyfikację skryptu
  \textit{trainer.py}, zawierającego definicję przebiegu procesu uczenia.
  Zalecane jest jednak zaznajomienie się z biblioteką \textit{PyTorch} przed wykonaniem
  jakichkolwiek zmian.

  Jedną z najbardziej oczywistych ścieżek rozwoju projektu jest
  potrzeba zdefiniowania GUI ułatwiającego pracę z frameworkiem, a co
  za tym idzie z samymi sieciami neuronowymi.
