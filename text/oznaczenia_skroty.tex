\section*{WYKAZ WAŻNIEJSZYCH OZNACZEŃ I SKRÓTÓW}
\addcontentsline{toc}{section}{WYKAZ WAŻNIEJSZYCH OZNACZEŃ I SKRÓTÓW} % Dodanie to TOC

  \bigskip

  \begin{itemize}
    \item[ANN] (ang. artificial neural network) - Sztuczna sieć neuronowa
    \item[DNN] (ang. deep neural network) - Głęboka sieć neuronowa
    \item[CNN] (ang. convolutional neural network) - Splotowa sieć neuronowa
    \item[FCN] (ang. Fully Convolutional Network) - Warstwa gęsta
    \item[GAN] (ang. generative adversarial network) - Sieci o modelu generatywnym
    \item[VAE] (ang. Variational Autoencoder) - Autoenkodery wariacyjne
    \item[ReLU] (ang. Rectified Linear Unit) - Jednostronnie obcięta funkcja liniowa
    \item[BatchNorm] (ang. Batch normalization) - Normalizacja zbioru danych
    pogrupowanych w pakiety
    \item[YUV] - Model barw, w którym Y odpowiada za luminancję obrazu, a UV
    są to dwa kanały chrominancji i kodują barwę
    \item[IcGAN] (ang. Invertible conditional Generative Adversarial Network) -
    Odwracalny warunkowy model generatywny
    \item[cGAN] (ang. conditional Generative Adversarial Network) - warunkowy
    model generatywny

  \end{itemize}
