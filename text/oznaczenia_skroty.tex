\section*{WYKAZ WAŻNIEJSZYCH OZNACZEŃ I SKRÓTÓW}
\addcontentsline{toc}{section}{WYKAZ WAŻNIEJSZYCH OZNACZEŃ I SKRÓTÓW} % Dodanie to TOC

  \bigskip

  \begin{itemize}
    \item[AdaDelta] (ang. Adaptive Delta) - Rozszerzenie Adaptacyjnego Algorytmu Gradientowego
    \item[AdaGrad] (ang. Adaptive Gradient Algorithm) - Adaptacyjny Algorytm Gradientowy
    \item[Adam] (ang. Adaptive Moment Estimation) - Estymacja Momentu Adaptacyjnego
    \item[AI] (ang. Artificial Intelligence) - Sztuczna inteligencja
    \item[ANN] (ang. Artificial Neural Network) - Sztuczna sieć neuronowa
    \item[BatchNorm] (ang. Batch Normalization) - Normalizacja zbioru danych
    pogrupowanych w pakiety
    \item[cGAN] (ang. conditional Generative Adversarial Network) - Warunkowe,
    generatywne sieci przeciwstawne
    \item[CNN] (ang. Convolutional Neural Network) - Splotowa sieć neuronowa
    \item[CPU] (ang. Central Processing Unit) - Centralna Jednostka Obliczeniowa
    \item[DNN] (ang. Deep Neural Network) - Głęboka sieć neuronowa
    \item[Dropout] (pol. algorytm odrzucania) - Technika
    regularyzacji mająca na celu ograniczać przeuczanie się sieci neuronowych
    \item[FCL] (ang. Fully Connected Layer) - Warstwa gęsta
    \item[FCN] (ang. Fully Convolutional Network) - Sieć w pełni splotowa
    \item[GAN] (ang. Generative Adversarial Network) - Generatywne sieci
    przeciwstawne
    \item[GPU] (ang. Graphics Processing Unit) - Graficzna Jednostka Obliczeniowa
    \item[hiperparametry] - Parametry warunkujące przebieg procesu uczenia sieci neuronowych,
    takie jak np. długość kroku treningowego czy ilość epok treningowych
    \item[IcGAN] (ang. Invertible conditional Generative Adversarial Network) -
    Odwracalne, warunkowe, generatywne sieci przeciwstawne
    \item[JSON] (ang. JavaScript Object Notation) - Lekki, tekstowy format wymiany danych komputerowych
    \item[NN] (ang. Neural Network) - Sieć neuronowa
    \item[OpenCV] (ang. Open Source Computer Vision Library) - Biblioteka wizji komputerowej z licencją zgodną z zasadami wolnego oprogramowania
    \item[p] - Prawdopodobieństwo dezaktywowania neuronu sieci przez warstwę Dropout
    \item[PReLU] (ang. Parametric Rectified Linear Unit) - Parametryczna,
    jednostronnie obcięta funkcja liniowa
    \item[ReLU] (ang. Rectified Linear Unit) - Jednostronnie obcięta funkcja liniowa
    \item[ResNet] (ang. Residual Networks) - Sieć szczątkowa
    \item[RGB] (ang. Red Green Blue) - Model przestrzeni barw
    \item[RReLU] (ang. Randomized Leaky Rectified Linear Unit) - Losowo nieszczelna,
    jednostronnie obcięta funkcja liniowa
    \item[SAIL] (ang. Stanford Artificial Intelligence Laboratory) - Laboratorium Sztucznej Inteligencji Stanforda
    \item[SGD] (ang. Stochastic Gradient Descent) - Stochastyczny Spadek Gradientu
    \item[tensor] - Macierzowa struktura danych biblioteki PyTorch
    \item[YUV] - Model barw, w którym kanał Y odpowiada za luminancję obrazu, a UV
    są to dwa kanały chrominancji kodujące barwy
    \item[VAE] (ang. Variational Autoencoder) - Autoenkodery wariacyjne

  \end{itemize}
