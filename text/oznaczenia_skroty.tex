\section*{WYKAZ WAŻNIEJSZYCH OZNACZEŃ I SKRÓTÓW}
\addcontentsline{toc}{section}{WYKAZ WAŻNIEJSZYCH OZNACZEŃ I SKRÓTÓW} % Dodanie to TOC

  \bigskip

  \begin{itemize}
    \item[ANN] (ang. Artificial Neural Network) - Sztuczna sieć neuronowa
    \item[DNN] (ang. Deep Neural Network) - Głęboka sieć neuronowa
    \item[FCL] (ang. Fully Connected Layer) - Warstwa gęstwa
    \item[CNN] (ang. Convolutional Neural Network) - Splotowa sieć neuronowa
    \item[FCN] (ang. Fully Convolutional Network) - Sieć w pełni splotowa
    \item[GAN] (ang. Generative Adversarial Network) - Sieci o modelu generatywnym
    \item[VAE] (ang. Variational Autoencoder) - Autoenkodery wariacyjne
    \item[ReLU] (ang. Rectified Linear Unit) - Jednostronnie obcięta funkcja liniowa
    \item[BatchNorm] (ang. Batch Normalization) - Normalizacja zbioru danych
    pogrupowanych w pakiety
    \item[YUV] - Model barw, w którym Y odpowiada za luminancję obrazu, a UV
    są to dwa kanały chrominancji i kodują barwę
    \item[IcGAN] (ang. Invertible conditional Generative Adversarial Network) -
    Odwracalny warunkowy model generatywny
    \item[cGAN] (ang. conditional Generative Adversarial Network) - warunkowy
    model generatywny
    \item[Dropout] (!!!! ang. spadkowicz - tłumaczył Piotr Winkler) - technika
    regularyzacji mająca na celu ograniczać przeuczanie się sieci neuronowych
    przez zapobieganie złożonej koadaptacji danych treningowych
    \item[PReLU] (ang. Parametric Rectified Linear Unit) - Parametryczna
    jednostronnie obcięta funkcja liniowa
    \item[RReLU] (ang. Randomized Leaky Rectified Linear Unit) - Losowo nieszczelna
    jednostronnie obcięta funkcja liniowa
    \item[Adam] (ang. Adaptive Moment Estimation) - Adaptacyjna estymacja pędu.
    \item[SGD] (ang. Stochastic Gradient Descent) - Metoda stochastycznego spadku
    gradientowego.
    \item[AdaGrad] (ang. Adaptive Gradient Algorithm) - Adaptacyjny algorytm gradientowy.



  \end{itemize}
