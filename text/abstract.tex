\section*{Abstract}

  The subject of this project is to explore the possibilities of using neural networks for
  image editing. The main goal was to create tools, based on
  trained neural networks, used for proper processing and modifying of
  images. Then the effectiveness of these tools was evaluated and
  compared with solutions based on classic image processing methods.

  Two approaches were investigated, automatic coloring of black and white images
  and applying simple filters on images such as Sobel-Feldman filter detecting edges.

  To make the research easier, an original framework named TorchFrame has been created.
  It has the task of controlling
  data flow in the process of teaching and testing the examined models and hence
  reduce the need for user interference while ensuring freedom
  in conducted experiments and implemented data modifications.

  For simple image filters, solutions based on convolutional neural networks proved to
  be slower and more labor-intensive in implementation than classical methods,
  while presenting comparable results.
  However, with their help it was possible to prove the enormous universality
  of convolutional neural networks, which can be used in issues related to
  simple image processing. Furthermore, the similarity between traditional filtration
  methods and the way, how the neurons in convolutional networks works, has been shown.
  Experiments, carried out as part of this project, proves that neural networks can
  be successfully used as tools in general image editing process. However, despite
  their remarkable possibilities, success highly depends on the well considered
  architecture and correctly carried out training process.

  \bigskip

  \noindent\textbf{Keywords:} neural network, image processing, convolutional
  neural network, generative adversarial network, deep neural network,
  image filters, automatic image colorizing, framework

  \bigskip

  \noindent\textbf{Field of science and technology in accordance with the
  requirements of the OECD:} Engineering and technology, Electrical engineering,
  Electronic engineering, Information engineering, Computer hardware and
  architecture
