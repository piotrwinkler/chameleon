\subsection{Filtry AI}

  Filtrowanie obrazów cyfrowych to bardzo popularny i powszechnie stosowany
  obecnie proces. Pozwala wyostrzyć niewyraźne zdjęcie, zmienić kontrast obrazu,
  czy zniwelować szumy tła. W rzeczywistości filtrowanie to nic innego, jak
  operacja matematyczna wykonywana na pikselach. Wykorzystywanie wartości wielu
  pikseli obrazu źródłowego w celu określenia wartości pojedynczego piksela w
  obrazie wynikowym. Sposób w jaki wartości te są pobierane oraz przetwarzane
  określają tak zwane maski. Przyjmują one postać macierzy kwadratowych różnych
  rozmiarów, a przechowywane w nich wartości decydują o wyniku filtracji.

  Poniższy rozdział tej pracy spróbuje udzielić odpowiedzi na pytanie, czy
  sieci neuronowe mogą sprawnie posłużyć w procesie filtrowania obrazów.
  Składa się na niego seria eksperymentów, w których specjalnie dobrane modele
  sieci spróbują odtworzyć wartości masek użytych do przygotowania danych
  treningowych, a następnie wykorzystają je do przetworzenia zupełnie nowych
  obrazów.

  Dane referencyjne składają się z zestawu obrazów przetworzonych za pomocą
  filtrów wbudowanych w bibliotekę \textit{OpenCV} takich, jak filtr Sobela, czy
  sepia.

  Wszystkie modele wytrenowane zostały w oparciu o framework TorchFrame.

  \subsubsection{Filtr Sobela}


  \subsubsection{Sepia}

  \subsubsection{Filtr górnoprzepustowy}
