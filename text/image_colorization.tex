\subsection{Automatyczne kolorowanie czarno-białych obrazów}

  Problem kolorowania czarno-białych obrazów cieszy się dużym zainteresowaniem z
  wielu powodów. Od potrzeb kulturowych takich jak możliwość lepszego
  zwizualizowania oraz zrozumienia przeszłości poprzez kolorowania zdjęć z
  czasów, kiedy występowały one jedynie w kolorach czerni i bieli, po potrzeby
  technologiczne takie jak rekonstrukcja filmów oraz poprawa obrazu cyfrowego.

  Pomimo braku informacji o kolorze w czarno-białych zdjęciach, ludzie są w
  stanie określić potencjalne, rzeczywiste barwy obiektów na zdjęciach bazując
  na treści tych zdjęć oraz swoim doświadczeniu. Można z tego wywnioskować, że
  zdjęcia te zawierają informacje wystarczające do oszacowania potencjalnych
  kolorów. Pozwala to założyć, że do tego zagadnienia można skutecznie wykorzystać
  konwolucyjne sieci neuronowe, które cechują się niezwykłą umiejętnością do
  rozpoznawania wzorców oraz posiadają wyjątkowe zdolności do adaptacji. Z tego
  właśnie powodu sieci splotowe zostaną użyte w przedstawionym rozwiązaniu.

  \subsubsection{Podejście}

  Rozważając możliwe sposoby pokolorowania czarno-białego zdjęcia można spostrzec,
  że kiedy niektóre powierzchnie na zdjęciu mają zazwyczaj oczywiste barwy, niebo
  jest zazwyczaj niebieskie, a trawa zielona, to są też powierzchnie, które
  posiadają szeroki wachlarz możliwych kolorów, na przykład samochodów może być
  zarówno czerwony jak i niebieski albo zielony. Z tego powodu celem zaprezentowanego
  rozwiązania jest niekoniecznie odtworzenie rzeczywistych barw obrazu, a raczej
  wygenerowanie barw, które mogłyby być barwami rzeczywistymi.

  LAB format

  \subsubsection{Model podstawowy}

  \subsubsection{Wykorzystywany zbiór treningowy}

  \subsubsection{Funkcje kosztów}

  \subsubsection{Funkcje aktywacji}

  \subsubsection{BatchNorm}

  \subsubsection{Dropout}

  \subsubsection{Trening}

  \subsubsection{Rezultaty}
