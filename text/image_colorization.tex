\subsection{Automatyczne kolorowanie czarno-białych obrazów}

  Problem kolorowania czarno-białych obrazów cieszy się dużym zainteresowaniem z
  wielu powodów. Od potrzeb kulturowych takich jak możliwość lepszego
  zwizualizowania oraz zrozumienia przeszłości poprzez kolorowania zdjęć z
  czasów, kiedy występowały one jedynie w kolorach czerni i bieli, po potrzeby
  technologiczne takie jak rekonstrukcja filmów oraz poprawa obrazu cyfrowego.

  Pomimo braku informacji o kolorze w czarno-białych zdjęciach, ludzie są w
  stanie określić potencjalne, rzeczywiste barwy obiektów na zdjęciach bazując
  na treści tych zdjęć oraz swoim doświadczeniu. Można z tego wywnioskować, że
  zdjęcia te zawierają informacje wystarczające do oszacowania potencjalnych
  kolorów. Pozwala to założyć, że do tego zagadnienia można skutecznie wykorzystać
  konwolucyjne sieci neuronowe, które cechują się niezwykłą umiejętnością do
  rozpoznawania wzorców oraz posiadają wyjątkowe zdolności do adaptacji. Z tego
  właśnie powodu sieci splotowe zostaną użyte w przedstawionym rozwiązaniu.

  \subsubsection{Podejście}

  Rozważając możliwe sposoby pokolorowania czarno-białego zdjęcia można spostrzec,
  że kiedy niektóre powierzchnie na zdjęciu mają zazwyczaj oczywiste barwy, niebo
  jest zazwyczaj niebieskie, a trawa zielona, to są też powierzchnie, które
  posiadają szeroki wachlarz możliwych kolorów, na przykład samochodów może być
  zarówno czerwony jak i niebieski albo zielony. Z tego powodu celem zaprezentowanego
  rozwiązania jest niekoniecznie odtworzenie rzeczywistych barw obrazu, a raczej
  wygenerowanie barw, które mogłyby być barwami rzeczywistymi.

  Aby zwiększyć efektywność uczenia wykorzystano przestrzeń barw CIELab. W tej
  przestrzeni barwę obrazu opisują 3 składowe:
  \begin{itemize}
  \item L -jasność (luminacja)
  \item A -barwa od zielonej do magenty
  \item B - barwa od niebieskiej do żółtej
  \end{itemize}
  Przestrzeń barw CIELab została przedstawiona na Rysunku \ref{fig:CIELab}

  \begin{figure}[ht]
    \centering
    \includegraphics[width=4in]{CIELab}
    \caption[Przestrzeń barw CIELab - źródło:
    \url{https://www.flickr.com/photos/greenmambagreenmamba/4236391637}]
    {Przestrzeń barw CIELab.}
    \label{fig:CIELab}
  \end{figure}

  Zaletą zastosowania CIELab jest fakt, że jest ona najbardziej równomierną
  przestrzenią barw, co oznacza, że jeśli barwy znajdują się w jednakowej
  odległości od siebie w tej przestrzeni, to będą one postrzegane jako jednakowo
  różniące się od siebie. Powinno to zwiększyć skuteczność uczenia sieci oraz
  zapewnić bardziej realistyczne kolorowanie.

  Składowa \textit{L}, jako, że jest identyczna dla obrazu kolorowego jak i
  czarno-białego, stanowi w tym przypadku wejście sieci, na jej podstawie sieć
  odtwarza składowe \textit{A} oraz \textit{B}, które reprezentują przewidziane
  kolory dla obrazu wejściowego.

  Jako rozwiązanie podanej problematyki wpierw został oceniony autorski
  model podstawowy, na jego podstawie zostało przeprowadzone porównanie skuteczności
  różnych konfiguracji, w których uczony był model. Do elementów poddanych
  testom należą algorytm optymalizacyjny, funkcja straty, funkcja aktywacja oraz
  sposób przetwarzania wstępnego danych treningowych.

  \subsubsection{Model podstawowy}

  \subsubsection{BatchNorm}

  \subsubsection{Dropout}

  \subsubsection{Wykorzystywany zbiór treningowy}

  W trakcie uczenia obraz w przestrzeni barw RGB jest konwertowany do CIELab,
  składowa \textit{L} stanowi w tym przypadku wejście sieci, a składowe \textit{A}
  oraz \textit{B}

  \subsubsection{Funkcje kosztów}

  \subsubsection{Funkcje aktywacji}

  \subsubsection{Trening}

  \subsubsection{Rezultaty}
