\documentclass[10pt]{article}
\usepackage[utf8]{inputenc}
\usepackage[MeX]{polski}
\usepackage{graphicx}
\usepackage{amsmath,amssymb,amsfonts}
\usepackage{pdfpages} % paczka żeby móc importować gotowe strony z plików pdf do pdfa z latexa
\usepackage{titlesec} % może być potrzebne jeśli będziemy chcieli edytować tytuły
\usepackage{tocloft}  % żeby móc ręcznie dodawać sections do TOC (Table of Content)
\usepackage[nottoc,notlot,notlof]{tocbibind}  % to chyba jest potrzebne, żeby bibliografia dobrze się dodawała do TOC
% to powyższe można zastąpić 2 liniami:
% \usepackage{tocbibind}
% \tocbibind{nottoc,notlot,notlof}

\usepackage{geometry} % Do ustawiania wyglądu stron
\geometry{margin=1.2in} % Ustawia margines

\usepackage{sectsty}  % Chyba żeby móc ustawić baselineskip i linespread

%\renewcommand{\chaptername}{Rozdział}
%\renewcommand{\contentsname}{Spis treści}
%\renewcommand{\figurename}{Rys.}
%\renewcommand{\tablename}{Tab.}
%\renewcommand{\listfigurename}{Spis rysunków}
%\renewcommand{\listtablename}{Spis tabel}
%\renewcommand{\bibname}{Bibliografia}

\newcommand\tab[1][1cm]{\hspace*{#1}}

\pagestyle{plain}
\pagenumbering{arabic}
\sectionfont{\fontsize{13}{15}\selectfont}  % Ustawia wielkośc czcionki tytułów sekcji
% First {} is a font size for title of sections and second {} is baselineskip,
% that is size of empty space between section title and text below


\begin{document}
\baselineskip=12pt  % odległość między liniami chyba
\linespread{1.25} % W wymogach jest, że ma być  1.5, ale podobno ten końcowy line spacing
% wylicza się jako ten linespread * (baselineskip/fontsize) i w naszym przypadku
% to powinno być 1.25 * (12/10) = 1.5, no ale zobaczymy

% Strone tytulowa trzeba bedzie pobrac z moja pg
\includepdf[pages=-]{StronaTytulowa_bb.pdf} %pages=- to załączenie wszystkich stron


\begin{titlepage}
  \begin{center}

    \vspace*{1cm}
    \Huge
    \textbf{POLITECHNIKA GDAŃSKA}
    \newline
    \vspace{0.5cm}
    \LARGE
    Katedra Systemów Decyzyjnych i Robotyki

    \vspace{1.5cm}
    \textbf{PRACA INŻYNIERSKA}
    \\[0.5cm]
    \textbf{Zastosowanie sieci neuronowych do edycji obrazów}

    \vspace{2.5cm}
    \Large
    \textbf{Autorzy}\\
    Piotr Winkler\\
    Bartosz Bieliński

    \vspace{3.5cm}
    Gdańsk 2019

  \end{center}
\end{titlepage}

\setcounter{page}{2}

\section*{Streszczenie}

  Tematem pracy jest zbadanie możliwości zastosowania sieci neuronowych do
  edycji obrazów. Głównym celem było stworzenie narzędzi opartych na wyuczonych
  sieciach neuronowych, służących do odpowiedniego przetwarzania i
  modyfikowania obrazu. Następnie skuteczność tych narzędzi została oceniona i
  porównana z rozwiązaniami opartymi na klasycznych metodach przetwarzania obrazu.

  Zbadane zostały dwie problematyki, automatyczne kolorowanie czarno-białych obrazów
  oraz nakładanie na obraz prostych filtrów takich jak wykrywający
  krawędzie filtr Sobela-Feldmana. Opisy zaimplementowanych rozwiązań zostały
  umieszczone w odpowiednich podrozdziałach
  rozdziału \ref{zaimplementowane rozwiazania}.

  Uzyskane wyniki pozwoliły dojść do wielu konstruktywnych obserwacji. W przypadku obu
  problematyk udało się osiągnąć rozwiązania oparte na sztucznych sieciach
  neuronowych.

  W celu ułatwienia przeprowadzania badań został opracowany autorski framework
  o nazwie \textit{TorchFrame}. Ma on za zadanie kontrolować
  przepływ danych w procesie uczenia i testowania badanych modeli, a co za tym
  idzie, ograniczyć konieczność ingerencji ze strony użytkownika, przy jednoczesnym
  zapewnieniu swobody w prowadzonych eksperymentach i implementowanych modyfikacjach danych. Przekłada
  się to na znacznie wydajniejszy oraz mniej złożony proces trenowania
  różnorodnych architektur sieci neuronowych.

  W przypadku prostych filtrów obrazu rozwiązania
  oparte na sieciach splotowych okazały się wolniejsze i bardziej pracochłonne w
  implementacji niż metody klasyczne przy porównywalnych rezultatach. Jednakże
  z ich pomocą udało się udowodnić olbrzymią uniwersalność sieci splotowych
  mogących znaleźć zastosowanie w zagadnieniach związanych z prostym
  przetwarzaniem obrazu. Ukazane zostało ponadto podobieństwo między tradycyjnymi
  metodami filtracji, a sposobem działania neuronów tworzących sieci konwolucyjne.

  Zagadnienie automatycznego kolorowania czarno-białych
  obrazów zostało rozwiązane z użyciem dwóch różnych modeli, autorskiego modelu
  prostego oraz bardziej zaawansowanego modelu zaimplementowanego z użyciem technik
  przeniesienia uczenia. Dla modelu autorskiego zostały przeprowadzone szczegółowe
  badania dotyczące zależności wyników od poszczególnych parametrów architekury
  sieci jak i konfiguracji procesu uczenia. Rezultaty końcowe uzyskane z użyciem
  modelu autorskiego były zadowalające, ale mocno zależne od wybranych parametrów.
  Dowodzi to, że odpowiednio skonfigurowane sieci splotowe mogą być skutecznie
  zastosowane w tym zagadnieniu, aczkolwiek to model złożony pozwolił osiągnąć
  największy sukces.
  Idea tego rozwiązania opiera się na integracji cech obrazu średniego oraz wysokiego
  poziomu uzyskiwanych z użyciem złożonej sieci splotowej wytrenowanej pierwotnie do
  zadania klasyfikacji. Cechy te, po przejściu przez proces fuzji, są następnie
  wykorzystywane przez sieć dekonwolucyjną do predykcji prawdopodobnych barw dla
  obrazu wejściowego. Dzięki tym dodatkowym informacjom o obrazie udało się
  znacznie zwiększyć efektywność procesu kolorowania, a co za tym idzie,
  wiarygodność generowanych barw.

  Badania przeprowadzone w ramach tej pracy dyplomowej są dowodem na to, że
  sieci neuronowe mogą być skutecznie zastosowane jako narzędzia do wszechstronnej
  edycji obrazu. Często są one jedynym dostępnym rozwiązaniem jeśli problematyka
  jest nadzwyczaj złożona. Jednakże, pomimo niezwykłych możliwości sieci
  neuronowych, ich sukces zależy w dużej mierze od dobrze przemyślanego
  wyboru stosowanej architektury oraz poprawnie przeprowadzonego
  procesu uczenia.

  \bigskip

  \noindent\textbf{Słowa kluczowe:} sieć neuronowa, przetwarzanie obrazu,
  konwolucyjna sieć neuronowa, splotowa sieć neuronowa,
  głęboka sieć neuronowa, filtry obrazu, automatyczne kolorowanie czarno-białych
  obrazów, platforma programistyczna

  \bigskip

  \noindent\textbf{Dziedzina nauki i techniki zgodna z OECD:} Nauki
  inżynieryjne i techniczne, Elektrotechnika, elektronika, inżynieria
  informatyczna, Sprzęt komputerowy i architektura komputerów

\section*{Abstract}

  The subject of this project is to explore the possibilities of using neural networks for
  image editing. The main goal was to create tools, based on
  trained neural networks, used for proper processing and modifying of
  images. Then the effectiveness of these tools was evaluated and
  compared with solutions based on classic image processing methods.

  Two approaches were investigated, automatic coloring of black and white images
  and applying simple filters on images such as Sobel-Feldman filter detecting edges.

  To make the research easier, an original framework named TorchFrame has been created.
  It has the task of controlling
  data flow in the process of teaching and testing the examined models and hence
  reduce the need for user interference while ensuring freedom
  in conducted experiments and implemented data modifications.

  For simple image filters, solutions based on convolutional neural networks proved to
  be slower and more labor-intensive in implementation than classical methods,
  while presenting comparable results.
  However, with their help it was possible to prove the enormous universality
  of convolutional neural networks, which can be used in issues related to
  simple image processing. Furthermore, the similarity between traditional filtration
  methods and the way, how the neurons in convolutional networks works, has been shown.

  To solve the problem of black and white images colorization, two different
  approaches were proposed. The first one included the application of simple author's
  model trained with many different hiperparameters and data preparation methods.
  Obtained results were satisfactory but they are greatly dependent from chosen
  hiperparameters and training configuration.

  The second approach included implementing more complex model using transfer
  learning technique. This approach was based on the integration of medium
  and high level features of the image. The information obtained that way was
  then used to predict possible colors of the image.
  At the end, this method resulted in more realistic colors and hence
  highly satisfying results.

  Experiments, carried out as part of this project, proves that neural networks can
  be successfully used as tools in general image editing process. However, despite
  their remarkable possibilities, success highly depends on the well considered choice of used architecture
  and correctly carried out training process.

  \bigskip

  \noindent\textbf{Keywords:} neural network, image processing, convolutional
  neural network, generative adversarial network, deep neural network,
  image filters, automatic image colorizing, framework

  \bigskip

  \noindent\textbf{Field of science and technology in accordance with the
  requirements of the OECD:} Engineering and technology, Electrical engineering,
  Electronic engineering, Information engineering, Computer hardware and
  architecture

\section*{WYKAZ WAŻNIEJSZYCH OZNACZEŃ I SKRÓTÓW}
\addcontentsline{toc}{section}{WYKAZ WAŻNIEJSZYCH OZNACZEŃ I SKRÓTÓW} % Dodanie to TOC

  \bigskip

  \begin{itemize}
    \item[GAN] (ang. generative adversarial network) - sieci o modelu generatywnym
    \item[ANN] (ang. artificial neural network) - sztuczna sieć neuronowa
    \item[DNN] (ang. deep neural network) - głęboka sieć neuronowa
    \item[CNN] (ang. convolutional neural network) - splotowa sieć neuronowa
    \item[FCL] (ang. fully connected layer) - warstwa gęsta
    \item[VAE] (ang. Variational Autoencoder) - utoenkodery wariacyjne
  \end{itemize}



\newpage
  \tableofcontents

\section{Wstęp i cel pracy}
  Sztuczne sieci neuronowe sięgają swym początkiem lat 40. XX wieku.
  Historia ich rozwoju odnotowała trzy okresy, w których rozwiązania te
  odbijały się szerokim echem w środowisku naukowym.

  Pierwszy model neuronu, a potem perceptron zapoczątkowały
  rozwój tej dziedziny nauki, jednak pierwsze sieci jednowarstwowe nie były w
  stanie rozwiązywać złożonych problemów. Przeszkodę nie do pokonania stanowiła
  dla nich nawet prosta funkcja logiczna XOR. Z tego powodu badania sieci
  neuronowych zostały na długi czas porzucone.


  Pojawienie się algorytmu wstecznej propagacji błędów
  pozwalającego skutecznie uczyć wielowarstwowe sieci neuronowe ponownie
  wzmogło zainteresowanie tematem, jednak tym razem na drodze postępowi stanęły
  ograniczenia technologiczne ówczesnych czasów.


  Wreszcie wraz z nadejściem XXI wieku postępujący rozwój
  komputerów oraz internetu umożliwił sztucznym sieciom neuronowym rozwinięcie
  skrzydeł. Wejście w erę "big data" otworzyło dostęp do olbrzymich zbiorów
  danych niezbędnych do treningu sieci, a pojawienie się wysokowydajnych
  jednostek obliczeniowych pozwoliło znacznie ten proces przyspieszyć.


  Zapoczątkowany w ten sposób rozwój trwa do dnia dzisiejszego.
  Sztuczne sieci neuronowe odnajdują zastosowanie w wielu dziedzinach życia i
  nauki. Grają w gry, przeprowadzają symulacje, przewidują i prognozują
  zachowania rynku czy pogody, a także analizują i przetwarzają obrazy cyfrowe.


  Z punktu widzenia niniejszej pracy największe znaczenie ma
  oczywiście ostatni z wymienionych punktów. Zdefiniowanie sieci neuronowych,
  jako matematycznych modeli obliczeniowych ujawnia ich naturalne predyspozycje
  do pracy na obrazach cyfrowych. W praktyce stanowią one bowiem zbiór liczb,
  wartości poszczególnych pikseli, który sieć neuronowa jest w stanie
  analizować, przetwarzać i modyfikować.

  \subsection{Cel pracy}
    Celem pracy jest stworzenie szeregu narzędzi programistyczych
    oferujących szeroki wachlarz możliwości edytowania obrazu. Narzędzia te
    oparte mają być na technologii sieci neuronowych. W szczególności
    przetestowana będzie skuteczność rozwiązań dedykowanych do przetwarzania
    obrazów, takich jak sieci konwolucyjne albo modele generatywne.

    Po opracowaniu tychże narzędzi, opisane zostaną efekty pracy oraz
    zbadana zostanie skuteczność sieci neuronowych jako rozwiązania nakreślonej
    problematyki. Omówione zostaną także wykorzystane architektury zaimplementowanych
    modeli, wykorzystane funkcje kosztu, metody aktualizowania wag sieci oraz
    przebiegi treningu modeli.

  \subsection{Dotychczasowe dokonania}
    \textless Syntetyczny opis dotychczasowych dokonań w danej tematyce?
    \textgreater

  \subsection{Założenia projektowe}
    Głównym założeniem pracy było zaprojektowanie i zaimplementowanie
    narzędzi programistycznych służących do edycji obrazu. Narzędzia te muszą
    wykorzystywać do swoich celów nauczone sieci neuronowe. Na podstawie jakości
    działania tychże narzędzi, oceniona zostanie ich rzetelność oraz skuteczność.

    Następnie przeprowadzona zostanie analiza słuszności zastosowania sieci
    neuronowych jako rozwiązania przedstawionej problematyki. Uzyskane
    rozwiązanie zostanie także zestawione ze znanymi algorytmami do edycji
    obrazu nie opierającymi się na technologii sieci neuronowych.

  \subsection{Układ pracy}
    W pierwszym rozdziale przedstawiono zarys rozwoju Sieci Neuronowych
    na przestrzeni lat oraz niezbędne podstawy teoretyczne związane
    z przedstawioną problematyką i wybranym dla niej rozwiązaniem.

    W kolejnym rozdziale dokonano przeglądu już istniejących rozwiązań, opisano
    ich przeznaczenie, sposób działania oraz uzyskane rezultaty.
    Oceniono także wpływ danego rozwiązania na rozwój sieci neuronowych w
    dziedzinie przetwarzania i edytowania obrazów.


\section{Przegląd rozwiązań}
  Sieci neuronowe, dzięki swoim cechom, znalazły wiele rzeczywistych zastosowań w produktach
  przemysłowych. W tym podrozdziale skupiono się na przedstawieniu istniejących
  rozwiązań rozważanej problematyki. \textless Do zredagowania jeszcze \textgreater

  \subsection{Neural photo editing}
    W 2017 roku Andrew Brock, Theodore Lim, J.M. Ritchie and Nick Weston
    zaprezenowali Neural Photo Editor \cite{neural_photo_editor}, narzędzie
    do edytowania obrazu wyposażone w mechanizmy wykrywania kontekstu zmiany.
    Twórcy opisują swój twór następująco:

    \begin{quote}
      'An interface that leverages the power of generative neural networks to
      make large, semantically coherent changes to existing images.'

      'Interfejs wykorzystujący moc generatywnych sieci neuronowych do
      wprowadzania dużych, semantycznie spójnych zmian w istniejących obrazach.'
    \end{quote}

    Użytkowanie wygląda następująco: użytkownik pędzlem o określonym kolorze i
    rozmiarze maluje na wybranym obrazie, jednak zamiast zmieniać wartości
    pojedynczych pikseli, interfejs odczytuje konteks edycji i wprowadza zmiany
    semantyczne w kontekście żądanej zmiany koloru. Efekt działania interfejsu
    został przedstawiony na Rysunku \ref{fig:npe}.

    \begin{figure}[h]
      \centering
      \includegraphics[width=4in]{NPE}
      \caption{Efekt działania Neural Photo Editor}
      \label{fig:npe}
    \end{figure}

    Skuteczność NPE (Neural Photo Editor) polega na zastosowaniu IAN
    (ang. Introspective Adversarial Network), czyli sieci złożonej z połączonych
    VAE (ang. Variational Autoencoder) oraz GAN, w taki sposób, żę dekodująca
    sieć autoenkodera jest używana jako sieć generująca w GAN.
    Poprzez przechwytywanie przez model dalekosiężnych zależności, wykorzystanie
    bloku obliczeniowego bazującego na rozszerzonych splotach o
    współdzielonych wagach oraz dzięki zastosowaniu ulepszonej generalizacji,
    udało się osiągnąć dokładną rekonstrukcje obrazu bez strat na jakości detali.


  \subsection{Colorful image colorization}

    Wraz z rozwojem sieci neuronowych, rosło zainteresowanie możliwościami zastosowania
    ich do kolorowania czarno-białych obrazów. Jedno z dostępnych rozwiązań tego
    zagadnienia zostało przedstawione przez grupę pracowników Uniwersytetu w
    Berkeley \cite{colorful_image_colorization}. Zamiarem ich pracy było stworzenie
    modelu, który niekoniecznie odtwarza oryginalne barwy obrazu, ale generuje
    barwy prawdopodbne, zdolne przekonać ludzkiego obserwatora o autentyczności
    obrazu. Uzyskane rezultaty zostały przedstawione na
    Rysunku \ref{fig:colorful_image_colorization}.

    \begin{figure}[h]
      \centering
      \includegraphics[width=4in]{image_colorization}
      \caption{Efekt działania Neural Photo Editor}
      \label{fig:colorful_image_colorization}
    \end{figure}

    Wykorzystany model składa się z wielu warst CNN, w których skład wchodzą
    warsta filtrów konwolucyjnych, warsta ReLU (ang. Rectified
    Linear Unit) oraz warstwa BatchNorm (ang. Batch normalization).
    Aby zapobiec utracie informacji przestrzennych, sieć nie posiada warst łączących.
    Istotny był także sposób
    przygotowania zbioru danych do trenowania modelu. Obrazy ze zbioru uczącego
    były wpier konwertowane do modelu YUV, a następnie kanał Y był podawany na
    wejście modelu, warstwy UV pełniły funkcję pożądnej odpowiedzi w uczeniu
    nadzorowanym.

    Ważnym aspktem zbadanym w artykule było także dobranie odpowiedniej
    funkcji kosztu. Nieodpowiedni wybór skutkował desaturacją kolorowanych
    obrazów, jedną z potencjalnych przyczyn tego zjawiska może być tendencja
    sieci do tworzenia bardziej konserwatwynych odpowiedzi. Aby zniwelować ten
    efekt w modelu została zastosowana specjalna technika modyfikacji
    funkcji kosztu. Polega ona na przewidywaniu dystrybucji możliwych kolorów
    dla każdego piksela i zmienianiu kosztu dla modelu, w celu wyróżnienia rzadko
    spotykanych kolorów.

  \subsection{Transforming photos to comics using convolutional neural networks}

  \subsection{A neural algorithm of artistic style}

  \subsection{Face App}


\section{Projekt Sieci Neuronowych}

\section{Podsumowanie}


\newpage
\begin{thebibliography}{99} %{99} numery co najwyżej dwucyfrowe
  \bibitem{gan} Ian J.~Goodfellow, Jean~Pouget-Abadie, Mehdi~Mirza, Bing~Xu, David~Warde-Farley, Sherjil~Ozair, Aaron~Courville, Yoshua~Bengio:
  \emph{Generative Adversarial Networks}, ('2014)
\end{thebibliography}

\section*{Załączniki}
\addcontentsline{toc}{section}{Załączniki}  % Dodanie to TOC
  \tab Załączniki i dodatki\:



\end{document}
