\documentclass[10pt, a4paper, twoside, notitlepage, openright]{article}

%--------------------------------------------------------------------

% language - polish by default
%--------------------------------------------------------------------
\usepackage[MeX]{polski}
\usepackage[T1]{fontenc}
\usepackage[utf8]{inputenc}
%--------------------------------------------------------------------

% basic font:
%--------------------------------------------------------------------
\usepackage[scaled]{helvet} %Zastępczy font dla Ariala
\usepackage{mathpazo} % Zalecany font dla równań matematycznych
\renewcommand{\familydefault}{\sfdefault}
%--------------------------------------------------------------------

% baseline:
%--------------------------------------------------------------------
\linespread{1.3} % factor 1.3 gives ,,one and a half'' line spacing
% W wymogach jest, że ma być  1.5, ale podobno ten końcowy line spacing
% wylicza się jako ten linespread * (baselineskip/fontsize) i w naszym przypadku
% to powinno być 1.25 * (12/10) = 1.5, no ale zobaczymy bo aktualnie ten parametr wygląda jakby nie działał
%--------------------------------------------------------------------

% margins:
%--------------------------------------------------------------------
% for margins left, right top bottom
\usepackage{anysize}
\marginsize{3.5cm}{2.5cm}{2.5cm}{2.5cm}
% \marginsize{3.5cm}{2.5cm}{2.5cm}{2.5cm} % Te 3.5 sprawia, że lepiej się będzie czytało wydrukowaną dwustronnie pracę
\let\origdoublepage\cleardoublepage    % komenda wstawiająca czyste kartki
\newcommand{\clearemptydoublepage}{%
  \clearpage
  {\pagestyle{empty}\origdoublepage}%
}
\let\cleardoublepage\clearemptydoublepage
%--------------------------------------------------------------------

% Do rysunków:
%--------------------------------------------------------------------
\usepackage{graphicx}
\graphicspath{ {./images/} }  % Domyślna ścieżka do obrazów
%--------------------------------------------------------------------

\usepackage{amsmath,amssymb}  % To jest chyba do formuł matematycznych
\usepackage{setspace,cite}

\usepackage{pdfpages} % paczka żeby móc importować gotowe strony z plików pdf do pdfa z latexa
\usepackage{titlesec} % może być potrzebne jeśli będziemy chcieli edytować tytuły



% Do Nagłówków:
%--------------------------------------------------------------------
% First {} is a font size for title of sections and second {} is baselineskip,
% that is size of empty space between section title and text below
\usepackage{fancyhdr}
\titleformat{\section}[block]
  {\normalfont \fontsize{12}{12} \bfseries}{\thesection.}{0.5em}{\fontsize{12}{12} \MakeUppercase }
\titlespacing{\section}{0pt}{12pt}{6pt}

\titleformat{\subsection}
      {\normalfont\fontsize{10}{10} \bfseries \itshape}{\thesubsection.}{0.5em}{\bfseries \itshape \itshape}
\titlespacing{\subsection}{0pt}{12pt}{6pt}

\titleformat{\subsubsection}
      {\normalfont\fontsize{10}{10}\itshape}{\thesubsubsection.}{0.5em}{\normalfont \itshape}
\titlespacing{\subsubsection}{0pt}{12pt}{6pt}
%--------------------------------------------------------------------

% Uwaga, nie dodawać biblioteki sectsty bo to zresetuje formatowanie

% Do Table of Content:
%--------------------------------------------------------------------
\usepackage{titletoc}
\titlecontents{section}
    [0em] % no indentation
    {}
    {\contentslabel[\thecontentslabel.]{1.5em}} % suitable only for one-digit and a comma
    {\hspace*{-1.5em}}                          % !
    {\dotfill\contentspage}

\titlecontents{subsection}
    [2em] % ie, 1.5em (section) + 0.5em
    {}
    {\contentslabel[\thecontentslabel.]{2em}}   %two digits and two dots
    {\hspace*{-2.3em}}                          % !
    {\dotfill\contentspage}

\titlecontents{subsubsection}
    [5em] % ie, 2em (subsection) + 3em
    {}
    {\contentslabel[\thecontentslabel.]{3em}}  %three digits and three dots
    {\hspace*{-2.3em}}                          % !
    {\dotfill\contentspage}

\usepackage{tocloft}  % żeby móc ręcznie dodawać sections do TOC (Table of Content)
\usepackage[nottoc]{tocbibind}  % definiuje co jest w TOC
% nottoc -  Disables the inclusion of the ToC.

% to powyższe można zastąpić 2 liniami:
% \usepackage{tocbibind}
% \tocbibind{nottoc}
%--------------------------------------------------------------------

% Do Indentów itp:
%--------------------------------------------------------------------
\usepackage{indentfirst}  % Ustawienia aby pierwszy paragraf w sekcji też miał akapit, bo domyślnie nie ma
\setlength{\parindent}{1.25cm}
\setlength{\parskip}{1ex} %odstępy między akapitami
%--------------------------------------------------------------------

% Do spisu rysunków i tabel:
%--------------------------------------------------------------------
\usepackage{chngcntr} % Do edycji numerowania rysunków w sekcjach
\counterwithin{figure}{section} % Sprawia, że numeracja rysunków będzie podzielona na sekcje
\counterwithin{table}{section} % Sprawia, że numeracja tabel będzie podzielona na sekcje
%--------------------------------------------------------------------

\usepackage{csquotes} % Do cytowania
\AtBeginEnvironment{quote}{\itshape}  % Ustawia to co jest między \begin{quote}
% a \end{quote}, że będzie kursywą

\usepackage[hyphens]{url} % Dodatkowa opcja, żeby linki url łamały się jeśli są
% za długie
\usepackage{hyperref} % Do linków url

\usepackage{array} %Umożliwia definiowanie własnych wymiarów tabel
\usepackage{enumitem} % Do usunięcia indenta dla begin{itemize}
% --------------------------------------------------------------------

% Do odpowiednich itemize i enumerate:
%--------------------------------------------------------------------
% \setenumerate{
% %     labelindent=0em,
% %     itemindent=\indentenit,
%     topsep=1.5\parskip,
%     itemsep=1.5\parskip,
%     partopsep=10pt,
% %     align=left
%     }

\setitemize{
%     labelindent=0em,
%     itemindent=\indentenit,
    itemsep=0pt,
    topsep=0pt,
%     partopsep=0pt,
%     align=left
}

\RequirePackage{textcomp} % for \textbullet and \textopenbullet only
\setitemize[1]{label=\textbullet}   % first level \textbullet
\setitemize[2]{label=\textopenbullet}     % second level \textopenbullet
%--------------------------------------------------------------------

\usepackage{longtable} %Umożliwia definiowanie tabel obejmujących wiele stron

\usepackage{float}
\usepackage[caption = false]{subfig}
\usepackage[font=small,labelfont=bf,justification=centering]{caption}  % Umożliwia zdefiniowanie, że podpis pod rysunkiem ma być np wyśrodkowany
\usepackage[figuresright]{rotating}

% \usepackage{multirow}


%\renewcommand{\chaptername}{Rozdział}
%\renewcommand{\contentsname}{Spis treści}
% \renewcommand{\figurename}{Rys.}
%\renewcommand{\tablename}{Tab.}
\renewcommand{\listfigurename}{SPIS RYSUNKÓW}
\renewcommand{\listtablename}{SPIS TABEL}
\renewcommand{\bibname}{WYKAZ LITERATURY}

% Chyba lepiej używać latexowego \indent zamiast definiować własny akapit
% \newcommand\tab[1][1cm]{\hspace*{#1}}
% \newcommand\tab{\indent}

% \bibliographystyle{unsrt} % Powinno niby ułożyć dobrze bibliografię, ale nie działa

% pagination:
%--------------------------------------------------------------------
\pagenumbering{arabic}

% \RequirePackage{fancyhdr}

% \fancypagestyle{plain}{
%   \fancyhf{}
%   \fancyfoot[C]{\fontsize{9pt}{9pt}\selectfont\thepage}%
%    \renewcommand{\headrulewidth}{0.0pt}
% }

\pagestyle{plain}
%--------------------------------------------------------------------

\widowpenalties 1 10000 % żeby jak akapit się nie zmieści cały na danej stronie, to żeby był przeniesiony w pełni na kolejną stonę - kwestie estetyczne
\raggedbottom % Nie wiem w sumie co to robi, ale może jest potrzebne do czegoś


\begin{document}

% \parskip 1.5ex % paragraph spacing - zwiększa odstępy pomiędzy paragrafami

% \baselineskip=15pt  % odległość między liniami chyba


% Strone tytulowa trzeba bedzie pobrac z moja pg
\includepdf[pages=-]{StronaTytulowa_bb.pdf} %pages=- to załączenie wszystkich stron
\includepdf[pages=-]{Oswiadczenie_PW.pdf}


% \begin{titlepage}
%   \begin{center}
%
%     \vspace*{1cm}
%     \Huge
%     \textbf{POLITECHNIKA GDAŃSKA}
%     \newline
%     \vspace{0.5cm}
%     \LARGE
%     \newline
%     Katedra Systemów Decyzyjnych i Robotyki
%
%     \vspace{1.5cm}
%     \textbf{PRACA INŻYNIERSKA}
%     \\[0.5cm]
%     \textbf{Zastosowanie sieci neuronowych do edycji obrazów}
%
%     \vspace{2.5cm}
%     \Large
%     \textbf{Autorzy}\\
%     Piotr Winkler\\
%     Bartosz Bieliński
%
%     \vspace{3.5cm}
%     Gdańsk 2019
%
%   \end{center}
% \end{titlepage}

\setcounter{page}{2}

\section*{Streszczenie}

  Tematem pracy jest zbadanie możliwości zastosowania sieci neuronowych do
  edycji obrazów. Głównym celem było stworzenie narzędzi opartych na wyuczonych
  sieciach neuronowych, służących do odpowiedniego przetwarzania i
  modyfikowania obrazu. Następnie skuteczność tych narzędzi została oceniona i
  porównana z rozwiązaniami opartymi na klasycznych metodach przetwarzania obrazu.

  Zbadane zostały dwie problematyki, automatyczne kolorowanie czarno-białych obrazów
  oraz nakładanie na obraz prostych filtrów takich jak wykrywający
  krawędzie filtr Sobela-Feldmana. Opisy zaimplementowanych rozwiązań zostały
  umieszczone w odpowiednich podrozdziałach
  rozdziału \ref{zaimplementowane rozwiazania}.

  Uzyskane wyniki pozwoliły dojść do wielu konstruktywnych obserwacji. W przypadku obu
  problematyk udało się osiągnąć rozwiązania oparte na sztucznych sieciach
  neuronowych.

  W celu ułatwienia przeprowadzania badań został opracowany autorski framework
  o nazwie \textit{TorchFrame}. Ma on za zadanie kontrolować
  przepływ danych w procesie uczenia i testowania badanych modeli, a co za tym
  idzie, ograniczyć konieczność ingerencji ze strony użytkownika, przy jednoczesnym
  zapewnieniu swobody w prowadzonych eksperymentach i implementowanych modyfikacjach danych. Przekłada
  się to na znacznie wydajniejszy oraz mniej złożony proces trenowania
  różnorodnych architektur sieci neuronowych.

  W przypadku prostych filtrów obrazu rozwiązania
  oparte na sieciach splotowych okazały się wolniejsze i bardziej pracochłonne w
  implementacji niż metody klasyczne przy porównywalnych rezultatach. Jednakże
  z ich pomocą udało się udowodnić olbrzymią uniwersalność sieci splotowych
  mogących znaleźć zastosowanie w zagadnieniach związanych z prostym
  przetwarzaniem obrazu. Ukazane zostało ponadto podobieństwo między tradycyjnymi
  metodami filtracji, a sposobem działania neuronów tworzących sieci konwolucyjne.

  Zagadnienie automatycznego kolorowania czarno-białych
  obrazów zostało rozwiązane z użyciem dwóch różnych modeli, autorskiego modelu
  prostego oraz bardziej zaawansowanego modelu zaimplementowanego z użyciem technik
  przeniesienia uczenia. Dla modelu autorskiego zostały przeprowadzone szczegółowe
  badania dotyczące zależności wyników od poszczególnych parametrów architekury
  sieci jak i konfiguracji procesu uczenia. Rezultaty końcowe uzyskane z użyciem
  modelu autorskiego były zadowalające, ale mocno zależne od wybranych parametrów.
  Dowodzi to, że odpowiednio skonfigurowane sieci splotowe mogą być skutecznie
  zastosowane w tym zagadnieniu, aczkolwiek to model złożony pozwolił osiągnąć
  największy sukces.
  Idea tego rozwiązania opiera się na integracji cech obrazu średniego oraz wysokiego
  poziomu uzyskiwanych z użyciem złożonej sieci splotowej wytrenowanej pierwotnie do
  zadania klasyfikacji. Cechy te, po przejściu przez proces fuzji, są następnie
  wykorzystywane przez sieć dekonwolucyjną do predykcji prawdopodobnych barw dla
  obrazu wejściowego. Dzięki tym dodatkowym informacjom o obrazie udało się
  znacznie zwiększyć efektywność procesu kolorowania, a co za tym idzie,
  wiarygodność generowanych barw.

  Badania przeprowadzone w ramach tej pracy dyplomowej są dowodem na to, że
  sieci neuronowe mogą być skutecznie zastosowane jako narzędzia do wszechstronnej
  edycji obrazu. Często są one jedynym dostępnym rozwiązaniem jeśli problematyka
  jest nadzwyczaj złożona. Jednakże, pomimo niezwykłych możliwości sieci
  neuronowych, ich sukces zależy w dużej mierze od dobrze przemyślanego
  wyboru stosowanej architektury oraz poprawnie przeprowadzonego
  procesu uczenia.

  \bigskip

  \noindent\textbf{Słowa kluczowe:} sieć neuronowa, przetwarzanie obrazu,
  konwolucyjna sieć neuronowa, splotowa sieć neuronowa,
  głęboka sieć neuronowa, filtry obrazu, automatyczne kolorowanie czarno-białych
  obrazów, platforma programistyczna

  \bigskip

  \noindent\textbf{Dziedzina nauki i techniki zgodna z OECD:} Nauki
  inżynieryjne i techniczne, Elektrotechnika, elektronika, inżynieria
  informatyczna, Sprzęt komputerowy i architektura komputerów

\section*{Abstract}

  The subject of this project is to explore the possibilities of using neural networks for
  image editing. The main goal was to create tools, based on
  trained neural networks, used for proper processing and modifying of
  images. Then the effectiveness of these tools was evaluated and
  compared with solutions based on classic image processing methods.

  Two approaches were investigated, automatic coloring of black and white images
  and applying simple filters on images such as Sobel-Feldman filter detecting edges.

  To make the research easier, an original framework named TorchFrame has been created.
  It has the task of controlling
  data flow in the process of teaching and testing the examined models and hence
  reduce the need for user interference while ensuring freedom
  in conducted experiments and implemented data modifications.

  For simple image filters, solutions based on convolutional neural networks proved to
  be slower and more labor-intensive in implementation than classical methods,
  while presenting comparable results.
  However, with their help it was possible to prove the enormous universality
  of convolutional neural networks, which can be used in issues related to
  simple image processing. Furthermore, the similarity between traditional filtration
  methods and the way, how the neurons in convolutional networks works, has been shown.

  To solve the problem of black and white images colorization, two different
  approaches were proposed. The first one included the application of simple author's
  model trained with many different hiperparameters and data preparation methods.
  Obtained results were satisfactory but they are greatly dependent from chosen
  hiperparameters and training configuration.

  The second approach included implementing more complex model using transfer
  learning technique. This approach was based on the integration of medium
  and high level features of the image. The information obtained that way was
  then used to predict possible colors of the image.
  At the end, this method resulted in more realistic colors and hence
  highly satisfying results.

  Experiments, carried out as part of this project, proves that neural networks can
  be successfully used as tools in general image editing process. However, despite
  their remarkable possibilities, success highly depends on the well considered choice of used architecture
  and correctly carried out training process.

  \bigskip

  \noindent\textbf{Keywords:} neural network, image processing, convolutional
  neural network, generative adversarial network, deep neural network,
  image filters, automatic image colorizing, framework

  \bigskip

  \noindent\textbf{Field of science and technology in accordance with the
  requirements of the OECD:} Engineering and technology, Electrical engineering,
  Electronic engineering, Information engineering, Computer hardware and
  architecture


\newpage
  \tableofcontents

\section*{WYKAZ WAŻNIEJSZYCH OZNACZEŃ I SKRÓTÓW}
\addcontentsline{toc}{section}{WYKAZ WAŻNIEJSZYCH OZNACZEŃ I SKRÓTÓW} % Dodanie to TOC

  \bigskip

  \begin{itemize}
    \item[GAN] (ang. generative adversarial network) - sieci o modelu generatywnym
    \item[ANN] (ang. artificial neural network) - sztuczna sieć neuronowa
    \item[DNN] (ang. deep neural network) - głęboka sieć neuronowa
    \item[CNN] (ang. convolutional neural network) - splotowa sieć neuronowa
    \item[FCL] (ang. fully connected layer) - warstwa gęsta
    \item[VAE] (ang. Variational Autoencoder) - utoenkodery wariacyjne
  \end{itemize}


\section{Wstęp i cel pracy}
  Sztuczne sieci neuronowe sięgają swym początkiem lat 40. XX wieku.
  Historia ich rozwoju odnotowała trzy okresy, w których rozwiązania te
  odbijały się szerokim echem w środowisku naukowym.

  Pierwszy model neuronu, a potem perceptron zapoczątkowały
  rozwój tej dziedziny nauki, jednak pierwsze sieci jednowarstwowe nie były w
  stanie rozwiązywać złożonych problemów. Przeszkodę nie do pokonania stanowiła
  dla nich nawet prosta funkcja logiczna XOR. Z tego powodu badania sieci
  neuronowych zostały na długi czas porzucone.


  Pojawienie się algorytmu wstecznej propagacji błędów
  pozwalającego skutecznie uczyć wielowarstwowe sieci neuronowe ponownie
  wzmogło zainteresowanie tematem, jednak tym razem na drodze postępowi stanęły
  ograniczenia technologiczne ówczesnych czasów.


  Wreszcie wraz z nadejściem XXI wieku postępujący rozwój
  komputerów oraz internetu umożliwił sztucznym sieciom neuronowym rozwinięcie
  skrzydeł. Wejście w erę "big data" otworzyło dostęp do olbrzymich zbiorów
  danych niezbędnych do treningu sieci, a pojawienie się wysokowydajnych
  jednostek obliczeniowych pozwoliło znacznie ten proces przyspieszyć.


  Zapoczątkowany w ten sposób rozwój trwa do dnia dzisiejszego.
  Sztuczne sieci neuronowe odnajdują zastosowanie w wielu dziedzinach życia i
  nauki. Grają w gry, przeprowadzają symulacje, przewidują i prognozują
  zachowania rynku czy pogody, a także analizują i przetwarzają obrazy cyfrowe.


  Z punktu widzenia niniejszej pracy największe znaczenie ma
  oczywiście ostatni z wymienionych punktów. Zdefiniowanie sieci neuronowych,
  jako matematycznych modeli obliczeniowych ujawnia ich naturalne predyspozycje
  do pracy na obrazach cyfrowych. W praktyce stanowią one bowiem zbiór liczb,
  wartości poszczególnych pikseli, który sieć neuronowa jest w stanie
  analizować, przetwarzać i modyfikować.

  \subsection{Cel pracy}
    Celem pracy jest stworzenie szeregu narzędzi programistyczych
    oferujących szeroki wachlarz możliwości edytowania obrazu. Narzędzia te
    oparte mają być na technologii sieci neuronowych. W szczególności
    przetestowana będzie skuteczność rozwiązań dedykowanych do przetwarzania
    obrazów, takich jak sieci konwolucyjne albo modele generatywne.

    Po opracowaniu tychże narzędzi, opisane zostaną efekty pracy oraz
    zbadana zostanie skuteczność sieci neuronowych jako rozwiązania nakreślonej
    problematyki. Omówione zostaną także wykorzystane architektury zaimplementowanych
    modeli, wykorzystane funkcje kosztu, metody aktualizowania wag sieci oraz
    przebiegi treningu modeli.

  \subsection{Dotychczasowe dokonania}
    \textless Syntetyczny opis dotychczasowych dokonań w danej tematyce?
    \textgreater

  \subsection{Założenia projektowe}
    Głównym założeniem pracy było zaprojektowanie i zaimplementowanie
    narzędzi programistycznych służących do edycji obrazu. Narzędzia te muszą
    wykorzystywać do swoich celów nauczone sieci neuronowe. Na podstawie jakości
    działania tychże narzędzi, oceniona zostanie ich rzetelność oraz skuteczność.

    Następnie przeprowadzona zostanie analiza słuszności zastosowania sieci
    neuronowych jako rozwiązania przedstawionej problematyki. Uzyskane
    rozwiązanie zostanie także zestawione ze znanymi algorytmami do edycji
    obrazu nie opierającymi się na technologii sieci neuronowych.

  \subsection{Układ pracy}
    W pierwszym rozdziale przedstawiono zarys rozwoju Sieci Neuronowych
    na przestrzeni lat oraz niezbędne podstawy teoretyczne związane
    z przedstawioną problematyką i wybranym dla niej rozwiązaniem.

    W kolejnym rozdziale dokonano przeglądu już istniejących rozwiązań, opisano
    ich przeznaczenie, sposób działania oraz uzyskane rezultaty.
    Oceniono także wpływ danego rozwiązania na rozwój sieci neuronowych w
    dziedzinie przetwarzania i edytowania obrazów.


\section{Podstawy teoretyczne}

  W dzisiejszych czasach sieci neuronowe zajmują ważną pozycję na rynku narzędzi
  do edycji obrazu. Jest to głównie spowodowane ich umiejętnością do
  reprodukowania i modelowania nieliniowych procesów, a także nowoczesnymi
  technikami przetwarzania plików graficznych.
  Jednak pierwsze architektury ANN (ang. artificial neural network) nie nadawały
  się do przetwarzania grafik.
  Było to częściwo spowodowane faktem, że obrazy, będące w rzeczywistości macierzami
  wartości pikseli,
  % należało przekształcić w długie wektory liczbowe aby móc podać je
  ciężko było skutecznie podać
  na wejscie typowych architekur DNN (ang. deep neural network) zbudowanych
  pierwotnie z wielu warst ukrytych, pomiędzy którymi połączenia są na zasadzie
  każdy z każdym oraz mają swoje wagi podlegające modyfikacji w trakcie procesu
  uczenia.
  Obrazy o niskiej rozdzielczości można było przekształcić w wektory
  wartości poszczególnych pikseli i w takiej postaci podawać na wejście sieci,
  jednak w przypadku obrazów o wyższej rozdzieloczości to rozwiązanie, ze
  względu na znaczną długość powstałych wektorów, nie oferowało dobrych
  rezultatów.
  Typowa stuktura DNN została przedstawiona na Rysunku \ref{fig:dnn}.
  Dopiero nowe architektury sieci spowodowały przełom w tej dziedzienie.

  \begin{figure}[h]
    \centering
    \includegraphics[width=4in]{dnn}
    \caption{Struktura DNN}
    \label{fig:dnn}
  \end{figure}

  Jedną z takich przełomowych architektur jest CNN (ang. convolutional neural
  network). Są to sieci o hierarchicznej strukturze, gdzie obrazy wejściowe w
  postaci macierzy wpier poddawane są ekstrakcji cech poprzez dokonanie operacji
  konwolucji na obrazie poprzez przesuwanie zestawu filtrów wzdłuż obrazu.
  Filtry te są zazywczaj inicjowane losowymi wartościami i w miarę trenowania,
  dopasowują swoje parametry do wybranej problematyki.
  Na wyjściu filtrów otrzymuje się macierze o mniejszej rozdzielczości
  reprezentujące wyniki operacji konwolucji w danym punkcie. Otrzymane macierze
  wyjściowe mogą być podawane na kolejne warstwy konwolucyjne w celu ekstrakcji
  kolejnych cech. Dzięki temu procesowi kolejne warstwy filtrów uczą się
  rozpoznawać kluczowe cechy na obrazie, od drobnych elemtów takich jak jak
  krawędzie albo kształty po bardziej złożone takie jak części ciała albo
  przedmioty.
  Końcowe warstwy dokonują spłaszczania, czyli zamieniania wielowymiarowych
  macierzy cech na jednowymiarowe wektore, które podawane są na wejście
  FCL (ang. fully connected layer).



  \subsection{Zastosowanie}
    Sieci neuronowe, dzięki swoim cechom, znalazły wiele zastosowań w produktach
    przemysłowych. W tym podrozdziale skupiono się na przedstawieniu istniejących
    rozwiązań rozważanej problematyki. \textless Do zredagowania jeszcze \textgreater

    \subsubsection{Neural Photo Editing}
      W 2017 roku Andrew Brock, Theodore Lim, J.M. Ritchie and Nick Weston
      zaprezenowali Neural Photo Editor \cite{neural_photo_editor}, narzędzie
      do edytowania obrazu wyposażone w mechanizmy wykrywania kontekstu zmiany.
      Twórcy opisują swój twór następująco:

      \begin{quote}
        'An interface that leverages the power of generative neural networks to
        make large, semantically coherent changes to existing images.'
      \end{quote}

      Użytkowanie wygląda następująco: użytkownik pędzlem o określonym kolorze i
      rozmiarze maluje na wybranym obrazie, jednak zamiast zmieniać wartości
      pojedynczych pikseli, interfejs odczytuje konteks edycji i wprowadza zmiany
      semantyczne w kontekście żądanej zmiany koloru. Efekt działania interfejsu
      został przedstawiony na Rysunku \ref{fig:npe}.

      \begin{figure}[h]
        \centering
        \includegraphics[width=4in]{NPE}
        \caption{Efekt działania Neural Photo Editor}
        \label{fig:npe}
      \end{figure}

      Skuteczność NPE (Neural Photo Editor) polega na zastosowaniu IAN
      (ang. Introspective Adversarial Network), czyli sieci złożonej z połączonych
      VAE (ang. Variational Autoencoder) oraz GAN, w taki sposób, żę dekodująca
      sieć autoenkodera jest używana jako sieć generująca w GAN.
      Poprzez przechwytywanie przez model dalekosiężnych zależności, wykorzystanie
      bloku obliczeniowego bazującego na rozszerzonych splotach o
      współdzielonych wagach oraz dzięki zastosowaniu ulepszonej generalizacji,
      udało się osiągnąć dokładną rekonstrukcje obrazu bez strat na jakości detali.


    \subsubsection{TRANSFORMING PHOTOS TO COMICS USING CONVOLUTIONAL NEURAL NETWORKS}

    \subsubsection{Face App}

    \subsubsection{Coloring black and white world using Deep Neural Nets}

    \subsubsection{A Neural Algorithm of Artistic Style}


\section{Przegląd rozwiązań}
  Sieci neuronowe, dzięki swoim cechom, znalazły wiele rzeczywistych zastosowań w produktach
  przemysłowych. W tym podrozdziale skupiono się na przedstawieniu istniejących
  rozwiązań rozważanej problematyki. \textless Do zredagowania jeszcze \textgreater

  \subsection{Neural photo editing}
    W 2017 roku Andrew Brock, Theodore Lim, J.M. Ritchie and Nick Weston
    zaprezenowali Neural Photo Editor \cite{neural_photo_editor}, narzędzie
    do edytowania obrazu wyposażone w mechanizmy wykrywania kontekstu zmiany.
    Twórcy opisują swój twór następująco:

    \begin{quote}
      'An interface that leverages the power of generative neural networks to
      make large, semantically coherent changes to existing images.'

      'Interfejs wykorzystujący moc generatywnych sieci neuronowych do
      wprowadzania dużych, semantycznie spójnych zmian w istniejących obrazach.'
    \end{quote}

    Użytkowanie wygląda następująco: użytkownik pędzlem o określonym kolorze i
    rozmiarze maluje na wybranym obrazie, jednak zamiast zmieniać wartości
    pojedynczych pikseli, interfejs odczytuje konteks edycji i wprowadza zmiany
    semantyczne w kontekście żądanej zmiany koloru. Efekt działania interfejsu
    został przedstawiony na Rysunku \ref{fig:npe}.

    \begin{figure}[h]
      \centering
      \includegraphics[width=4in]{NPE}
      \caption{Efekt działania Neural Photo Editor}
      \label{fig:npe}
    \end{figure}

    Skuteczność NPE (Neural Photo Editor) polega na zastosowaniu IAN
    (ang. Introspective Adversarial Network), czyli sieci złożonej z połączonych
    VAE (ang. Variational Autoencoder) oraz GAN, w taki sposób, żę dekodująca
    sieć autoenkodera jest używana jako sieć generująca w GAN.
    Poprzez przechwytywanie przez model dalekosiężnych zależności, wykorzystanie
    bloku obliczeniowego bazującego na rozszerzonych splotach o
    współdzielonych wagach oraz dzięki zastosowaniu ulepszonej generalizacji,
    udało się osiągnąć dokładną rekonstrukcje obrazu bez strat na jakości detali.


  \subsection{Colorful image colorization}

    Wraz z rozwojem sieci neuronowych, rosło zainteresowanie możliwościami zastosowania
    ich do kolorowania czarno-białych obrazów. Jedno z dostępnych rozwiązań tego
    zagadnienia zostało przedstawione przez grupę pracowników Uniwersytetu w
    Berkeley \cite{colorful_image_colorization}. Zamiarem ich pracy było stworzenie
    modelu, który niekoniecznie odtwarza oryginalne barwy obrazu, ale generuje
    barwy prawdopodbne, zdolne przekonać ludzkiego obserwatora o autentyczności
    obrazu. Uzyskane rezultaty zostały przedstawione na
    Rysunku \ref{fig:colorful_image_colorization}.

    \begin{figure}[h]
      \centering
      \includegraphics[width=4in]{image_colorization}
      \caption{Efekt działania Neural Photo Editor}
      \label{fig:colorful_image_colorization}
    \end{figure}

    Wykorzystany model składa się z wielu warst CNN, w których skład wchodzą
    warsta filtrów konwolucyjnych, warsta ReLU (ang. Rectified
    Linear Unit) oraz warstwa BatchNorm (ang. Batch normalization).
    Aby zapobiec utracie informacji przestrzennych, sieć nie posiada warst łączących.
    Istotny był także sposób
    przygotowania zbioru danych do trenowania modelu. Obrazy ze zbioru uczącego
    były wpier konwertowane do modelu YUV, a następnie kanał Y był podawany na
    wejście modelu, warstwy UV pełniły funkcję pożądnej odpowiedzi w uczeniu
    nadzorowanym.

    Ważnym aspktem zbadanym w artykule było także dobranie odpowiedniej
    funkcji kosztu. Nieodpowiedni wybór skutkował desaturacją kolorowanych
    obrazów, jedną z potencjalnych przyczyn tego zjawiska może być tendencja
    sieci do tworzenia bardziej konserwatwynych odpowiedzi. Aby zniwelować ten
    efekt w modelu została zastosowana specjalna technika modyfikacji
    funkcji kosztu. Polega ona na przewidywaniu dystrybucji możliwych kolorów
    dla każdego piksela i zmienianiu kosztu dla modelu, w celu wyróżnienia rzadko
    spotykanych kolorów.

  \subsection{Transforming photos to comics using convolutional neural networks}

  \subsection{A neural algorithm of artistic style}

  \subsection{Face App}


\section{Zaimplementowane rozwiązania}  \label{zaimplementowane rozwiazania}

  W celu zbadania skuteczności sieci neuronowych, jako narzędzi do edycji obrazu,
  należało wybrać przykładowe zagadnienia z tej dziedziny, rozwiązać je z
  użyciem technik sztucznej inteligencji oraz ocenić ich skuteczność.

  Zadania te zostały zrealizowane, a szczegółowe opisy rozwiązań umieszczone w
  następującym rozdziale.

  W pierwszej kolejności zaprezentowany został autorski framework (pol. platforma
  programistyczna) \textit{TorchFrame},
  stworzony w języku programistycznym Python i przeznaczony do przyspieszenia i uproszczenia
  pracy z sieciami neuronowymi. Z jego pomocą wykonane zostały modele przeznaczone
  do różnego rodzaju filtrowania i przetwarzania obrazów. Rozwiązania te
  skupiły się na rozległej tematyce, poczynając od prostych filtrów demonstrujących
  działanie sztucznych sieci splotowych, a kończąc na złożonym modelu przeznaczonym do
  inteligentnego kolorowania czarno-białych obrazów.

  W ramach przedstawionych rozwiązań zamieszczone zostały porównania wyników
  osiągniętych rozmaitymi metodami treningowymi wraz z rozważaniami teoretycznymi
  związanymi z zastosowaniem różnorodnych hiperparametrów w procesie uczenia sieci.
  Ponadto, w ramach kolejnych podrozdziałów, opisane zostały istotne
  elementy składające się na proces projektowania i tworzenia sztucznych sieci
  neuronowych.

  \subsection[TorchFrame (Piotr Winkler)]{TorchFrame} \label{TorchFrame}

  Sztuczne sieci neuronowe stanowią dziedzinę nauki opartą w dużej mierze na
  eksperymentach. Dostarczane przez nie rozwiązania, choć często tak spektakularne,
  są mocno zawoalowane, a droga do celu wiedzie przez kolejne treningi
  i doświadczalny dobór hiperparametrów sieci. Ogromne znaczenie w procesie uczenia
  ma również sposób przetworzenia danych wykorzystywanych do treningów, zarówno
  tych podawanych na wejście sieci, jak i tych otrzymywanych na jej wyjściu.

  Niniejsza praca dotyczy przede wszystkim zastosowania sieci neuronowych w
  procesie przetwarzania obrazów cyfrowych. Wiąże się to z koniecznością przygotowywania
  zbiorów treningowych złożonych z ogromnej ilości danych wizyjnych poddanych
  odpowiedniemu przetworzeniu i filtracji, aby mogły właściwie spełnić swoją rolę
  w czasie treningu sieci.

  Wspomniane zabiegi, jak również konieczność częstego powtarzania treningów,
  wymagają dużych nakładów pracy i czasu, aby mogły przynieść zamierzone efekty.
  W celu ułatwienia całego procesu przygotowany został framework TorchFrame
  stanowiący bazę pod eksperymenty podejmowane w ramach tej pracy i opisane w
  dalszych jej rozdziałach.

  Sam framework umożliwia użytkownikom przeprowadzanie treningów sieci neuronowych,
  udostępniając wachlarz modyfikowalnych hiperparametrów oraz zestaw filtrów i metod
  przeznaczonych do obróbki danych treningowych. Właściwie użyty TorchFrame kontroluje
  przepływ danych uczących od początku do końca ograniczając konieczność ingerencji ze
  strony użytkownika do minimum, jednocześnie nie ograniczając przy tym potencjału
  eksperymentalnego sztucznych sieci neuronowych.

  W ramach frameworka udostępniony został również prosty interfejs testowy
  umożliwiający ocenę efektów uzyskanych w procesie uczenia.

  Niniejszy rozdział zostanie poświęcony analizie architektury TorchFrame'a oraz
  opisowi sposobu jego działania.

  \subsubsection{PyTorch}

    U podstaw TorchFrame'a leżą mechanizmy innego frameworka, napisanego w języku Python i
    przeznaczonego do uczenia maszynowego o nazwie PyTorch. Jest to otwartoźródłowa
    biblioteka programistyczna stworzona przez oddział sztucznej inteligencji firmy
    Facebook. W jednym z artykułów \cite{pytorch} opublikowanych w ramach konferencji NIPS 2017
    grupa badaczy opisuje ją następująco:

    \begin{quote}
      'PyTorch - biblioteka zaprojektowana w celu umożliwienia szybkiego badania
      modeli uczenia maszynowego. Bazuje na kilku projektach, głównie Lua Torch,
      Chainer i HIPS Autograd, oraz dostarcza wysoko wydajnościowe środowisko z
      łatwym dostępem do automatycznego różnicowania modeli wykonywanych na różnych
      urządzeniach (CPU i GPU). Aby uczynić prototypowanie łatwiejszym, PyTorch
      nie podąża za podejściem symbolicznym używanym w wielu innych frameworkach
      do uczenia głębokiego, ale skupia się na różnicowaniu czysto imperatywnych
      programów, skupiając się na rozszerzalności i małym narzucie.' (...)

      'PyTorch, podobnie jak większość innych bibliotek do uczenia głębokiego obsługuje
      automatyczne różnicowanie funkcji skalarnych w trybie wstecznym, czyli jedną z najważniejszych form
      automatycznego różnicowania dla aplikacji uczenia głębokiego, które zwykle
      różnicują skalarną funkcję celu.'
    \end{quote}

    TorchFrame wykorzystuje zdefiniowane w ramach PyTorch'a mechanizmy budowania sieci neuronowych
    korzystając z predefiniowanych metod opisu warstw sieci. Wykorzystuje ponadto
    zdefiniowane odgórnie hiperparametry, takie jak funkcje kosztu. Sam rdzeń TorchFrame'a
    bazuje na przytoczonym mechanizmie różnicowania, umożliwiając przeprowadzanie treningów
    na CPU oraz GPU.

  \subsubsection{Architektura TorchFrame}

    TorchFrame podzielony został na bloki funkcjonalne, które powiązane ze sobą
    umożliwiają kontrolowany przepływ danych w procesie uczenia, a jednocześnie
    zapewniają elastyczność przy wprowadzaniu rozmaitych modyfikacji. Schemat funkcjonalny
    całego systemu przedstawiony został na Rysunku \ref{fig:torchframetrain}.

      \begin{figure}[h!]
        \centering
        \includegraphics[width=5.5in]{torchframetrain}
        \caption[Architektura TorchFrame - źródło: Praca własna]{Architektura TorchFrame}
        \label{fig:torchframetrain}
      \end{figure}

    Przepływ danych przez framework rozpoczyna się w obrębie trzech plików konfiguracyjnych,
    oznaczonych na przytoczonym schemacie odpowiednio jako: \textit{consts}, \textit{.json config}
    oraz \textit{NN model}.

    W pliku \textit{consts} użytkownik TorchFrame definiuje zmienne środowiskowe, takie, jak
    ścieżki do plików kluczowych w procesie uczenia. Konieczne jest jedynie wskazanie
    położenia pliku konfiguracyjnego w formacie JSON. Pozostałe parametry są opcjonalne
    i mogą służyć do wskazania miejsca zapisu gotowego modelu sieci na dysku czy miejsca
    przechowywania danych treningowych, które zostaną następnie przetworzone i przekazane
    na wejście sieci w procesie uczenia.

    \textit{NN model} to skrypt języka Python, w którym użytkownik definiuje strukturę sieci
    neuronowej zgodnie z paradygmatem frameworka PyTorch przytoczonego w poprzednim
    rozdziale. Struktura TorchFrame narzuca na użytkownika konieczność zdefiniowania
    metody \textit{forward}, która określa sposób obliczania wartości wyjściowych sieci na podstawie danych
    wejściowych między innymi poprzez określenie funkcji aktywacji poszczególnych warstw
    sztucznych neuronów.

    Najważniejszym punktem zestawu konfiguracyjnego jest wspomniany już plik JSON.
    Udostępnia on szerokie możliwości manipulowania zarówno hiperparametrami sieci
    neuronowej, jak i szeregiem przekształceń możliwych do zaimplementowania
    na danych treningowych i testowych. Architektura TorchFrame zapewnia odpowiednie
    rozpropagowanie zgromadzonych tutaj danych w ramach procesu uczenia oraz podczas
    testów. Dokładna struktura tego pliku zostanie opisana w następnym rozdziale.

    \textit{JsonParser} jest klasą odpowiedzialną za odczytanie danych z pliku konfiguracyjnego,
    w formie słownika języka Python, oraz przekazanie ich w dalszą drogę w obrębie TorchFrame.

    Centralnym punktem całej zaprezentowanej architektury jest \textit{entrypoint}. Stanowi on
    punkt wejścia dla aplikacji użytkownika. Jego uruchomienie powoduje
    agregację danych ze wspomnianych już plików konfiguracyjnych, pogrupowanie ich
    oraz rozpropagowanie do dalszych komponentów odpowiedzialnych za przetwarzanie
    danych treningowych oraz przeprowadzanie samego treningu sieci neuronowej.

    Część danych przekazywanych przez \textit{entrypoint} trafia do \textit{SetupCreator'a}. Komponent
    ten odpowiada za skomponowanie listy konwersji wymienionych w pliku
    konfiguracyjnym. Konwersje te zostaną następnie zaimplementowane na danych treningowych
    na etapie tworzenia zbioru uczącego. Źródłem danych do którego odwołuje się
    \textit{SetupCreator} są skrypty \textit{transforms} oraz \textit{conversions}. Zawierają one gotową bazę
    przekształceń przeznaczonych do pracy na obrazach cyfrowych, a także metody
    formatowania danych do struktury \textit{tensorów} (przypominających macierze z biblioteki \textit{numpy}
    języka Python) wykorzystywanych przez bibliotekę PyTorch między innymi w mechanizmach
    automatycznego różnicowania modeli sieci neuronowych. Choć gotowa baza oferuje
    liczne przekształcenia, intuicyjna formuła pozwala użytkownikom w łatwy sposób definiować
    własne metody konwersji i implementować je zarówno na obrazach, jak i dowolnym innym
    rodzaju danych treningowych.

    Tak skomponowany zestaw przekształceń wykorzystywany jest przy konstruowaniu zbioru
    uczącego w komponencie \textit{Dataset}. Jego uniwersalny szkielet o nazwie \textit{BaseDataset}
    odpowiada za sprawny przepływ danych w ramach TorchFrame zapewniając łatwy dostęp
    do zbioru implementowanych konwersji oraz umożliwiając ich zastosowanie poprzez dedykowane
    do tego metody. Posiada również funkcjonalność odczytu danych ze wskazanej przez
    użytkownika ścieżki w pliku \textit{consts}, a także zdolność przetwarzanie ich na bieżąco, co pozwala
    zaoszczędzić pamięć w przypadku pracy na dużych zbiorach danych. W ramach tworzenia
    frameworka TorchFrame udostępnione zostały różne implementacje komponentu \textit{Dataset} opierające się o
    \textit{BaseDataset}. Ponownie jednak elastyczna struktura umożliwia użytkownikom
    skomponowanie własnych implementacji dostosowanych do indywidualnych potrzeb.

    Ostatecznym elementem ścieżki treningowej, w którym skupiają się wszystkie
    zgromadzone dotąd dane jest komponent \textit{Trainer}. Opiera się on o mechanizmy
    frameworka PyTorch w celu obliczania rezultatów pracy sieci, a także wyznaczanie
    kierunku uczenia w ramach funckji celu i wstęczną propagację modyfikującą wagi
    sieci w czasie treningu. \textit{Trainer} udostępnia na bieżąco dane pozwalające
    określić skuteczność uczenia sieci takie jak aktualny błąd sieci, ilość
    przetworzonych danych, czy numer epoki treningowej, w której aktualnie znajduje
    się model. W trakcie całego procesu możliwe jest również zapisywanie rezultatów
    w celu ich ponownego wykorzystania lub oceny.

  \subsubsection{Konfiguracja w TorchFrame}

    Kluczem do właściwego przeprowadzenia treningu sztucznej sieci neuronowej
    jest odpowiedni dobór hiperparametrów. Ich modyfikacja w znaczący sposób
    wpływa na otrzymywane rezultaty przez co istotne jest, aby w ramach
    kolejnych eksperymentów można było w łatwy sposób dostosowywać je do potrzeb.
    Framework TorchFrame udostępnia pojedynczy interfejs użytkownika w postaci
    pliku konfiguracyjnego JSON skupiającego wszystkie najistotniejsze
    elementy w jednym miejscu i pozwalającego zachować przejrzystość stosowanej
    konfiguracji. W poniższym rozdziale opisane zostaną dostępne parametry wraz
    z wartościami, jakie mogą przyjmować.

    \begin{enumerate}
    \item \textbf{\textit{Net model}} - ten parametr przyjmuje nazwę klasy, w której użytkownik
    zdefiniował strukturę sieci neuronowej na etapie inicjalizacji wraz z metodą
    \textit{forward} definiującą sposób przepływu danych w ramach inferencji modelu.
    \item \textbf{\textit{Criterion}} - parametr ten pozwala zdefiniować funckję kosztu
    używaną w procesie uczenia. W czasie treningu sieci neuronowych dąży się
    najczęściej do minimalizacji błędu określanego na bazie porównania rezultatów
    pracy sieci ze spodziewanym efektem. Zdefiniowana na tej podstawie funckja błędu
    określana jest jako funkcja celu, a proces uczenia sieci sprowadza się do zdefiniowania
    zestawu wag, dla których jej wartość jest możliwie najmniejsza. Zagadnienie
    to opisane zostało w książce \textit{Deep Learning} \cite{deeplearn} z 2016 roku:

    \begin{quote}
      'Funckja, którą chcemy minimalizować, lub maksymalizować nazywana jest funkcją celu lub kryterium.
      W przypadku minimalizacji możemy również nazywać ją funkcją kosztu, funkcją straty, lub funkcją błędu.'
    \end{quote}

    Na funckji kosztu spoczywa bardzo ważne zadanie. Musi ona wiernie destylować wszystkie
    aspekty modelu w jedną liczbę, w taki sposób, aby ulepszenia tej liczby były
    oznaką poprawy całego modelu. \cite{neuralsmithing}

    \begin{quote}
      'Funkcja kosztu redukuje wszystkie dobre i złe aspekty potencjalnie złożonego
      systemu do pojedynczej liczby, wartości skalarnej, która umożliwia
      klasyfikację i porównanie możliwych rozwiązań.'
    \end{quote}

    Dobór odpowiedniej funkcji kosztu może stanowić spore wyzwanie, ponieważ funkcja
    ta musi uchwycić właściwości danego problemu i być motywowana założeniami
    istotnymi z punktu widzenia realizowanego projektu. \cite{neuralsmithing}

    \begin{quote}
      '(...) Dlatego ważne jest, aby funkcja wiernie reprezentowała nasze cele projektowe.
      Jeśli wybierzemy słabą funkcję błędu i uzyskamy niezadowalające wyniki, to
      wina za złe określenie celu poszukiwań spoczywa na nas.'
    \end{quote}

    Opisy rozmaitych funkcji kosztu dostępnych w TorchFrame zamieszczone zostały
    w tabeli \ref{tab:cost_functions}.

    \begin{longtable}{ |m{2cm}|m{11cm}| }

     \hline
       \textbf{Funkcja kosztu} & \textbf{Opis} \\

     \hline
       \textbf{L1Loss} &

       Funkcja kosztu mierząca średni błąd bezwzględny pomiędzy
       odpowiadającymi sobie elementami ze zbioru wejściowego i docelowego.
       Można ją opisać za pomocą następującego wzoru:

       \[l(x,y) = L = \{l_1,...,l_N\}^T, l_n = |x_n - y_n|,\]

       gdzie \textit{N} jest rozmiarem pojedynczego pakietu danych, a \textit{x} i
       \textit{y} to tensory o arbitralnym kształcie, z których każdy posiada
       \textit{n} elementów.

       Parametry:
       \begin{itemize}
       \item reduction - parametr ten może przyjmować trzy wartości \textit{none},
       \textit{mean} oraz \textit{sum}. Pierwsza opcja spowoduje, że wartość funkcji
       celu zostanie wyznaczona zgodnie z podanym powyżej wzorem. Wartość \textit{mean}
       spowoduje wyznaczenie średniej wartości elementów wyjściowych. \textit{Sum}
       oznacza natomiast, że wyznaczona zostanie ich suma.
       \end{itemize} \\

     \hline
       \textbf{MSELoss} &

       Funkcja kosztu mierząca średni błąd kwadratowy pomiędzy każdym elementem
       wejściowym \textit{x} i celem \textit{y}. Opisuje ją poniższy wzór:

       \[l(x,y) = L = \{l_1,...,l_N\}^T, l_n = (x_n - y_n)^2,\]

       gdzie \textit{N} jest rozmiarem pojedynczego pakietu danych, a \textit{x} i
       \textit{y} to tensory o arbitralnym kształcie, z których każdy posiada
       \textit{n} elementów.

       Parametry:
       \begin{itemize}
       \item reduction - czytaj \textit{reduction} dla funkcji \textit{L1Loss}.
       \end{itemize} \\

     \hline
       \textbf{KLDivLoss} &

       Funkcja kosztu nazywana rozbieżnością Kullback'a - Leibler'a. Jest użyteczną
       miarą dla rozkładów ciągłych i często jest przydatna podczas wykonywania
       bezpośredniej regresji w przestrzeni (dyskretnie próbkowanych) ciągłych
       rozkładów wyjściowych.

       Kryterium to wymaga, aby rozmiary tensorów wejściowych i wyjściowych były
       identyczne.

       Wzór matematyczny opisujący rozbieżność KL przedstawiony został poniżej:

       \[l(x,y) = L = \{l_1,...,l_N\}, l_n = y_n \cdot (\log_{}y_n - x_n),\]

       gdzie indeks \textit{N} obejmuje wszystkie wymiary wejściowe, a \textit{L}
       ma ten sam kształt, co wejście.

       Parametry:
       \begin{itemize}
       \item reduction - poza wartościami opisanymi dla funckji \textit{L1Loss}
       może przyjmować również parametr \textit{batchmean}. Wymusza on sumowanie
       wartości wyjściowych, a następnie podział sumy przez rozmiar pakietu danych.
       \end{itemize} \\

     \hline
       \textbf{BCELoss} &

       Kryterium wyznaczające wartość Binarnej Entropii Krzyżowej pomiędzy wyjściem
       sieci, a spodziewanymi rezultatami jej pracy.

       Opisuje to poniższy wzór:

       \[l(x,y) = L = \{l_1,...,l_N\}^T, l_n = -w_n[y_n \cdot \log_{}x_n + (1 - y_n) \cdot \log_{}(1 - x_n)],\]
       \[w_n = waga[n],\]

       gdzie \textit{N} jest rozmiarem pakietu danych.

       Parametry:
       \begin{itemize}
       \item \textit{weight} - manualna wartość wagi przeskalowującej funkcję kosztu
       każdego elementu w pakiecie danych.
       \item \textit{reduction} - czytaj \textit{reduction} dla funkcji \textit{L1Loss}.
       \end{itemize} \\

     \hline
     \textbf{SmoothL1-
     Loss} &

       Kryterium przyjmujące postać funckji \textit{MSE} w przypadku, gdy wartość
       błędu bezwzględnego spada poniżej 1 oraz funkcji \textit{L1} w przeciwnym
       wypadku. Funkcja ta jest mniej czuła na wartości odstające niż
       \textit{MSELoss}, a w niektórych przypadkach zapobiega zjawisku
       eksplodującego gradientu. Znana jest również jako funkcja kosztu \textit{Huber'a}.

       Wzór opisujący:

       \[loss(x,y) = \frac{1}{n}\sum_{i}^{}z_i,\]

       gdzie $z_i$ zdefiniowane jest następująco:

       \[
       y = \left\{ \begin{array}{ll}
       0.5 \cdot (x_i - y_i)^2 & \textrm{gdy $|x_i - y_i| < 1$}\\
       |x_i - y_i| - 0.5 & \textrm{gdy $|x_i - y_i| \ge 1$}
       \end{array} \right.
       \]

       Parametry:
       \begin{itemize}
       \item reduction - czytaj \textit{reduction} dla funckji \textit{L1Loss}.
       \end{itemize} \\

     \hline

    \caption{Funkcje kosztu w TorchFrame}
    \label{tab:cost_functions}
    \end{longtable}

    \item \textbf{\textit{Optimizer}} - pozwala wybrać rodzaj optymalizatora używanego w procesie
    uczenia sieci neuronowej. Jest to algorytm odpowiedzialny za aktualizowanie
    wag modelu. Korzysta on z wartości funkcji kosztu, jak z drogowskazu
    wskazującego kierunek prowadzący do osiągnięcia globalnego minimum w
    procesie minimalizacji, jakim jest trening sieci.

    Zadanie zlokalizowania globalnego minimum nie jest zadaniem trywialnym.
    Optymalizacja sieci neuronowych jest optymalizacją niewypukłą. Oznacza to, że
    funckja celu posiada wiele optimów, z czego tylko jedno jest poszukiwanym
    optimum globalnym. Przykładowa płaszczyzna funckji celu przedstawiona
    została na Rysunku \ref{fig:plaszczyzna_optymalizacji}.

    \begin{figure}[H]
      \centering
      \includegraphics[width=4in]{plaszczyzna_optymalizacji}
      \caption[Przykładowa płaszczyzna funkcji celu - źródło: \url{https://towardsdatascience.com/neural-network-optimization-7ca72d4db3e0}]{Przykładowa płaszczyzna funkcji celu}
      \label{fig:plaszczyzna_optymalizacji}
    \end{figure}

    Można powiedzieć, że argumentami funkcji celu są wagi modelu sieci neuronowej.
    Każde kolejne połączenie w sieci posiadające własną wagę zwiększa wymiar
    płaszczyzny poszukiwań. Oznacza to, że dla modelu opisanego trzema wagami
    obszarem poszukiwań będzie płaszczyzna trójwymiarowa. Zazwyczaj modele sieci
    są jednak dużo większe. I tak dla modelu, na który składa się przykładowo sto
    wag poszukiwania optimum są w rzeczywistości prowadzone na hiperpłaszczyźnie
    posiadającej sto wymiarów.

    Najbardziej podstawowym algorytmem optymalizacji leżącym u podstaw innych
    metod jest tak zwany spadek gradientu. Najlepiej opisuje go programistyczna
    formuła aktualizacji wag sieci:

    \[\Theta = \Theta - \eta \cdot \nabla J(\Theta),\]

    gdzie $\eta$ jest długością kroku treningowego, a $\nabla J(\Theta)$ oznacza
    gradient funkcji kosztu zależnej od wag sieci $\Theta$. Warto zauważyć, że
    wartość gradientu wskazuje na płaszczyźnie funkcji celu położenie maksimum,
    dlatego w procesie minimalizacji musimy kierować się w przeciwną stronę i odejmować
    gradient od wag modelu.

    Sam gradient wyznaczany jest z wykorzystaniem bardzo popularnego obecnie
    algorytmu nazywanego propagacją wsteczną. Jego ogólna idea jest stosunkowo
    prosta. Rezultaty pracy sieci ewaluowane są względem pożądanych wyników w
    ramach opisanej w poprzednim punkcie funkcji celu. Jeśli wyniki oceny nie są
    zadowalające wagi sieci są modyfikowane, a cała operacja powtarzana jest aż
    do momentu zakończenia treningu.

    Raul Rojas w swojej książce \cite{systematic_introduction} z 1996 roku tak
    opisuje etapy tego algorytmu:

    \begin{quote}
      'Rozważmy sieć neuronową z pojedynczym wejściem rzeczywistym $x$ i funkcją
      sieci $F$. Pochodna $F(x)$ jest obliczana w dwóch fazach:

      Przekazywanie: wejście $x$ jest podawane do sieci. Funkcje aktywacji i ich pochodne są oceniane w każdym węźle (neuronie) sieci. Pochodne są przechowywane.

      Propagacja wsteczna: stała 1 jest podawana do jednostki wyjściowej i sieć biegnie wstecz. Informacje przychodzące do węzła są dodawane, a wynik jest mnożony przez wartość zmagazynowaną w lewej części jednostki. Rezultat jest transmitowany na lewo od jednostki. Wynik zgromadzony w jednostce wejściowej stanowi pochodną funckji sieci względem $x$.'
    \end{quote}

    Sam algorytm wstecznej propagacji działa poprawnie również dla sieci posiadających
    więcej niż jedną jednostkę wejściową, w których zaangażowanych jest więcej
    niezależnych zmiennych. Idea jego działania pozostaje wówczas ta sama.

    Wynika z tego, że algorytm wstecznej propagacji, a co za tym idzie czerpiący z
    niego w czystej postaci spadek gradientu, cechują się dużą stałością i
    przewidywalnością działania. W wielu przypadkach jest to pożądana cecha, jednak
    zwiększająca się ilość parametrów sieci może powodować pojawianie się trudności
    wymagających od algorytmów optymalizacji większej elastyczności w podejmowaniu działań.

    Do trudności takich zaliczyć można między innymi dobór odpowiedniej długości
    kroku treningowego tak, aby znaleźć złoty środek między szybkością wyznaczenia
    rozwiązania, a jego dokładnością. Dodatkowo w przypadku wspomnianych algorytmów
    wszystkie parametry sieci modyfikowane są w jednakowy sposób, co nie zawsze jest
    korzystne. Przykładowo, gdy dane treningowe są rzadkie, a różne cechy występują
    w nich z różnymi częstotliwościami, pożądanym zjawiskiem byłoby przeprowadzanie większych
    aktualizacji wag sieci dla cech występujących rzadziej. Pozostaje także niebezpieczeństwo
    związane z potencjalnym utknięciem modelu w jednym z minimów lokalnych, lub co gorsza
    w puncie siodłowym, którego przykładowy kształt przedstawia Rysunek \ref{fig:punkt_siodlowy}.
    Posiada on jednocześnie cechy zarówno minimum, jak i maksimum lokalnego i jest najczęściej
    otoczony przez płaskowyż cechujący się tą samą wartością funkcji kosztu. Bardzo często
    uniemożliwia to ucieczkę takim algorytmom jak spadek gradientu, ponieważ wartość
    gradientu jest wówczas bliska zeru we wszystkich wymiarach.

    \begin{figure}[H]
      \centering
      \includegraphics[width=4in]{punkt_siodlowy}
      \caption[Przykładowy kształt punktu siodłowego - źródło: \url{https://towardsdatascience.com/neural-network-optimization-7ca72d4db3e0}]{Przykładowy kształt punktu siodłowego}
      \label{fig:punkt_siodlowy}
    \end{figure}

    W odpowiedzi na opisane problemy opracowana została technika tak zwanego
    pędu \cite{Momentum}. Ogranicza ona oscylacje spadku gradientu w niewłaściwych kierunkach i
    przyspiesza proces zbiegania rozwiązania. Pęd sprowadza się do dodawania ułamka
    $\gamma$ wektora aktualizacji z poprzedniego kroku algorytmu do bieżącego wektora
    aktualizacji. Przedstawia to poniższy wzór:

    \[V(t)=\gamma \cdot V(t - 1) + \eta \cdot \nabla J(\Theta),\]
    który prowadzi ostatecznie do następującej modyfikacji parametrów:
    \[\Theta = \Theta - V(t).\]

    Nazwa tej metody nawiązuje do pędu znanego z fizyki. Można powiedzieć, że
    w miarę uczenia modyfikacje wag sieci nabierają prędkości we właściwym kierunku, co
    sprawia, że nie są już tak podatne na ewentualne nieprawidłowe zmiany kierunku i
    szybciej osiągają właściwy cel.

    Badacz Yurii Nesterov zauważył jednak zasadniczą wadę związaną z metodą pędu.
    W sytuacji, gdy algorytm przemieszcza się w dół zbocza funkcji celu w żaden
    sposób nie kontroluje, czy znalazł się już na dnie. Gdy je osiąga wartość pędu
    jest dość wysoka, co może spowodować, że punkt optymalny zostanie pominięty
    lub osiągnięty z opóźnieniem. Nesterov zaproponował nieco inne rozwiązanie.
    W jego metodzie najpierw wykonywany jest duży skok bazujący na poprzedniej
    wartości pędu, a następnie w nowym, potencjalnym położeniu, obliczana jest wartość 
    gradientu, która dokonuje korekcji miejsca docelowego. Dopiero wówczas dokonywana
    jest rzeczywista aktualizacja parametrów modelu. Metodę tę można przedstawić za
    pomocą następującego wzoru:

    \[V(t)=\gamma \cdot V(t - 1) + \eta \cdot \nabla J(\Theta - \gamma \cdot V(t - 1)),\]
    który prowadzi do aktualizacji:
    \[\Theta = \Theta - V(t).\]

    Argument $\nabla J$, $\Theta - \gamma \cdot V(t - 1)$ odpowiada w tym przypadku za określenie przewidywanego punktu docelowego.
    Pozwala to wyznaczyć wartość gradientu nie dla obecnego położenia modelu na płaszczyźnie
    funkcji celu, ale dla domniemanego położenia osiągniętego w przyszłości. Czyni to
    algorytm optymalizacji bardziej responsywnym na ewentualne zmiany i ogranicza ryzyko
    ominięcia punktu optymalnego ze względu na zbyt dużą wartość pędu.

    Bazując na opisanych tutaj algorytmach przygotowane zostały metody optymalizacji
    dostępne w TorchFrame, a zaczerpnięte z biblioteki PyTorch. Wiele z nich wprowadza
    również dodatkowe funkcjonalności, z których najistotniejsze opisane zostały w Tabeli \ref{tab:optimizers}.

    \begin{longtable}{ |m{3cm}|m{10cm}| }

     \hline
       \textbf{Optymalizator} & \textbf{Opis} \\

     \hline
       \textbf{SGD \cite{SGD}} &

       Algorytm stochastycznego spadku gradientu. Stanowi wariację bazowego spadku
       gradientu polegającą na aktualizowaniu parametrów sieci dla
       każdego przykładu uczącego. Opisuje to następujący wzór:

       \[\Theta = \Theta - \eta \cdot \nabla J(\Theta;x(i);y(i)),\]
       gdzie $\{x(i), y(i)\}$ jest zbiorem poszczególnych par treningowych.

       Stochastyczny spadek gradientu cechuje się częstszymi aktualizacjami i
       zmiennymi oscylacjami na płaszczyźnie funkcji celu. Czyni go to wolniejszym od
       algorytmu bazowego w dążeniu do rozwiązania, jednak zwiększa prawdopodobieństwo odkrycia
       bardziej optymalnych minimów.

       Możliwe jest również zastosowanie opisanych wcześniej mechanizmów pędu oraz
       algorytmu Nesterov'a w celu poprawy wydajności działania tego algorytmu. \\

     \hline
       \textbf{Adagrad \cite{Adagrad}} &

        Algorytm pozwalający dopasować długość kroku treningowego do poszczególnych
        wag modelu. Parametry rzadko biorące udział w pracy sieci otrzymują wówczas
        większe aktualizacje niż te występujące stosunkowo często. Adagrad nadaje się
        przez to do pracy z rzadkimi zbiorami treningowymi. Wzór prezentujący sposób aktualizacji
        pojedynczego i-tego parametru w sieci prezentuje się następująco:

        \[\Theta_{t+1,i} = \Theta_{t,i} - \frac{\eta}{\sqrt{G_{t,ii} + \epsilon}} \cdot g_{t,i},\]

        gdzie $G_{t,ii}$ jest diagonalną macierzą zawierającą sumy kwadratów gradientów
        wyznaczonych aż do chwili $t$, $g_{t,i}$ jest bieżącą wartością gradientu dla parametru $\Theta_i$, a
        $\epsilon$ oznacza współczynnik wygładzający, zabezpieczający przed ewentualnym dzieleniem
        przez 0.

        Wzór ten oznacza, że Adagrad modyfikuje długość kroku uczącego w każdej iteracji $t$
        dla każdego parametru $i$ bazując na przeszłych wartościach gradientu wyznaczonych
        dla tej wagi.

        Istotną wadą tego optymalizatora jest fakt ciągłego zmniejszania długości
        kroku uczącego ze względu na rosnącą nieustannie wartość sumy kwadratów
        gradientów w mianowniku. W krytycznych przypadkach może to doprowadzić
        do całkowitego zaniku tego kroku, co w konsekwencji prowadzi do
        zablokowania możliwości dalszego treningu sieci. \\

     \hline
       \textbf{AdaDelta \cite{Adadelta}} &

       Algorytm ten stanowi rozszerzenie optymalizatora Adagrad. Rozwiązuje
       trapiący go problem zanikającego kroku treningowego poprzez wprowadzenie
       ograniczenia, co do ilości przeszłych gradientów mających wpływ na aktualizację
       wag w bieżącej iteracji.

       Dodatkowo zamiast przechowywać określoną liczbę przeszłych kwadratów gradientów,
       AdaDelta wyznacza ich sumę poprzez rekursywne definiowanie zanikającej
       średniej wartości przeszłych kwadratów tych gradientów.

       Sposób działania tego mechanizmu opisuje poniższy wzór:

       \[E[g^2]_t = \gamma \cdot E[g^2]_{t-1} + (1 - \gamma) \cdot g^2_t,\]

       gdzie $E[g^2]_t$ oznacza średnią wartość sumy kwadratów gradientów w bieżącej
       iteracji, $E[g^2]_{t-1}$ definiuje tę średnią dla poprzedniego kroku, a
       $g^2_t$ stanowi kwadrat bieżącej wartości gradientu. Parametr $\gamma$
       pełni tutaj podobną rolę, co w przypadku mechanizmu pędu.

       Ostatecznie parametry modelu aktualizowane są zgodnie z następującą formułą:

       \[\Delta \Theta_t = -\frac{\eta}{\sqrt{E[g^2]_t + \epsilon}} \cdot g_t\] \\

     \hline
       \textbf{Adam \cite{Adam}} &

       Optymalizator Adam podobnie jak AdaDelta pozwala wyznaczać adaptacyjne długości
       kroku uczącego dla każdego parametru sieci. Przechowuje on również
       eksponencjalnie zanikającą średnią wartość kwadratów przeszłych gradientów $v_t$.
       Poza tym przechowuje również eksponencjalnie zanikającą średnią
       przeszłych gradientów $m_t$ podobną do wartości pędu.

       Wartości $m_t$ oraz $v_t$ są szacunkami odpowiednio pierwszego momentu
       (średnia) oraz drugiego momentu (wariancja niecentrowana), stąd nazwa metody.

       Autorzy Adama zauważyli, że $m_t$ oraz $v_t$, które inicjowane są jako
       wektory zerowe dążą tendencyjnie do zera zwłaszcza podczas początkowych
       iteracji. Aby przeciwdziałać temu procesowi wyznaczane są odpowiednie
       estymaty korekcyjne obu momentów zgodnie ze wzorami:

       \[m'_t = \frac{m_t}{1 - \beta^t_1}\]
       \[v'_t = \frac{v_t}{1 - \beta^t_2}\]

       Na tej podstawie wyznaczana jest reguła aktualizacyjna Adama, postaci:

       \[\Theta_{t+1} = \Theta_t - \frac{\eta}{\sqrt{v'_t} + \epsilon} \cdot m'_t.\]

       Wartości parametrów $\beta_1$, $\beta_2$ oraz $\epsilon$ mogą być dobierane eksperymentalnie,
       jednak rekomendowane jest przyjęcie następujących liczb: $\beta_1 = 0.9$, $\beta_2 = 0.999$, $\epsilon = 10^{-8}$. \\
     \hline

    \caption{Optymalizatory w TorchFrame}
    \label{tab:optimizers}
    \end{longtable}

    \item \textbf{\textit{Scheduler}} - umożliwia wybór planisty. Odpowiada on za
    modyfikowanie długości kroku treningowego w zależności od aktualnego numeru
    epoki, w jakim znajduje się proces uczenia.

    Jest to istotny element pozwalający dynamicznie dostosowywać krok do
    aktualnego położenia modelu na hiperpłaszczyźnie funkcji celu. Przykładowo
    gdy sieć znajduje się daleko od optimum globalnego, lub gdy wpadnie w jedno z
    optimów lokalnych możliwe jest zwiększenie kroku treningowego, aby przyspieszyć
    proces optymalizacji. Gdy natomiast model znajduje się w pobliżu optimum globalnego
    długość kroku może zostać zmniejszona, co przełoży się na zwiększenie dokładności
    uzyskanych wyników.

    \item \textbf{\textit{Init epoch}} - parametr wskazujący numer epoki od którego
    rozpocznie się trening sieci neuronowej.

    \item \textbf{\textit{Training epochs}} - ilość epok, jaką przejdzie model w danej
    sesji treningowej. W każdej epoce modelowi przekazane zostaną wszystkie dane
    wskazane w zbiorze treningowym.

    \item \textbf{\textit{Training Monitoring Period}} - określa co ile pakietów danych
    obiekt \textit{Trainer} będzie wyświetlał w konsoli dane treningowe, takie jak
    aktualna wartość funkcji kosztu.

    \item \textbf{\textit{Saving Period}} - parametr określający co ile epok
    obiekt \textit{Trainer} będzie dokonywał zapisu pośrednich wyników treningu sieci.

    \item \textbf{\textit{Dataloader Parameters}} - zestaw trzech parametrów.

    \textit{Batch size} określa ilość pojedynczych próbek ze zbioru treningowego które zostaną zgrupowane
    w jeden pakiet danych. W trakcie procesu uczenia sieci wartość funckji kosztu wyznaczana
    jest dla wszystkich elementów pakietu, a następnie uśredniana i dopiero wtedy
    wykorzystywana przez optymalizator do aktualizacji wag modelu.

    \textit{Shuffle} pozwala określić, czy dane ze zbioru treningowego zostaną
    posortowane, aby uniknąć powtarzalnej kolejności ich przekazywania na wejście
    sieci. Pozwala to lepiej przygotować sieć do pracy z danymi, których nie widziała
    w procesie uczenia.

    \textit{Num workers} to parametr  określający ilość procesów równolegle przetwarzających
    dane i tym samym przyspieszających cały proces treningu.

    \item \textbf{\textit{Retrain}} - wartość \textit{true} tego parametru powoduje, że
    TorchFrame podejmie próbę wczytania modelu sieci, optymalizatora oraz
    planisty zgodnie ze ścieżkami zdefiniowanymi w pliku \textit{consts}. Brak którejkolwiek
    ścieżki spowoduje, że dany parametr zostanie zainicjowany losowo, co pozwala
    dotrenować sieć w dowolnej konfiguracji parametrów startowych.

    \item \textbf{\textit{Train on gpu}} - wartość \textit{true} tego parametru sprawi, że
    TorchFrame podejmie próbę rozpoznania, czy dostępna jest graficzna jednostka
    obliczeniowa, a następnie przeniesie na nią dane umożliwiając przeprowadzanie
    za jej pomocą obliczeń w celu przyspieszenia procesu uczenia.

    \item \textbf{\textit{Dataset}} - stanowi zbiór parametrów określających sposób
    wstępnego przetwarzania danych w celu przygotowania zbioru uczącego.

    \textit{Name} - zawiera nazwę klasy definiującej sposób przetwarzania
    i przekazywania danych ze zbioru uczącego. Klasa ta powinna być oparta o
    szkielet \textit{BaseDataset} zapewniający sprawny przepływ danych przez framework
    TorchFrame.

    \textit{Input conversions} - jest to lista konwersji przeznaczonych do
    przetwarzania danych treningowych przekazywanych na wejście sieci. Każda
    konwersja składa się z nazwy popartej definicją w pliku \textit{conversions} oraz z
    parametrów, których używa w trakcie pracy na danych.

    \textit{Output conversions} - podobnie jak poprzedni parametr jest to lista
    konwersji. Ich implementacja pozwala przygotować zbiór danych wyjściowych
    wykorzystywanych jako punkt odniesienie dla rezultatów pracy sieci przy wyznaczaniu
    wartości funkcji celu.

    \textit{Transforms} - lista transformacji zdefiniowanych w pliku \textit{transforms},
    zapewniających między innymi formatowanie danych do postaci tensorów
    obsługiwanych przez bazowe mechanizmy biblioteki PyTorch.

    \item \textbf{\textit{Additional params}} - słownik języka Python, w którym użytkownik
    TorchFrame ma możliwość zdefiniowania dowolnych dodatkowych parametrów, jakie
    mogą wydać się pomocne w procesie treningu sieci neuronowej.

    \end{enumerate}

  \subsubsection{Testowanie w TorchFrame}
    Podstawą do określenia skuteczności przeprowadzanych treningów sieci neuronowych
    jest nie tylko obserwacja parametrów takich jak wartość funckji kosztu. Bardzo
    często konieczna jest wizualizacja wyników działania wytrenowanych modeli.
    Komponentowa budowa TorchFrame pozwala w łatwy sposób osiągnąć ten cel.
    Na bazie fragmentów wydzielonych z głównej architektury skomponowany został
    system testowy, którego schemat przedstawia Rysunek \ref{fig:torchframetest}.

    Większość przedstawionych na schemacie elementów zachowuje funkcjonalności
    opisane w rozdziale \textit{Architektura TorchFrame}. Istotną różnicą jest
    pojawienie się obiektu \textit{Tester}, którego zadaniem jest przeprowadzanie
    zdefiniowanych przez użytkownika testów. Podobnie jak obiekt \textit{Dataset} bazuje
    on na szkielecie \textit{BaseTester} odpowiadającym za funkcjonalne połączenie z
    resztą frameworka. W ramach tej pracy przygotowane zostały różne warianty
    testowe, jednak elastyczna forma szkieletu pozwala użytkownikom
    na definiowanie własnych scenariuszy.

    Inaczej zachowuje się również \textit{SetupCreator}. Element ten wyposażony został w specjalną
    metodę przeznaczoną do przygotowania środowiska testowego na bazie parametrów
    dostarczonych w pliku konfiguracyjnym JSON.

    Plik ten różni się od tego omawianego w przypadku procesu treningowego. Jego
    zawartość została w znacznym stopniu ograniczona, gdyż proces testowy nie
    wymaga już tak wielu parametrów konfiguracyjnych. Pozwala to zachować
    przejrzystość środowiska testowego i ułatwia rozpropagowanie danych
    w systemie.

    \begin{figure}[h!]
      \centering
      \includegraphics[width=6.5in]{torchframetest}
      \caption[Architektura testowa TorchFrame - źródło: Praca własna]{Architektura testowa TorchFrame}
      \label{fig:torchframetest}
    \end{figure}


  \subsection{Filtry \textit{AI}}
\label{Filtry_AI}

  Filtrowanie obrazów cyfrowych to bardzo popularny i powszechnie stosowany
  obecnie proces. Pozwala wyostrzyć niewyraźne zdjęcie, zmienić kontrast obrazu,
  czy zniwelować szumy tła. W rzeczywistości filtrowanie to nic innego, jak
  operacja matematyczna wykonywana na pikselach. Wykorzystywanie wartości wielu
  pikseli obrazu źródłowego w celu określenia wartości pojedynczego piksela w
  obrazie wynikowym. Sposób w jaki wartości te są pobierane oraz przetwarzane
  określają tak zwane maski. Przyjmują one postać macierzy kwadratowych różnych
  rozmiarów, a przechowywane w nich wartości decydują o wyniku filtracji.

  Poniższy rozdział tej pracy spróbuje udzielić odpowiedzi na pytanie, czy
  sieci neuronowe mogą sprawnie posłużyć w procesie filtrowania obrazów.
  Składa się na niego seria eksperymentów, w których specjalnie dobrane modele
  sieci spróbują odtworzyć wartości masek użytych do przygotowania danych
  treningowych, a następnie wykorzystają je do przetworzenia zupełnie nowych
  obrazów.

  Dane referencyjne składają się z zestawu obrazów przetworzonych za pomocą
  filtrów wbudowanych w bibliotekę \textit{OpenCV} takich, jak filtr Sobela, czy
  sepia.

  Wszystkie modele wytrenowane zostały w oparciu o framework \textit{TorchFrame}.

  \subsubsection{Filtr Sobela}

    Jednym z podstawowych i najbardziej znanych obecnie filtrów obrazu jest
    filtr Sobela-Feldmana, który zaprezentowany został po raz pierwszy w 1968
    roku na konferencji Laboratorium Sztucznej Inteligencji uniwersytetu Stanforda
    (\textit{SAIL}). Znajduje on przede wszystkim zastosowanie w procesie wykrywania krawędzi
    na obrazach cyfrowych. Sam Irwin Sobel opisuje ten filtr następująco \cite{sobel}:

    \begin{quote}
      "Motywacją w rozwijaniu tego rozwiązania było stworzenie wydajnej obliczeniowo
      estymacji gradientu, która byłaby bardziej izotropowa niż popularny wówczas operator
      "Krzyża Roberts'a"."
    \end{quote}

    Operator izotropowy to, w kontekście przetwarzania obrazów, operator, którego
    działanie jest równoważne dla wszystkich kierunków na obrazie. Filtr Sobela
    wyznacza przybliżenie gradientu funkcji natężenia obrazu. Dla każdego pojedynczego
    piksela wynikiem jego działania jest wektor gradientu (lub jego długość) wskazującego kierunek wzrostu
    intensywności obrazu, wyznaczony na bazie otaczających wartości ośmiu innych pikseli.

    Na klasyczny filtr Sobela-Feldmana składają się dwie maski:

    \[G_x =
    \begin{bmatrix}
    -1 & 0 & +1 \\
    -2 & 0 & +2 \\
    -1 & 0 & +1
    \end{bmatrix}
    \]

    \[G_y =
    \begin{bmatrix}
    -1 & -2 & -1 \\
    0 & 0 & 0 \\
    +1 & +2 & +1
    \end{bmatrix}
    \]

    $G_x$ odpowiada za filtrowanie krawędzi w pionie, a $G_y$ w poziomie. Obie maski
    mogą być stosowane oddzielnie. Sam proces filtrowania bazuje na konwolucji
    opisanej w ramach sieci splotowych w rozdziale \ref{sieci_splotowe}, polegającej na
    równomiernym przesuwaniu stosowanych filtrów wzdłuż analizowanego obrazu, przy
    jednoczesnym wykonywaniu zdefiniowanych w nich obliczeń w każdym punkcie. Można
    w tym miejscu dostrzec spore podobieństwo pomiędzy klasycznymi maskami i ich zastosowaniem,
    a neuronami wchodzącymi w skład warstw konwolucyjnych sztucznych sieci splotowych. Nie bez powodu
    neurony te nazywane są filtrami.

    W przeprowadzonych doświadczeniach zastosowany został filtr Sobela z
    maską $G_x$. Zbiór uczący wykorzystywany w procesie treningu sieci składał się
    z różnorodnych obrazów dobieranych w sposób losowy. Przykładowe zdjęcia wchodzące
    w skład tego zbioru przedstawia Rysunek \ref{fig:dataset_filters}.

    \begin{figure}[H]
      \centering
      \includegraphics[width=6in]{dataset_filters}
      \caption[Przykładowe obrazy ze zbioru treningowego - źródło: Praca własna]{Przykładowe obrazy ze zbioru treningowego}
      \label{fig:dataset_filters}
    \end{figure}

    Aby mogły właściwie pełnić swoją rolę obrazy treningowe zostały w pierwszej kolejności
    poddane odpowiedniemu przetworzeniu wstępnemu. W ramach tego procesu rozmiar każdego zdjęcia
    zmniejszony został do wymiarów $256x256$ pikseli, a wartości kolorów ograniczone
    do zakresu $<0, 1>$ w celu usprawnienia obliczeń wykonywanych przez sieć m.in.
    poprzez zapobieganie zjawisku eksplodującego gradientu. Dodatkowo w przypadku tego
    filtru zastosowana została konwersja obrazów do formatu czarno-białego, co pozwoliło
    wyodrębnić pojedynczy kanał kolorystyczny z oryginałów. Tak przetworzony zbiór uczący
    podawany był na wejście sieci neuronowej, a rezultaty jej pracy porównywane
    z obrazami na które dodatkowo nałożony został filtr Sobela za pomocą biblioteki
    \textit{OpenCV}. W przypadku obrazów referencyjnych po zastosowaniu filtracji
    konieczne okazało się również przeskalowanie wartości pikseli do przedziału
    $<0, 1>$, ponieważ w sieci zastosowana została funkcja aktywacji \textit{ReLU}, która
    opisana została w rozdziale \ref{funkcje_aktywacji}. Jej charakterystyka
    wyklucza pojawianie się wartości ujemnych, jako rezultatów pracy modelu, co
    w przypadku braku odpowiedniej normalizacji prowadziło do niepoprawnych wyników.

    Sam model sieci składa się w tym przypadku z pojedynczegwo neuronu w warstwie
    konwolucyjnej, filtrującego obraz za pomocą macierzy kwadratowej stopnia
    trzeciego. Odzwierciedla to oryginalną macierz filtracji $G_x$ w stosunku jeden
    do jednego, ponieważ każda z dziewięciu wag sieci odpowiada jednemu polu w tej
    macierzy. Takie podejście pozwala jednoznacznie ocenić stopień odwzorowania
    maski przez sieć poprzez analizę wartości jej parametrów.

    W ramach przeprowadzonych eksperymentów, wypróbowane zostały różne konfiguracje
    hiperparametrów treningowych. Ostatecznie najlepszy rezultat udało się uzyskać
    przy zastosowaniu następującej konfiguracji:

    \begin{itemize}
    \item Funkcja kosztu: \textit{SmoothL1Loss}
    \item Optymalizator: \textit{Adam}
    \item Funkcja aktywacji: \textit{ReLU}
    \item Ilość epok treningowych: 3
    \item Rozmiar pakietu danych: 8
    \end{itemize}

    Efekt działania wytrenowanego modelu przedstawia Rysunek \ref{fig:sobel_result}.

    \begin{figure}[H]
      \centering
      \includegraphics[width=6in]{sobel_result}
      \caption[Działanie filtru Sobela - źródło: Praca własna]{Działanie filtru Sobela}
      \label{fig:sobel_result}
    \end{figure}

    Zgodnie z opisem oryginalne, czarno-białe zdjęcie poddawane filtracji zamieszczone
    zostało na środku. Po lewej stronie przedstawiony został obraz przefiltrowany
    z wykorzystaniem biblioteki \textit{OpenCV}, a po prawej obraz przetworzony przez
    wytrenowany model sieci neuronowej. Wizualnie otrzymane rezultaty są niemal
    identyczne. Zdjęcie wygenerowane przez sieć charakteryzuje się nieco ciemniejszą
    barwą, co może mieć związek z normalizacją danych przeprowadzaną w celu skuteczniejszego
    uczenia sieci. Poprawne rezultaty treningu najlepiej ocenić można analizując
    macierz wag modelu, która przedstawia się następująco:

    \[G_{nn} =
    \begin{bmatrix}
    -0.1135 & -0.0144 & +0.1363 \\
    -0.3239 & -0.0016 & +0.3340 \\
    -0.1199 & -0.0058 & +0.1335
    \end{bmatrix}
    \]

    Porównując uzyskane rezultaty z oryginalną maską $G_x$ łatwo zauważyć można,
    że sieć neuronowa właściwie odtworzyła panujące w niej proporcje. Odpowiednio
    mniejszy rząd wielkości odzwierciedla normalizację danych na których uczony
    był model. Środkowa kolumna składa się w całości z wartości bliskich zeru.
    Kolumna prawa zawiera wartości dodatnie z wyraźną dominacją elementu środkowego.
    Podobnie rozkładają się wartości w kolumnie lewej zawierającej wyłącznie wartości
    ujemne.

    Wskazówką w określaniu poprawności przeprowadzanych treningów może być również
    wykres wartości funkcji kosztu w kolejnych krokach uczenia. Przebieg taki
    przedstawiony został na Rysunku \ref{fig:tensorboard_sobel}.

    \begin{figure}[H]
      \centering
      \includegraphics[width=6in]{tensorboard_sobel}
      \caption[Wykres wartości funkcji kosztu w zależności od ilości kroków treningowych - źródło: Praca własna]{Wykres wartości funkcji kosztu w zależności od ilości kroków treningowych}
      \label{fig:tensorboard_sobel}
    \end{figure}

    W początkowej fazie przedstawiona charakterystyka cechuje się znaczącym spadkiem
    wartości, po czym utrzymuje się na stosunkowo stałym poziomie. Jest to spodziewany
    efekt, spowodowany niewielkimi rozmiarami modelu, co przełożyło się na szybkie
    zlokalizowanie globalnego minimum przez zastosowany optymalizator. Warto wspomnieć,
    że zastosowanie tej metryki do analizy działania sieci może być zwodnicze.
    Ciągły spadek wartości funkcji kosztu nie zawsze oznacza wzrost dokładności działania
    trenowanego modelu. Sieć neuronowa może zacząć w zbyt dużym stopniu dostosowywać się
    do dostępnych danych treningowych tracąc zdolność do generalizacji rozwiązania dla
    przykładów spoza tego zbioru. Zjawisko takie nazywane jest przeuczeniem i najczęściej
    objawia się spadkiem dokładności, przy jednoczesnym opadaniu wartości funkcji kosztu.

    Liczne eksperymenty związane z zastosowaniem rozmaitych hiperparametrów wykazały, że
    pomimo niewielkich rozmiarów modelu zlokalizowanie minimum globalnego nie było
    zadaniem trywialnym. Prosty optymalizator, taki jak \textit{SGD} nie był w stanie odnaleźć
    odpowiedniego punktu, a rezultaty jego działania w dużej mierze zależały od
    losowych wartości przypisanych do wag sieci na początku każdej sesji treningowej.
    Dopiero zastosowanie algorytmu adaptacyjnego, jakim jest \textit{Adam} pozwoliło
    uzyskać powtarzalność w osiąganiu właściwych rezultatów. Algorytm \textit{SGD}
    zastosowany został dopiero w końcowej fazie uczenia z bardzo małym krokiem
    treningowym rzędu $\eta = 10^{-8}$, co pozwoliło nieznacznie poprawić uzyskane
    wyniki.

    Olbrzymi wpływ na rezultaty miały również zastosowane sposoby przetworzenia danych
    treningowych, między innymi wspomniane już ograniczenie wartości pikseli
    w przedziale $<0,1>$, tak aby współgrały z zastosowaną funckją aktywacji.

  \subsubsection{Sepia}

    Sepia to filtr obrazu nadający zdjęciom charakterystycznego czerwonawo-brązowego
    koloru. Jego nazwa pochodzi od rodzaju mątwy, Sepii, z której pozyskiwany
    był atrament cechujący się takim właśnie zabarwieniem. Choć obecnie materiał ten
    nie jest już powszechnie wykorzystywany w malarstwie, to sepia wciąż cieszy się
    sporą popularnością w dziedzinie fotografii i cyfrowego przetwarzania obrazów.

    Zastosowanie sepii, jako filtru, wiąże się z konwolucyjnym przetworzeniem
    zdjęcia za pomocą następującej maski:

    \[G_x =
    \begin{bmatrix}
    +0.131 & +0.534 & +0.272 \\
    +0.168 & +0.686 & +0.349 \\
    +0.189 & +0.769 & +0.393
    \end{bmatrix}
    \]

    Sieć neuronowa użyta do jej odtworzenia składa się z trzech neuronów w
    warstwie splotowej. Jest to minimalna struktura niezbędna do odtworzenia
    trójkanałowego obrazu wyjściowego, jako że każdy filtr odpowiada za wygenerowanie
    pojedynczej warstwy kolorystycznej.

    W procesie uczenia sieci zastosowany został identyczny zbiór treningowy, co w
    przypadku filtru Sobela, opisanego w poprzednim rozdziale. Ponownie w ramach
    wstępnego przetwarzania danych, wartości pikseli ograniczone zostały w przedziale
    $<0,1>$, a każdy obraz przeskalowany do wymiarów $256x256$. W przypadku obrazów
    referencyjnych zastosowany został oczywiście filtr zapewniający efekt sepii
    pochodzący z biblioteki \textit{OpenCV}. Dodatkowo wartości, które w wyniku filtracji
    przekroczyły wartość 1, zostały ograniczone do jej poziomu. Główną różnicą w stosunku
    do opisanego uprzednio filtru Sobela jest fakt, że sieć trenowana była na
    obrazach posiadających pełny zakres trzech kanałów kolorystycznych, a nie jak
    poprzednio z wykorzystaniem formatu czarno-białego.

    Optymalny rezultat działania sieci udało się uzyskać dla następującego
    zestawu hiperparametrów:

    \begin{itemize}
    \item Funkcja kosztu: \textit{SmoothL1Loss}
    \item Optymalizator: \textit{Adam}
    \item Funkcja aktywacji: \textit{ReLU}
    \item Ilość epok treningowych: 5
    \item Rozmiar pakietu danych: 8
    \end{itemize}

    Większość parametrów nie uległa zmianie w stosunku do filtru Sobela. Świadczy to
    przede wszystkim o uniwersalności algorytmu \textit{Adam} w odnajdywaniu optymalnych
    rozwiązań w procesie minimalizacji. Rezultaty pracy wytrenowanego modelu przedstawione
    zostały na Rysunku \ref{fig:sepia_result}.

    \begin{figure}[H]
      \centering
      \includegraphics[width=6in]{sepia_result}
      \caption[Działanie sepii - źródło: Praca własna]{Działanie sepii}
      \label{fig:sepia_result}
    \end{figure}

    Wizualnie wyniki działania sztucznej sieci neuronowej cechują się nieco bardziej
    przytłumionymi kolorami w stosunku do obrazów wygenerowanych z wykorzystaniem
    oryginalnej maski. Powodem takiego zachowania jest najprawdopodobniej dążenie
    sieci do zminimalizowania wartości funkcji kosztu w obrębie całego zbioru
    treningowego, co wiąże się z pewnym mimowolnym uśrednieniem wag modelu. Przekłada
    się ono na trudności w odwzorowaniu obrazów cechujących się dużym kontrastem.

    Dodatkowo w rozważanym przypadku rozmiar zastosowanej sieci, choć pozornie
    niewielki, nie pozwala jednoznacznie przełożyć uzyskanych parametrów modelu na
    referencyjną maskę $G_x$. Każdy z trzech zastosowanych filtrów przetwarza
    obraz za pomocą macierzy $3x3$ dla każdej z trzech warstw kolorystycznych.
    Oznacza to, że pojedynczy neuron powiązany jest z 27 wagami, a w całym modelu
    jest ich sumarycznie 81. Informacja o sposobie filtracji jest więc mocno
    rozproszona i ciężko jednoznacznie odczytać postać maski jaką w praktyce
    implementuje wytrenowany model.

    W takiej sytuacji, poza wizualną oceną działania sieci, przydatnych informacji
    dostarcza wykres wartości funkcji kosztu przedstawiony na Rysunku \ref{fig:tensorboard_sepia}.

    \begin{figure}[H]
      \centering
      \includegraphics[width=6in]{tensorboard_sepia}
      \caption[Wykres wartości funkcji kosztu w zależności od ilości kroków treningowych - źródło: Praca własna]{Wykres wartości funkcji kosztu w zależności od ilości kroków treningowych}
      \label{fig:tensorboard_sepia}
    \end{figure}

    Wyraźne zbocze opadające w początkowej fazie treningu świadczy o poprawnym dążeniu
    sieci do rozwiązania optymalnego, a następujące po nim wypłaszczenie oznacza, że
    rozwiązanie to zostało osiągnięte. Porównując Rysunki \ref{fig:tensorboard_sepia}
    i \ref{fig:tensorboard_sobel} zauważyć można że w przypadku sepii długość zbocza
    opadającego jest znacząco większa. Może być to spowodowane większą złożonością
    funkcji kosztu w przypadku tego filtru, lub niekorzystnym początkowym umiejscowieniem
    algorytmu optymalizacji na płaszczyźnie celu, wynikającym z losowego doboru startowych parametrów sieci.
    Najprawdopodobniej oba te czynniki wywarły swój wpływ na przebieg procesu uczenia.

    W ramach przeprowadzonych eksperymentów ponownie testowane były rozmaite funkcje
    optymalizacji, jednak żadna nie pozwoliła osiągnąć takiej powtarzalności w osiąganiu
    poprawnych rezultatów, jak \textit{Adam}. Nawet zastosowanie innych optymalizatorów
    adaptacyjnych bardzo często kończyło się na osiągnięciu jedynie jednego z minimów
    lokalnych. Przykładem takiego rozwiązania może być efekt działania algorytmu
    \textit{Adagrad} przedstawiony na Rysunku \ref{fig:sepia_fail}.

    \begin{figure}[H]
      \centering
      \includegraphics[width=6in]{sepia_fail}
      \caption[Przykład działania sieci w minimum lokalnym - źródło: Praca własna]{Przykład działania sieci w minimum lokalnym}
      \label{fig:sepia_fail}
    \end{figure}

    Choć na obrazie wygenerowanym przez sieć dostrzec można pewne podobieństwo do
    poprawnie działającego filtru, to wyraźnie widać że nie jest to zadowalający
    rezultat, a znalezione na hiperpłaszczyźnie funkcji celu minimum nie jest z pewnością
    minimum globalnym.

  \subsubsection{Filtr górnoprzepustowy}

    Filtry górnoprzepustowe używane są w celu uwypuklenia szczegółów występujących na
    obrazie. Tłumią one elementy o niskiej częstotliwości, a wzmacniają te cechujące
    się częstotliwościami wysokimi poprzez zwiększenie ich jasności lub barwy.
    Efektem zastosowania filtru górnoprzepustowego jest najczęściej zwiększenie
    kontrastu poprzez podkreślenie ostrych krawędzi obiektów. W pracy
    \textit{"Image Enhancement Techniques using Highpass and
    Lowpass Filters"} \cite{highpass_filter} znaleźć można następujący opis:

    \begin{quote}
      "Filtr górnoprzepustowy to filtr, który dobrze przepuszcza wysokie częstotliwości,
      ale tłumi częstotliwości niższe niż częstotliwość graniczna. Ostrzenie jest
      zasadniczo operacją górnoprzepustową w dziedzinie częstotliwości.

      Istnieje kilka standardowych form filtrów górnoprzepustowych, takich jak
      filtr Butterworth'a, czy Gauss'a. Wszystkie filtry górnoprzepustowe ($H_{hp}$) są
      zazwyczaj reprezentowane poprzez ich relację z filtrami dolnoprzepustowymi ($H_{lp}$):

      \[H_{hp} = 1 - H_{lp}\]."
    \end{quote}

    W ramach niniejszego eksperymentu zastosowany został filtr górnoprzepustowy
    reprezentowany przez maskę następującej postaci:

    \[G_x =
    \begin{bmatrix}
    -1 & -1 & -1 \\
    -1 & +9 & -1 \\
    -1 & -1 & -1
    \end{bmatrix}
    \]

    W celu odwzorowania działania filtru zastosowana została sztuczna sieć
    splotowa złożona z trzech neuronów, podobnie jak miało to miejsce w przypadku
    sepii. Model sieci uczony był na tych samych danych, z tą różnicą, że dane
    referencyjne zostały przetworzone z wykorzystaniem filtru górnoprzepustowego
    $G_x$.

    Wyniki działania wytrenowanego modelu przedstawia Rysunek \ref{fig:sharpen_result}.

    \begin{figure}[H]
      \centering
      \includegraphics[width=6in]{sharpen_result}
      \caption[Działanie filtru górnoprzepustowego - źródło: Praca własna]{Działanie filtru górnoprzepustowego}
      \label{fig:sharpen_result}
    \end{figure}

    Parametry dla których uzyskany został przedstawiony efekt są następujące:

    \begin{itemize}
    \item Funkcja kosztu: \textit{SmoothL1Loss}
    \item Optymalizator: \textit{Adam}
    \item Funkcja aktywacji: \textit{ReLU}
    \item Ilość epok treningowych: 5
    \item Rozmiar pakietu danych: 2
    \end{itemize}

    Na uwagę zasługuje tutaj przede wszystkim rozmiar pakietu danych odróżniający
    ten zbiór hiperparametrów od dwóch poprzednich. Został on zmniejszony w celu
    ograniczenia wpływu uśrednienia wartości funkcji kosztu na modyfikację wag
    modelu. Pomimo tego zabiegu wyraźnie widać, że w ramach wszystkich przeprowadzonych
    eksperymentów to właśnie efekt działania sieci górnoprzepustowej najbardziej
    odbiega od obrazu referencyjnego. Jest to spowodowane charakterem działania
    badanego filtru. Jego celem jest uwypuklanie pojedynczych elementów obrazu, takich
    jak krawędzie, co stoi w niejakiej sprzeczności z procesem treningu, którego celem
    jest osiągnięcie jak najmniejszej wartości błędu w obrębie wszystkich danych
    treningowych. Oznacza to, że sieć będzie dążyła do jak najlepszego uśrednienia
    procesu filtracji górnoprzepustowej. Efekt ten został niewątpliwie osiągnięty.
    Obraz wygenerowany przez model cechuje się wyraźniejszymi konturami, jednak
    nie w takim stopniu jak obraz docelowy.

    Wykres wartości funkcji kosztu przedstawia Rysunek \ref{fig:tensorboard_sharpen}.

    \begin{figure}[H]
      \centering
      \includegraphics[width=6in]{tensorboard_sharpen}
      \caption[Wykres wartości funkcji kosztu w zależności od ilości kroków treningowych - źródło: Praca własna]{Wykres wartości funkcji kosztu w zależności od ilości kroków treningowych}
      \label{fig:tensorboard_sharpen}
    \end{figure}

    Podobnie jak w dwóch poprzednich przypadkach początkowy spadek wartości
    prowadzi model do punktu optymalnego rozwiązania. W punkcie tym algorytm optymalizacji
    zaczyna nieznacznie oscylować podejmując kolejne próby poprawy uzyskanych wyników.

    W ramach testów różnych rozwiązań na uwagę zasługują rezultaty osiągnięte
    z wykorzystaniem optymalizatora \textit{AdaDelta} przedstawione na Rysunku \ref{fig:sharpen_fail}.

    \begin{figure}[H]
      \centering
      \includegraphics[width=6in]{sharpen_fail}
      \caption[Przykład działania sieci w minimum lokalnym - źródło: Praca własna]{Przykład działania sieci w minimum lokalnym}
      \label{fig:sharpen_fail}
    \end{figure}

    Wygenerowany obraz został wyostrzony w stosunku do oryginału, jednak
    nieznacznie wypaczone zostały przy tym kolory tła, co jest zjawiskiem niekorzystnym i
    może wskazywać na osiągnięcie przez model jednego z lokalnych minimów funkcji celu.

  \subsubsection{Podsumowanie}

    Przeprowadzone w ramach tego rozdziału eksperymenty dowodzą, że sztuczne
    sieci splotowe z powodzeniem mogą posłużyć w procesie filtrowania obrazów.
    Należy zadać sobie jednak pytanie, czy rezultaty ich działania są w stanie
    zrekompensować stosunkowo czasochłonny proces treningu związany z odpowiednią
    obróbką danych oraz eksperymentalnym doborem właściwych hiperparametrów.

    W przypadku prostych filtrów, jak sepia, czy filtr górnoprzepustowy korzystniejsze
    okazuje się zastosowanie klasycznych metod przetwarzania w postaci
    odpowiednich masek konwolucyjnie nakładanych na docelowe obrazy. Rozwiązania te
    są często pozbawione typowej dla sieci neuronowych tendencji do uśredniania
    wyników, co przekłada się na lepsze rezultaty w sytuacjach, w których celem jest
    na przykład osiągnięcie dużego kontrastu przetwarzanych obrazów.

    Nie oznacza to jednak, że sieci neuronowe mogą zostać zastąpione w każdej sytuacji.
    Ich zdolności adaptacyjne sprawiają, że znajdują one zastosowanie w rozwiązaniach, w
    których klasyczne metody zawodzą. Dowodem na to będą kolejne rozdziały tej pracy.


  \subsection{Automatyczne kolorowanie czarno-białych obrazów}

  Problem kolorowania czarno-białych obrazów cieszy się dużym zainteresowaniem z
  wielu powodów. Od potrzeb kulturowych takich jak możliwość lepszego
  zwizualizowania oraz zrozumienia przeszłości poprzez kolorowania zdjęć z
  czasów, kiedy występowały one jedynie w kolorach czerni i bieli, po potrzeby
  technologiczne takie jak rekonstrukcja filmów oraz poprawa obrazu cyfrowego.

  Pomimo braku informacji o kolorze w czarno-białych zdjęciach, ludzie są w
  stanie określić potencjalne, rzeczywiste barwy obiektów na zdjęciach bazując
  na treści tych zdjęć oraz swoim doświadczeniu. Można z tego wywnioskować, że
  zdjęcia te zawierają informacje wystarczające do oszacowania potencjalnych
  kolorów. Pozwala to założyć, że do tego zagadnienia można skutecznie wykorzystać
  konwolucyjne sieci neuronowe, które cechują się niezwykłą umiejętnością do
  rozpoznawania wzorców oraz posiadają wyjątkowe zdolności do adaptacji. Z tego
  właśnie powodu sieci splotowe zostaną użyte w przedstawionym rozwiązaniu.

\subsubsection{Podejście}

  Rozważając możliwe sposoby pokolorowania czarno-białego zdjęcia można spostrzec,
  że kiedy niektóre powierzchnie na zdjęciu mają zazwyczaj oczywiste barwy, niebo
  jest zazwyczaj niebieskie, a trawa zielona, to są też powierzchnie, które
  posiadają szeroki wachlarz możliwych kolorów, na przykład samochodów może być
  zarówno czerwony jak i niebieski albo zielony. Z tego powodu celem zaprezentowanego
  rozwiązania jest niekoniecznie odtworzenie rzeczywistych barw obrazu, a raczej
  wygenerowanie barw, które mogłyby być barwami rzeczywistymi.

  Aby zwiększyć efektywność uczenia wykorzystano przestrzeń barw CIELab. W tej
  przestrzeni barwę obrazu opisują 3 składowe:
  \begin{itemize}
  \item L -jasność (luminacja)
  \item A -barwa od zielonej do magenty
  \item B - barwa od niebieskiej do żółtej
  \end{itemize}
  Przestrzeń barw CIELab została przedstawiona na Rysunku \ref{fig:CIELab}

  \begin{figure}[ht]
    \centering
    \includegraphics[width=3.5in]{CIELab}
    \caption[Przestrzeń barw CIELab - źródło:
    \url{https://www.flickr.com/photos/greenmambagreenmamba/4236391637}]
    {Przestrzeń barw CIELab.}
    \label{fig:CIELab}
  \end{figure}

  Zaletą zastosowania CIELab jest fakt, że jest ona najbardziej równomierną
  przestrzenią barw, co oznacza, że jeśli barwy znajdują się w jednakowej
  odległości od siebie w tej przestrzeni, to będą one postrzegane jako jednakowo
  różniące się od siebie. Powinno to zwiększyć skuteczność uczenia sieci oraz
  zapewnić bardziej realistyczne kolorowanie.

  Składowa \textit{L}, jako, że jest identyczna dla obrazu kolorowego jak i
  czarno-białego, stanowi w tym przypadku wejście sieci, na jej podstawie sieć
  odtwarza składowe \textit{A} oraz \textit{B}, które reprezentują przewidziane
  kolory dla obrazu wejściowego.

  Jako rozwiązanie podanej problematyki wpierw został oceniony autorski
  model podstawowy, na jego podstawie zostało przeprowadzone porównanie skuteczności
  różnych konfiguracji, w których uczony był model. Do elementów poddanych
  testom należą algorytm optymalizacyjny, funkcja straty, funkcja aktywacja oraz
  sposób przetwarzania wstępnego danych treningowych.

\subsubsection{Model podstawowy}

  Opracowany przez nas model jest to FCN. Konwolucyjna część sieci składa się
  z 12 warstw splotowych
  mających na celu nauczyć się mapować składową wejściową \textit{L} na wyjściowe
  składowe \textit{A} i \textit{B}. Składowe wyjściowe muszą mieć takie same
  wymiary jak składowa wejściowe, co oznacza, że kluczowym było odpowiednie dobranie
  parametrów warstw takich jak \textit{padding} (pol. otoczka), \textit{stride}
  (pol. krok) oraz wielkość filtrów. Pełna architektura sieci została przedstawiona w
  Tabeli \ref{table:model_architecture}.
  \noindent\begin{table}[h!]
    \center
    \begin{tabular}{|c | c | m{3.3em} | c | c | m{4em} | c | c| }
     \hline
     Nr & Warstwa & Rozmiar filtra & Stride & Padding & Batch Normalization &
     Fun. aktywacji & Ilość kanałów wej./wyj. \\ [0.5ex]
    \hline
    1 & Splotowa & 3x3 & 1 & 1 & Tak & RELU & 1/32 \\ \hline
    2 & Splotowa & 3x3 & 1 & 1 & Tak & RELU & 32/32 \\ \hline
    3 & Splotowa & 3x3 & 1 & 1 & Tak & RELU & 32/32 \\ \hline
    4 & Splotowa & 3x3 & 1 & 1 & Tak & RELU & 32/32 \\ \hline
    5 & Splotowa & 3x3 & 1 & 1 & Tak & RELU & 32/64 \\ \hline
    6 & Splotowa & 3x3 & 1 & 1 & Tak & RELU & 64/64 \\ \hline
    7 & Splotowa & 3x3 & 1 & 1 & Tak & RELU & 64/64 \\ \hline
    8 & Splotowa & 3x3 & 1 & 1 & Tak & RELU & 64/32 \\ \hline
    9 & Splotowa & 3x3 & 1 & 1 & Tak & RELU & 32/32 \\ \hline
    10 & Splotowa & 1x1 & 1 & 0 & Tak & RELU & 32/32 \\ \hline
    11 & Splotowa & 1x1 & 1 & 0 & Tak & RELU & 32/32 \\ \hline
    12 & Splotowa & 1x1 & 1 & 0 & Nie & - & 32/2 \\ \hline
    \end{tabular}
    \caption{Architektura modelu podstawowego.}
    \label{table:model_architecture}
  \end{table}
  Pierwsza warstwa konwolucyjna rozkłada wejściowy kanał na 32 kanały, co
  pozwala wyciągnąć z niego jak najwięcej informacji o cechach obrazu. Warstwa
  ta ma wielkość filtra 3x3, tak więc aby zachować wymiar kanałów zostały
  zastosowane parametry $\textit{padding}=1$ oraz $\textit{stride}=1$.

  Kolejne trzy warstwy ekstraktują z wejściowych 32 kanałów najbardziej istotne
  cechy związane z powiązaniem treści obrazu z szacowanym kolorem jego powierzchni.
  Warstwy te na swoje wyjście przekazują po 32 kanały zawierające wykryte
  powiązanie pomiędzy pikselami kanałów wejściowych.

  Warstwa piąta rozciąga wejściowe 32 kanały na 64 kanały, dzięki temu kolejne
  2 warstwy przyjmujące na wejście te 64 kanały i przekazujące je na wyjście są
  w stanie wydobyć z obrazu cechy o większym poziomie abstrakcji, co znacznie
  zwiększa skuteczność działania sieci.

  Warstwa ósma ogranicza ilość kanałów w sieci z 64 do 32 wyciągając z nich
  cechy najbardziej przydatne do rozwiązania danej problematyki. Kanały te są
  następnie ponownie przetwarzane przez warwę z wielkością filtru 3x3, co ma
  służyć agregacji rozłożonych cech w bardziej spójną całość, która może być
  już składana w pożądane wyjście.

  Kolejne dwie warstwy w sieci są to warstwy konwolucyjne o wielkości filtru 1x1,
  odpowiadają one warstwom gęstym i mają na celu przekonwertowanie wartości
  funkcji aktywacji z poprzednich warstw na wartości kolorów odpowiadających
  pikseli w przestrzeni barw CIELab. Ostatnia warstwa, również z filtrem o
  wielkości 1x1 zwija 32 kanały otrzymywane na wejściu do 2 kanałów odpowiadających
  składowym \textit{A} oraz \textit{B}, które stanowią pożądany rezultat działania
  sieci.

  Po wszystkich, oprócz ostatniej, warstwach konwolucyjnych znajdują się dodatkowo
  warstwa BatchNorm oraz warstwa funkcji aktywacji RELU mające na celu
  ustabilizować proces uczenia oraz zwiększyć jego efektywność.

\subsubsection{BatchNorm} \label{BatchNorm}

  Warstwa Batch Normalization została przedstawiona w 2015 roku przez S. Ioeffe
  oraz C. Szegedy jako odpowiedź
  na problem zmieniającej się podczas uczenia dystrybucji wartości wejść
  każdej z warstw sieci \cite{BatchNorm}. Ma ona na celu usprawnić i ustabilizować
  trening sieci poprzez normalizację wartości podawanych na funkcje aktywacji.
  Zmienna dystrybucja tych wartości znacznie spowalnia i utrudnia proces uczenia
  poprzez potrzebę przemyślanego inicjowania
  wag sieci w celu zwiększenia prawdopodobieństwa nakierowania modelu na pożądane
  rozwiązania w trakcie procesu uczenia oraz przez
  konieczność używania mniejszych wartości współczynnika uczenia, aby
  przeciwdziałać problemom zanikającego oraz wybuchającego gradientu
  \cite{exploding_vanishing_grad}.

  Problemy te zostały już zauważone i opisane w 1994 roku przez Y. Bengio
  oraz jego współpracowników \cite{exploding_vanishing_grad}. Dowodzą oni, że:
  \begin{quote}
    % 'Gradient descent becomes increasingly inefficient when the
    % temporal span of the dependencies increases'
    'Metoda gradientu prostego staje się coraz bardziej nieefektywna, gdy
    rośnie czasowy zakres zależności'
  \end{quote}
  Wskazują także, że problemy powstają podczas treningu DNN w fazie wstecznej
  propagacji błędu, kiedy to gradient pochodzący z głębszych warstw przechodzi
  wielokrotnie przez operacje mnożenia macierzowego. Jeśli wartość gradientu
  jest niewielka, to z każdą operacją mnożenia staje się jeszcze mniejsza, aż
  maleje do takich wartości, które nie umożliwiają modelowi uczenia się, a jeśli
  wartość ta jest wysoka to, wraz z przechodzeniem przez kolejne warstwy, rośnie
  jeszcze bardziej co przy bardzo dużych wartościach może doprowadzić do
  destabilizacji procesu uczenia. Są to zjawiska zdecydowanie niepożądane i z
  tego powodu powstało wiele rozwiązań, aby im przeciwdziałać takich jak
  ograniczanie maksymalnej wartości gradientu (ang. gradient clipping) albo
  zastosowanie warstw BatchNorm.

  Zastosowanie warstw BatchNorm sprawia, że podczas uczenia metodą mini-batch
  (pol. małych paczek) każda paczka jest
  normalizowana w sposób zapewniający zerową wartość średnią oraz
  równą jedności wariancję na przestrzeni wszystkich kanałów wejściowych.
  Zaletą takiego podejścia jest poprawienie przepływu korygującego gradientu
  przez kolejne warstwy sieci podczas fazy wstecznej propagacji błędu. Ponadto
  warstwy BatchNorm zapewniają większą odporność sieci na niekorzystnie zainicjowane
  wagi początkowe modelu.

  Użycie tych warstw w modelu podstawowym tuż za warstwami RELU pozwoliło
  uzyskać bardziej korzystną zbieżność modelu oraz lepsze rezultaty końcowe.
  Ocenione zostało też rozwiązania, w którym warstwy BatchNorm znajdują się
  przed warstwami funkcji aktywacji, lecz dało ono gorsze rezultaty, niż
  podejście wspomniane jako pierwsze.

\subsubsection{Dropout} \label{Dropout}

  W trakcie pracy nad ostatecznym modelem podstawowym sprawdzona została
  skuteczność zastosowania warstw Dropout \cite{dropout}. Warstwy te w trakcie
  treningu dezaktywują część neuron aby przeciwdziałać efektowi przeuczania
  się sieci oraz zapewniać wydajny sposób łączenia wielu różnych architektur
  sieci stworzonych do jednego celu w jednolitą całość o skuteczności większej
  niż poszczególne sieci osobno.

  Wybór neuronów, dla których w danej iteracji treningu nie zostaną
  zaktualizowane wagi odbywa się z pewnym prawdopodobieństwem określonym
  podczas inicjalizacji modelu. Dezaktywowanie neuronów można interpretować jak
  przerywanie tymczasowo wszystkich połączeń danego neuronu, zarówno wejściowych
  jak i wyjściowych. Takie działania są równoznaczne z wyselekcjonowaniem z modelu
  mniejszej sieci i trenowaniu wyłącznie jej w aktualnej iteracji. Wagi tej
  podsieci są wtedy współdzielone z modelem źródłowym.
  Jako rezultat uzyskuje się model o znacznie ulepszonych zdolnościach generalizacji.

  Zastosowanie tych warstw w modelu nie przyniosło wyraźnego polepszenia rezultatów
  sieci. W przypadku danej problematyki oraz obranego podejścia do jej rozwiązania,
  przeuczenie sieci nie stanowi wyraźnego zagrożenia, a zastosowanie Dropout
  wiąże się z utratą części informacji kluczowych do odpowiedniego
  generowanie kolorów dla wejściowych obrazów. Model powinien nauczyć się jak
  największej różnorodności kolorów, a Dropout przeciwdziała uczeniu się
  nadmiernej ilości cech przez sieć, co wpływa niekorzystnie na otrzymywane
  rezultaty końcowe. Podczas testów skuteczności tych warstw zostały one umieszczone
  za warstwami BatchNorm. Wizualizacja skutków tej decyzji wraz z rozważeniem
  pozostałych czynników wpływających na efektywność sieci znajdują się w
  punkcie \ref{Rezultaty} związanym z rezultatami modelu podstawowego.

\subsubsection{Modyfikacja rozdzielczości}

  W modelach FCN powszechnie stosuje się różne metody zmiany rozdzielczości
  kanałów przechodzących przez sieć, najczęściej są do operacje poolingu mające
  na celu zredukować przestrzenną wielkość reprezentacji cech wyciągniętych z
  obrazu poprzez wyciągnięcie najbardziej istotnych wartości funkcji aktywacji z
  określonych obszarów reprezentacji w celu zredukowania rozmiaru
  sieci, a co za tym idzie, zmniejszenia ilości obliczeń koniecznych do wykonania
  przez sieć.

  Ponadto pooling wspomaga adaptację modelu do zmiennego położenia
  kluczowych wzorów rozpoznawanych przez sieć na obrazie wejściowym. Cecha ta
  zwana jest niezmiennością od translacji (ang. transaltion invariance). Jest
  ona rezultatem dokonywania operacji, takich jak wyliczanie wartości maksymalnej
  z poszczególnych obszarów, dzięki którym zmienne położenie wartości
  selekcjonowanej, a co tym idzie, zmienne położenie ekstraktowanej cechy w obrębie
  danego obszaru nie wpływa na końcową postać reprezentacji przestrzennej obrazu
  wejściowego po przejściu przez warstwę poolingu. Przykładowe operacja poolingu
  została przedstawiona na Rysunku \ref{fig:pooling}.

  \begin{figure}[h]
   \centering
   \includegraphics[width=5in]{pooling}
   \caption[Przykładowe operacje warstwy poolingu - źródło: Rysunek własny]{Przykładowe operacje warstwy poolingu}
   \label{fig:pooling}
  \end{figure}

  W modelu końcowym warstwy poolingu nie zostały zastosowane, aby uniknąć utraty
  kluczowych informacji przestrzennych koniecznych to właściwego wygenerowanie
  możliwych barw obrazu.

\subsubsection{Wykorzystywany zbiór treningowy}

  Do uczenia modelu został wykorzystany zbiór danych CIFAR-10 stworzony przez
  A. Krizhevsky oraz zaprezentowany w 2009 roku \cite{cifar-10}.
  Składa się on z 60000 obrazów w przestrzeni kolorów RGB o rozdzielczości 32 x 32 piksele.
  Spośród tych obrazów 10000 z nich zostało wykorzystanych jako zbiór
  walidacyjny do śledzenia skuteczności treningu modelu. Przykładowe obrazy
  ze zbioru danych zostały przedstawione na Rysunku \ref{fig:cifar10}.

  \begin{figure}[h]
   \centering
   \includegraphics[width=5.5in]{cifar10}
   \caption[Przykładowe obrazy z CIFAR-10 - źródło: Rysunek własny]
   {Przykładowe obrazy z CIFAR-10}
   \label{fig:cifar10}
  \end{figure}

  Zaletą CIFAR-10 jest niewielka rozdzielczość obrazów co pozwala na mniej
  złożony model oraz szybszy proces uczenia, który można skutecznie przeprowadzić
  nawet przy ograniczonych możliwościach obliczeniowych. Ponadto zbiór ten składa
  się z obrazów dzielących się na 10 klas, tak mała różnorodność klas, a co za
  tym idzie, stosunkowo niewielka ilość możliwych obiektów pojawiających się
  na zdjęciach powinna ułatwić sieci wyuczenie się właściwych barw dla
  identyfikowanych powierzchni. Z drugiej strony mała ilość klas może niekorzystnie
  wpłynąć na umiejętność generalizacji modelu, jednakże obrazy z CIFAR-10
  przedstawiają zróżnicowane otoczenia zawierające obiekty nie kwalifikujące
  się do żadnej klasy, co powinno umożliwić wyuczenia się generalizacji przez
  sieć.

\subsubsection{Przetwarzanie wstępne danych} \label{Przetwarzanie wstępne danych}

  Przed podaniem na wejście sieci obrazy uczące były wpierw poddawane
  przetwarzaniu wstępnemu mającemu na celu doprowadzić do szybszego oraz
  stabilniejszego treningu modelu. Przetwarzanie wstępne jest kluczowe, gdyż
  wartości o jakie aktualizowane są wagi neuronu zależą w dużej mierze od
  wartości wejść tego neuronu. W przypadku gdy przedziały wartości tych wejść
  nie są jednolite, to może wystąpić duża różnica w tempie aktualizacji wag
  sieci, niektóre wagi będą zmieniane o wiele szybciej niż inne co może spowodować
  destabilizację treningu. Przeskalowanie wszystkich wartości wejściowych do jednakowych
  przedziałów, o niewielkiej wartości maksymalnej i minimalnej oraz wartości
  średniej zbliżonej do zera zmniejsza możliwość wystąpienia tego problemu
  oraz sprzyja ujednoliceniu tempa uczenia się przez sieć rozpoznawania różnych cech.
  Ponadto brak przeskalowania wejść może doprowadzić do zjawisk wybuchającego
  oraz zanikającego gradientu.

  % Przeskalowanie wszystkich wartości wejściowych do jednakowych
  % przedziałów, o niewielkiej wartości maksymalnej i minimalnej oraz wartości
  % średniej zbliżonej do zera, sprzyja ujednoliceniu tempa uczenia się przez sieć
  % rozpoznawania różnych cech.
  \noindent
  W ramach badania rozwiązania przetestowane zostały różne metody przetwarzania
  wstępnego zarówno danych wejściowych jak i pożądanej odpowiedzi:
  \begin{itemize}
  \item Normalizacja danych na przestrzeni całego zbioru danych z użyciem wartości
  maksymalnej oraz minimalnej całego zbioru, tak aby wartości pikseli danej składowej
  dla każdego obrazu zawierały się w przedziale od -0.5 do 0.5.
  \item Standaryzacja danych na przestrzeni całego zbioru danych z użyciem wartości
  średniej oraz odchylenia standardowego całego zbioru, tak aby uzyskać wartość
  średnią równą w przybliżeniu zero oraz jednostkowe odchylenie standardowe.
  \item Zastosowanie rozmycia gaussowskiego(ang. Gaussian blur) o różnej
  wielkości filtru Gaussowskiego.
  % \item Przepuszczenie składowej przez filtr Gaussa o różnej wielkości filtru.
  \end{itemize}

  Rozmycie gaussowskie, zwane także wygładzaniem gaussowskim (ang. Gaussian smoothing),
  jest to operacja polegająca na modyfikacji obrazu z użyciem filtru Gaussa.
  Stosuje się je w celu rozmycia detali na przetwarzanym obrazie, a także by
  ograniczyć ilość występujących na nim zakłóceń oraz szumów. Jest ono powszechnie
  stosowane w fazie przetwarzania wstępnego danych graficznych. W praktyce
  operacja ta sprowadza się do dokonywania splotu kolejnych fragmentów obrazu
  z funkcją Gaussa. Zastosowanie jej na danych wejściowych miało na celu
  zmniejszenie znaczenie detali na obrazie oraz ułatwienie sieci nauczenia się
  rozróżniania rozmaitych powierzchni oraz wzorców.

  Wymienione metody przetwarzania były stosowane w różnych połączeniach oraz
  konfiguracjach zarówno na składowej \textit{L}, jak i składowych \textit{A}
  oraz \textit{B}, a uzyskane wyniki opisane zostały w punkcie \ref{Rezultaty}

\subsubsection{Przetwarzanie końcowe danych}  \label{Przetwarzanie końcowe danych}

  Jeśli w trakcie procesu uczenia były stosowane zabiegi przetwarzania
  wstępnego danych wejściowych to podczas testowania modelu dane testowe również
  muszą być przetworzone w ten sam sposób, aby zapewnić poprawną pracę sieci.
  To samo dotyczy się danych wyjściowych sieci, jeśli podczas treningu z nadzorem
  pożądane wyjście było w pewien sposób przetworzone wstępnie to sieć uczy się
  odtwarzać te wyjście tak samo przetworzone, aby uzyskać oczekiwany efekty
  końcowy należy dokonać operacji odwrotnych do tych zastosowanych w fazie
  przetwarzania wstępnego. Przykładowo jeśli pożądana odpowiedź w procesie
  uczenia była normalizowana to podczas testów sieci jej wyście należy
  zdenormalizować, aby uzyskać oczekiwany rezultat. Jednakże my dla rozważanej
  problematyki proponujemy metodę alternatywną.

  Polega ona na uwydatnianiu kolorów wygenerowanych przez sieć dla wysokich
  wartości naświetlenia - składowej \textit{L}. Może być ona stosowana
  niezależnie od metody przetwarzania wstępnego danych wejściowych. Algorytm
  polega na przeskalowaniu pikseli składowych \textit{A} i \textit{B} tak, aby
  ich wartości mogły pokryć cały dostępny przedział wartości, ale dla danego
  piksela jego przedział jest ograniczony proporcjonalnie do stosunku wartości
  tego piksela w składowej \textit{L} do maksymalnej możliwej wartości pikseli
  składowej \textit{L}.

  Dla przykładu załóżmy, że wartość danego pikselu dla składowej \textit{A}
  wynosi 70, dla składowej \textit{B} wynosi -20, a dla składowej \textit{L} 80.
  Przedział wartości składowych \textit{A} i \textit{B} jest od -127 do 128, a
  składowej \textit{L} od 0 do 100. W następnym kroku znajdowana jest maksymalna
  absolutna wartość składowych \textit{A} i \textit{B} dla aktualnego obrazu.
  Załóżmy, że dla \textit{A} wynosi ona 90, a dla \textit{B} 60.
  Następnie piksele składowych \textit{A} i \textit{B} są dzielone przez
  swoje maksymalne wartości, a następnie rozciągnięte na cały dostępny przedział
  przez pomnożenie przez odpowiedni czynnik, czynnik ten jest z przedziału
  od 0 do 127 i jest wprost proporcjonalny do stosunku aktualnego piksela
  składowej \textit{L} do maksymalnej wartości \textit{L}, oznacza to, że jeśli
  dany piksel \textit{L} jest równy 100 to czynnik jest równy 127, a jeśli
  dany piksel \textit{L} jest równy 50 to wartość czynnika znajduje się w
  połowie swojego przedziału i równy jest 63,5.

  \noindent
  Dla podanych założeń otrzymujemy następujące nowe wartości piksela składowej \textit{A}
  ($p_{n}^{A}$) i piksela składowej \textit{B} ($p_{n}^{B}$):
  \begin{equation}
  p_{n}^{A} = \frac{70}{90} * 127 * \frac{80}{100} = 79.02
  \end{equation}
  \begin{equation}
  p_{n}^{B} = \frac{-20}{60} * 127 * \frac{80}{100} = -33.87
  \end{equation}

  \noindent
  Algorytm konwertowanie pikseli danej składowej można przedstawić wzorem:
  \begin{equation}
  S_{i, j}^{n} = \frac{S_{i, j}^{p}}{max\{abs\{S\}\}} * 127 * \frac{L_{i, j}}{100}
  \end{equation}

  \noindent
  Gdzie:
  \begin{itemize}
    \item $S$ - Konwertowana składowa będąca dwuwymiarowa macierzą pikseli.
    \item $S_{i, j}^{n}$ - Nowa wartość piksela (i, j) dla danej składowej.
    \item $S_{i, j}^{p}$ - Stara wartość piksela (i, j) dla danej składowej.
    \item $L_{i, j}$ - Wartość piksela (i, j) składowej jasności.
  \end{itemize}

  Zastosowanie powyższego algorytmu pozwoliło uzyskać bardziej zadawalające
  rezultaty działania modelu poprzez faworyzowanie kolorów tworzonych przez sieć
  dla powierzchni o dużej wartości jasności, jako, że wysoka jasność oferuje
  bardziej intensywne barwy. Kolorystyka powierzchni ciemnych pozostaje
  przytłumiona, jako, że brak naświetlenia ogranicza natężenie kolorów.

\subsubsection{Augmentacja danych}

  W celu skutecznego treningu modelu potrzebny jest odpowiednio duży i różnorodny
  zbiór treningowy. W obliczu problemu niewystarczającej ilości danych stosuje
  się metody zwane augmentacją danych pozwalające poszerzyć zbiór obrazów uczących
  poprzez dodawanie nowych informacji do obrazów będących podstawą zbioru tworząc
  w ten sposób nowe obrazy, które mogą być wykorzystane w treningu. Augmentacja
  danych przeciwdziała uczeniu się przez sieć nieistotnych wzorów i cech takich jak
  orientacja obiektu, jego umiejscowienie albo rozmiar. Dzięki temu model jest
  w stanie dogłębniej analizować obrazy wejściowe ucząc się rozpoznawać cechy
  o coraz to większym poziomie abstrakcji co przekłada się na o wiele lepiej
  rozwiniętą zdolność modelu do generalizacji danej problematyki.

  \noindent
  Do augmentacji stosuje się proste przekształcenia obrazu takie jak:
  \begin{itemize}
  \item Rotacja obrazu o wybrany kąt.
  \item Odbicie obrazu względem osi pionowej.
  \item Obcinanie skrajnych fragmentów obrazów (ang. crop).
  \item Zmiana odcienia oraz saturacji barw obrazu.
  \item Przybliżanie albo oddalanie treści obrazu.
  \item Modyfikacja rozdzielczości obrazu poprzez jego rozciąganie lub ściskanie.
  \item Tworzenie niewielkich ubytków w obrazach w losowych miejscach (ang. coarse dropout).
  \end{itemize}

  Należy jednak pamiętać, że nie wszystkie metody augmentacji nadają się do każdego zbioru
  danych, kluczowe jest, aby wybrać takie operacje, które poszerzają zbiór
  uczący o dane niosące znaczące oraz sensowne informacje patrząc z punktu wybranej
  problematyki oraz nie przesłaniają wzorców kluczowych do wyuczenia przez sieć.
  W przypadku zagadnienia kolorowania czarno-białych obrazów kluczową informacją
  jest kolor analizowanej powierzchni oraz jej cechy charakterystyczne, z tego
  powodu, do treningu modelu podstawowego nie zostały wykorzystane żadne metody
  augmentacji wpływające na barwy albo jasność obrazów. Wykluczone zostały również takie
  metody jak tworzeniu ubytków w obrazach, gdyż powoduje to niepotrzebną utratę
  informacji.
  W związku z tym, że wykorzystany zbiór CIFAR-10 posiada dużą ilość obrazów
  to w procesie uczenia została wykorzystana jedynie augmentacja poprzez
  odbicie obrazu względem osi pionowej z prawdopodobieństwem równym 50\%.

  % W zagadnieniu kolorowania czarno-białych obrazów informacjami najważniejszymi
  % są wartości pikseli powiązane z daną powierzchnią oraz cechy charakterystyczne
  % tych powierzchni, z tego powodu, aby ograniczyć straty
  % w procesie uczenia nie zastosowano żadnej
  % metody związanej ze zmianą barw obrazu wejściowego, jednocześnie nie można
  % było sobie pozwolić

\subsubsection{Funkcje kosztów} \label{Funkcje kosztów}
  % Bez teorii
  Kluczem do właściwego funkcjonowania modelu jest wybór odpowiedniej funkcji
  kosztu. W przypadku problematyki kolorowania czarno-białych obrazów ważne
  jest, aby wybrana funkcja kosztu uwzględniała specyficzną naturę problemu,
  gdzie dla niektórych przypadków właściwych jest wiele odpowiedzi. Jako
  przykład można podać taki obiekt jak samochód, który może być zarówno
  zielony, czerwony jak i żółty, każdy z tych kolorów powinien być oceniony
  jako właściwy, a wartość błędu wyliczona z użyciem funkcji kosztu dla tak
  wybranych przez sieć kolorów powinna odpowiednio to wskazywać. W ramach
  poszukiwań najbardziej odpowiedniej funkcji kosztu przetestowane zostały
  następujące funkcje.

  \begin{enumerate}[leftmargin=*]
  \item MSELoss (ang. Mean Squared Error Loss) - koszt oparty na błędzie średniokwadratowym
  przedstawiony funkcją:
  \begin{equation}
  Koszt = \frac{1}{n}\sum_{i=0}^{n} (x_{i} - y_{i})^2
  \end{equation}

  % \noindent
  % gdzie $x$ są to składowe \textit{A} oraz \textit{B} przewidziane przez sieć,
  % a $y$ to rzeczywiste \textit{A} i \textit{B}.

  \noindent
  Po dopasowaniu równania MSELoss do rozważanej problematyki otrzyma się:
  \begin{equation}
  Koszt = \frac{1}{n}\frac{1}{m}\sum_{i=0}^{n}\sum_{j=0}^{m} ((A_{i, j}^{P} - A_{i, j}^{R})^2 + (B_{i, j}^{P} - B_{i, j}^{R}))^2
  \end{equation}

  % \noindent
  % gdzie $(A_{i, j}^{R}$ i $(B_{i, j}^{R}$ są to rzeczywiste wartości składowych
  % \textit{A} i \textit{B} dla pikselu obrazy o współrzędnych $(i, j)$, a
  % $(A_{i, j}^{P}$ i $(B_{i, j}^{P}$ są to wartości przewidziane przez sieć.

  \item L1Loss zwany także MAELoss (ang. Mean Absolute Error Loss) - koszt oparty na
  średnim błędzie bezwzględnym dany funkcją:
  \begin{equation}
  Koszt = \frac{1}{n}\sum_{i=0}^{n} |x_{i} - y_{i}|
  \end{equation}
  % gdzie $x$ są to składowe \textit{A} oraz \textit{B} przewidziane przez sieć,
  % a $y$ to rzeczywiste \textit{A} i \textit{B}.

  \noindent
  Równanie L1Loss dla rozważanego zagadnienia:
  \begin{equation}
  Koszt = \frac{1}{n}\frac{1}{m}\sum_{i=0}^{n}\sum_{j=0}^{m} (|A_{i, j}^{P} - A_{i, j}^{R}| + |B_{i, j}^{P} - B_{i, j}^{R}|)
  \end{equation}
  % gdzie $x$ są to składowe \textit{A} oraz \textit{B} przewidziane przez sieć,
  % a $y$ to rzeczywiste \textit{A} i \textit{B}.

  \item SmoothL1Loss - zwany także \textit{Huber loss}, odmiana L1Loss przedstawiona w
  2015 roku przez R. Girshick \cite{SmoothL1Loss}. Jej zaletami są mniejsza
  czułość na elementy odstające (ang. outliner) i zmniejszona szansa na wystąpienie
  zjawiska eksplodującego gradientu.
  SmoothL1Loss przedstawiony jest funkcją:
  \begin{equation}
  Koszt = \frac{1}{n}\sum_{i=0}^{n} z_{i}
  \end{equation}

  \noindent
  gdzie $z_{i}$ dane jest:
  \begin{equation}
  z_{i} = \begin{cases}
               0.5(x_{i} - y_{i})^2 \text{, jeśli } |x_{i} - y_{i}| < 1 \\
               |x_{i} - y_{i}| - 0.5 \text{, w pozostałych przypadkach}
            \end{cases}
  \end{equation}

  \noindent
  Zastosowanie SmoothL1Loss do modelu podstawowego da następującą funkcję:
  \begin{equation}
  Koszt = \frac{1}{n}\sum_{i=0}^{n} (z_{i} + k_{i})
  \end{equation}

  \noindent
  gdzie $z_{i}$ dane jest:
  \begin{equation}
  z_{i} = \begin{cases}
               0.5(A_{i, j}^{P} - A_{i, j}^{R})^2 \text{, jeśli } |A_{i, j}^{P} - A_{i, j}^{R}| < 1 \\
               |A_{i, j}^{P} - A_{i, j}^{R}| - 0.5 \text{, w pozostałych przypadkach}
            \end{cases}
  \end{equation}

  \noindent
  a $k_{i}$ dane jest:
  \begin{equation}
  z_{i} = \begin{cases}
               0.5(B_{i, j}^{P} - B_{i, j}^{R})^2 \text{, jeśli } |B_{i, j}^{P} - B_{i, j}^{R}| < 1 \\
               |B_{i, j}^{P} - B_{i, j}^{R}| - 0.5 \text{, w pozostałych przypadkach}
            \end{cases}
  \end{equation}

  \end{enumerate}
  \noindent
  Gdzie:
  \begin{itemize}
    \item $x$ są to składowe \textit{A} oraz \textit{B} przewidziane przez sieć.
    \item $y$ to rzeczywiste składowe \textit{A} i \textit{B}.
    \item $A_{i, j}^{R}$ i $B_{i, j}^{R}$ są to rzeczywiste wartości składowych
    \textit{A} i \textit{B} dla pikselu obrazy o współrzędnych $(i, j)$.
    \item $A_{i, j}^{P}$ i $B_{i, j}^{P}$ są to wartości składowych
    \textit{A} i \textit{B} przewidziane przez sieć dla pikselu obrazy o
    współrzędnych $(i, j)$.
  \end{itemize}

  \noindent
  Szczegółowy opis skutków zastosowanie poszczególnych funkcji kosztów umieszczony
  został w punkcie \ref{Rezultaty}.

\subsubsection{Funkcje aktywacji}
  % RELU
  W wyborze odpowiednej funkcji aktywacji kierowaliśmy się badaniami przeprowadzonymi
  przez B. Xu oraz jego współpracowników \cite{evaluatiuon_of_relu}. Testowali
  oni skuteczność takich funkcji jak ReLU, Leaky ReLU (pol. przepuszczające ReLU),
  PReLU (ang. Parametric ReLU) oraz RReLU (ang. Randomized Leaky ReLU) w
  zagadnieniu klasyfikacji treści obrazów. Bazując na otrzymanych przez nich
  rezultatach zdecydowaliśmy się wykorzystać w rozważanej problematyce funkcję
  aktywacji ReLU, przedstawioną na Rysunku \ref{fig:relu}

  \begin{figure}[h]
   \centering
   \includegraphics[width=4in]{relu}
   \caption[Funkcja aktywacji ReLU - źródło: \url{https://pytorch.org/docs/stable/_images/ReLU.png}]
   {Funkcja aktywacji ReLU}
   \label{fig:relu}
  \end{figure}

  Funkcja ta posiada wiele zalet takich jak przyspieszenie procesu zbieganie się stanu
  sieci do stanu pożądanego poprzez brak ograniczeń na maksymalną wartość
  funkcji oraz niska złożoność obliczeniowa.
  Ponadto funkcja ta, w związku z zerową wartością dla ujemnych argumentów,
  zapewnia aktywację neuronów modelu tylko wtedy, gdy analizują one wzorce
  kluczowe dla nich samych oraz danego zagadnienia, zwiększa to odporność modelu na
  szumy wejściowe oraz przeuczanie, a także poprawia zdolności predykcyjne sieci.

  Jednakże stosowanie ReLU tworzy zagrożenie blokowanie się procesu uczenia,
  jeśli dojdzie do sytuacji, w której wagi modelu dojdą do stanu, gdzie wartość
  aktywacji będzie zbliżona do zera. Spowoduje to zerową wartość gradientu
  podczas fazy wstecznej propagacji błędu powodując wstrzymanie się procesu
  aktualizowania wag modelu, a w rezultacie, nieefektywny proces uczenia.
  Pomimo to jednak zostaliśmy przy wyborze funkcji ReLU wierząc, że pozostałe
  zabiegi takie jak zastosowanie warstw BatchNorm pozwolą zminimalizować
  niepożądane efekty tego zjawiska.

\subsubsection{Algorytmy optymalizacyjne} \label{Algorytmy optymalizacyjne}
  % Bez teorii
  % TODO: Może dać bardziej szczegółowy opis
  % TODO: Zacytować twórców Adam i Adagrad, linki do paperów są w dokumentacji
  % pytorcha o tych algorytmach
  Podczas planowanie treningu modelu należy zdecydować się na odpowiedni
  algorytm optymalizacyjny, właściwa decyzja pozwala uniknąć zatrzymania się
  procesu uczenia w lokalnych minimach co zwiększa końcową dokładność oraz
  skuteczność modelu.
  Podczas badań przetestowane zostały różne algorytmy optymalizacyjne,
  a uzyskane rezultaty zostały szczegółowo opisane oraz porównane w punkcie
  \ref{Rezultaty}.\newline
  Wykorzystane zostały następujące algorytmy:
  \begin{itemize}
    \item Adam (ang. Adaptive Moment Estimation)
    \item Adagrad (ang. Adaptive Gradient Algorithm)
    \item SGD (ang. Stochastic Gradient Descent)
  \end{itemize}

  \noindent
  Przeprowadzone zostały także poszukiwania najbardziej odpowiednich hiperparametrów
  dla wymienionych algorytmów w celu osiągnięcia jak największej ich skuteczności
  w rozwiązaniu rozważanej problematyki.

% TODO: Od tego momentu było ekspres pisanie, wiec może do zredagowania
\subsubsection{Trening}

Model trenowaliśmy paczkami o wielkości 128 obrazów, cała epoka składała się z
50000 obrazów. Przed każdą epoką zbiór treningowy był przetasowywany, aby
przeciwdziałać przeuczaniu się sieci. Aby znaleźć najbardziej zadowalające rozwiązanie
przetestowane zostały różne konfiguracje treningowe, możliwe zmienne elementy
konfiguracji to algorytmy optymalizacyjne opisane w punkcie
\ref{Algorytmy optymalizacyjne}, funkcje kosztów opisane w punkcie \ref{Funkcje kosztów},
rodzaje przetwarzania wstępnego opisane w punkcie \ref{Przetwarzanie wstępne danych} oraz
skutki zastosowanie takich warstw jak Dropout (punkt \ref{Dropout}) oraz BatchNorm
(punkt \ref{BatchNorm}). Jako funkcja aktywacji wybrana została funkcja ReLU.
Dla różnych konfiguracji została także przedstawiona charakterystyka porównawcza.


\subsubsection{Rezultaty} \label{Rezultaty}

 Uzyskane wyniki są w znacznej mierze zależne od wybranej konfiguracji.
 Najbardziej satysfakcjonujące rezultaty uzyskane przez model zostały
 przedstawione na Rysunku \ref{fig:best_result}

 \begin{figure}[ht]
   \centering
   \subfloat[Obraz spoza zbioru uczącego]{\includegraphics[width = 4.5in]{best_results_plane}} \\
   \subfloat[Obraz spoza zbioru uczącego]{\includegraphics[width = 4.5in]{best_results_krajobraz}} \\
   \subfloat[Obraz ze zbioru uczącego]{\includegraphics[width = 4.5in]{best_results_horse}} \\
   \caption[Efekty działania sieci - źródło: Rysunek własny bazujący na:
   \url{https://fr.m.wikipedia.org/wiki/Fichier:An_F-A-18C_Hornet_launches_from_the_flight_deck_of_the_conventionally_powered_aircraft_carrier.jpg},
   \url{https://pl.wikipedia.org/wiki/Plik:PL_Bagno_Calowanie_2.jpg}, \cite{cifar-10}]
   {Rezultaty.}
   \label{fig:best_result}
 \end{figure}

 \noindent
 Efekt ten został uzyskany przy następującej konfiguracji:
 \begin{itemize}
   \item Funkcja kosztu MSELoss.
   \item Funkcja aktywacji ReLU.
   \item Algorytm optymalizacyjny Adam.
   \item Normalizacja składowej \textit{L} podczas przetwarzania wstępnego.
   \item Standaryzacja składowych \textit{A} i \textit{B} podczas przetwarzania wstępnego.
   \item Nie zastosowanie rozmycia Gaussowskiego.
   \item Zastosowanie warstwy BatchNorm oraz nie zastosowanie warstwy Dropout.
   \item Zastosowanie zaproponowanego przez nas algorytmu przetwarzania końcowego.
 \end{itemize}

 \noindent
 Na podstawie obrazów na Rysunku \ref{fig:best_result} oraz przeprowadzonych
 testów można wywnioskować, że model nauczył się najlepiej kolorować powierzchnie
 o zazwyczaj stałej barwie takie jak niebo albo trawa, widać, że ich barwy są
 wyjątkowo intensywne oraz jednoznacznie wskazują co to są za powierzchnie.
 Gorzej rzecz się ma z obiektami o zmiennych barwach takimi jak na przykład
 samochody, w związku z ich zmienną barwą na przestrzeni całego zbioru danych
 model, chcąc zminimalizować błąd liczony przez funkcję kosztu, uśredniał kolor
 tych obiektów co skutkuje ich przytłumionymi barwami przechodzącymi często
 w odcienie brązu. Z tego powodu pomniejsze detale na kolorowanych zdjęciach
 tracą swoje barwy, efekty tego zjawiska można zaobserwować analizując
 stateczniki odrzutowca ukazanego na obrazie \textit{(a)} Rysunku \ref{fig:best_result}.

  \noindent
  W dalszej części sekcji opisany został wpływ poszczególnych elementów konfiguracji.

  Największy wpływ na uzyskiwane rezultaty ma zastosowanie algorytmu przetwarzania
  końcowego opisanego w punkcie \ref{Przetwarzanie końcowe danych}. Pomógł on
  znacznie wzbogacić kolory generowane przez sieć. Odpowiednie porównanie
  zostało przedstawione na Rysunku \ref{fig:trick}. Wadami tej metody są generowane
  momentami zbyt intensywne barwy zdjęć oraz bardziej widoczne czasami
  zdarzające się "wycieki"
  kolorów jednych powierzchni na inne, przykładowo czerwony kolor z samochodu
  wychodzi poza powierzchnię samochodu i barwi fragmenty trawy przylegające do
  samochodu na czerwono. Efekt ten jest znacznie mniej widoczny przy zdjęciach
  o wysokiej rozdzielczości, co sprawia, że w zastosowaniu praktycznym ta
  wada modelu jest do zaakceptowania. Na Rysunku \ref{fig:trick} widać to
  poprzez drobne obszary żółci na granicy nieba i drzew.

  \begin{figure}[h]
   \centering
   \includegraphics[width=6in]{porownania_trick_krajobraz}
   \caption[Efekt zastosowania algorytmu przetwarzania końcowego - źródło
   rysunek własny na podstawie: \url{https://pl.wikipedia.org/wiki/Plik:PL_Bagno_Calowanie_2.jpg}]
   {Efekt zastosowania algorytmu przetwarzania końcowego}
   \label{fig:trick}
  \end{figure}

  \textless Skutki rożnych funkcji kosztu \textgreater

  \textless Skutki warstwy Dropout \textgreater

  \textless Skutki rożnych metod przetwarzania wstępnego \textgreater

  \textless Skutki rożnych algorytmów optymalizacyjnych \textgreater

  \textless Skutki rożnych algorytmów optymalizacyjnych \textgreater


\section{Podsumowanie}

  Celem zaprezentowanego projektu inżynierskiego było stworzenie oprogramowania,
  opartego na technologii sztucznych sieci neuronowych, umożliwiającego
  wszechstronną edycję obrazów cyfrowych. Przygotowane rozwiązania należało
  przetestować, a zastosowane implementacje przeanalizować pod kątem
  skuteczności działania ze względu na różnorodny dobór parametrów oraz
  architektury sieci.

  Dodatkowo, w celu uproszczenia i przyspieszenia wykonywanych eksperymentów,
  postawione zostało zadanie przygotowania frameworka w języku Python, umożliwiającego
  łatwe komponowanie oraz testowanie narzędzi opartych o mechanizmy sieci
  neuronowych.

  Cel ten został zrealizowany, a zaimplementowane rozwiązanie otrzymało nazwę
  \textit{TorchFrame}. Szczegółowy opis działania frameworka zawarty został w
  podrozdziale \ref{TorchFrame}. Opracowane oprogramowanie kontroluje przepływ danych
  w procesie uczenia sieci neuronowych udostępniając użytkownikom szeroki zakres gotowych do
  użycia metod konwersji obrazu. Ułatwia to znacznie zagadnienia
  wstępnego oraz wtórnego przetwarzania danych treningowych. Dodatkowo uproszczone
  zostało zagadnienie parametryzacji całego procesu. Użytkownik \textit{TorchFrame}
  definiuje wszystkie istotne elementy wykorzystując pojedynczy plik \textit{JSON}, którego
  struktura została szczegółowo opisana w ramach tej pracy. Komponentowa budowa
  frameworka umożliwiła wydzielenie części dedykowanej szeroko pojętym testom
  zaimplementowanych rozwiązań. W ramach przeprowadzonych eksperymentów zdefiniowane
  zostały przykładowe struktury programistyczne dostosowujące \textit{TorchFrame}
  do dedykowanych problemów. Jego elastyczna struktura umożliwia jednak
  intuicyjne implementowanie rozmaitych modyfikacji ograniczonych jedynie
  wyobraźnią użytkowników.

  Przygotowane w ten sposób narzędzie posłużyło w realizacji wspomnianego oprogramowania
  przeznaczonego do edycji obrazów cyfrowych. Realizacja tego zadania przedstawiona
  została w dwóch fazach.

  W podrozdziale \ref{Filtry_AI} opracowana została seria filtrów wzorowanych na
  klasycznych metodach przetwarzania obrazów, takich jak sepia, czy wykrywający
  krawędzie filtr Sobela-Feldmana. Każda z oryginalnych metod odtworzona
  została z wykorzystaniem jak najprostszych modeli neuronowych sieci splotowych.
  Uzyskane rezultaty przeanalizowane zostały pod kątem skuteczności działania
  w odniesieniu do zastosowanych hiperparametrów oraz metod przetwarzania danych
  treningowych. Kluczowym elementem tego rozdziału było wskazanie podobieństwa
  między klasycznymi metodami filtracji, a używanymi powszechnie sieciami
  splotowymi, zbudowanymi z warstw neuronów nazywanych właśnie filtrami.

  Rozdział \ref{image_colorization} przedstawia próby automatycznego kolorowania
  czarno-białych obrazów.
  Zagadnienie to zostało rozwiązane z użyciem dwóch różnych modeli, autorskiego modelu
  prostego oraz bardziej zaawansowanego modelu zaimplementowanego z użyciem technik
  przeniesienia uczenia. Dla modelu autorskiego zostały przeprowadzone szczegółowe
  badania dotyczące zależności wyników od poszczególnych parametrów architekury
  sieci, jak i konfiguracji procesu uczenia. Rezultaty końcowe uzyskane z użyciem
  tego modelu były zadowalające, ale mocno zależne od wybranych parametrów.
  Dowodzi to, że odpowiednio skonfigurowane sieci splotowe mogą być skutecznie
  zastosowane w tym zagadnieniu, aczkolwiek to model złożony pozwolił osiągnąć
  największy sukces.
  Idea tego rozwiązania opiera się na integracji cech obrazu średniego oraz wysokiego
  poziomu uzyskiwanych z użyciem złożonej sieci splotowej wytrenowanej pierwotnie do
  zadania klasyfikacji. Cechy te, po przejściu przez proces fuzji, są następnie
  wykorzystywane przez sieć dekonwolucyjną do predykcji prawdopodobnych barw dla
  obrazu wejściowego. Dzięki tym dodatkowym informacjom o obrazie udało się
  znacznie zwiększyć efektywność procesu kolorowania, a co za tym idzie,
  wiarygodność generowanych barw. Pokolorowane obrazy w wielu przypadkach
  wyglądały tak realistycznie, że niemożliwym było zauważenie, że ich kolory zostały
  wygenerowane automatycznie, co
  świadczy o nadzwyczaj wysokiej skuteczności bardziej zaawansowanego modelu.

  Przeprowadzone eksperymenty i uzyskane w ich wyniku rezultaty jasno
  wskazują na użyteczność sztucznych sieci neuronowych w procesie edycji obrazów.
  Ich zdolności adaptacyjne sprawiają, że są one w stanie odtworzyć niemal każde
  klasyczne rozwiązanie, a umiejętność rozpoznawania i wykorzystywania
  skomplikowanych wzorców czyni je ponadto niezastąpionymi w sytuacjach, w których
  zawodzą tradycyjne metody.

  Należy jednak pamiętać, że sieci neuronowe wciąż znajdują się w fazie rozwoju,
  a skuteczność ich działania bardzo często nie dorównuje stosowanym powszechnie
  algorytmom. Dobrym przykładem takiego stanu rzeczy są zaprezentowane filtry \textit{AI}.
  Zastosowanie prostych masek filtrujących dało w tym przypadku lepsze rezultaty niż,
  wymagające znacznych nakładów pracy, sieci splotowe, mimo przeprowadzenia licznych
  eksperymentów i prób dla różnorodnych zestawów hiperparametrów.

  Z drugiej strony, zaprezentowane rozwiązanie inteligentnego kolorowania
  czarno-białych obrazów pokazało, że zastosowanie złożonych architektur sieci
  neuronowych, a także odpowiednio duży nakład pracy włożony w proces treningowy,
  mogą doprowadzić do rozwiązania nieosiągalnego innymi metodami.

  Sztuczna inteligencja bez wątpienia stanowi fenomen w świecie nauki. Nie należy jednak
  traktować jej jak młotka, dla którego każdy problem wygląda jak gwóźdź, ale warto
  nieustannie śledzić dynamiczny rozwój tej dziedziny wiedzy. Doświadczenie pokazuje,
  że znajduje ona zastosowanie w coraz szerszym gronie aspektów życia codziennego.
  Bardzo często nie jesteśmy nawet świadomi jej obecności wokół nas. W związku z
  zachodzącymi zmianami nie należy pytać czy sztuczna inteligencja dorówna kiedyś
  innym, konkurencyjnym metodom. Należy zadać sobie pytanie kiedy to się stanie.

\begin{thebibliography}{99} %{99} numery co najwyżej dwucyfrowe

  \bibitem{cnn} T.~Liu, S.~Fang, Y.~Zhao, P.~Wang, J.~Zhang:
  \emph{Implementation of Training Convolutional Neural Networks}, arXiv ('2015), s.2.

  \bibitem{deeplearn} I.~Goodfellow, Y.~Bengio, A.~Courville:
  \emph{Deep Learning}, ('2016), s.342., s.82.

  \bibitem{gan} I. J.~Goodfellow, J.~Pouget-Abadie, M.~Mirza, B.~Xu, D.~Warde-Farley, S.~Ozair, A.~Courville, Y.~Bengio:
  \emph{Generative Adversarial Networks}, arXiv ('2014)

  \bibitem{colorful_image_colorization} R. Zhang, P. Isola, A. A. Efros:
  \emph{Colorful Image Colorization}, arXiv ('2016)

  \bibitem{image_style_transfer} L. A. Gatys, A. S. Ecker, M. Bethge:
  \emph{Image Style Transfer Using Convolutional Neural Networks}, IEEE ('2016)

  \bibitem{vgg_network} K. Simonyan, A. Zisserman:
  \emph{Very Deep Convolutional Networks For Large-Scale Image Recognition}, ('2016)

  \bibitem{gan_editing} G. Perarnau, J. van de Weijer, B. Raducanu, J. M. Álvarez:
  \emph{Invertible Conditional GANs for image editing}, arXiv ('2016)

  \bibitem{neural_photo_editor} A. Brock, T. Lim, J.M. Ritchie,
  N. Weston:
  \emph{Neural Photo Editing With Introspective Adversarial Networks}, arXiv ('2017)

  \bibitem{vae} D. P. Kingma, M. Welling:
  \emph{Auto-Encoding Variational Bayes}, arXiv ('2014)

  \bibitem{pytorch} A. Paszke, S. Gross, S. Chintala, G. Chanan,
  E. Yang, Z. DeVito, Z. Lin, A. Desmaison, L. Antiga, A. Lerer:
  \emph{Automatic differentiation in PyTorch}, 31st Conference on Neural Information
  Processing Systems (NIPS 2017), Long Beach, CA, USA

  \bibitem{neuralsmithing}  R. Reed,  R. J MarksII:
  \emph{Neural Smithing: Supervised Learning in Feedforward Artificial Neural Networks},
  Paperback – February 17, 1999, s.155.

  \bibitem{systematic_introduction} R. Rojas:
  \emph{Neural Networks - A Systematic Introduction},
  Springer-Verlag, Berlin, New-York, 1996, s.161.

  \bibitem{Momentum} J. López, Á. Barbero, J. R. Dorronsoro:
  \emph{Momentum Acceleration of Least–Squares Support Vector Machines},
  International Conference on Artificial Neural Networks 2011

  \bibitem{Nesterov_momentum} I. Sutskever, J. Martens, G. Dahl, G. Hinton:
  \emph{On the importance of initialization and momentum in deep learning},
  University of Toronto, 2013

  \bibitem{SGD} J. Kiefer, J. Wolfowitz:
  \emph{Stochastic Estimation of the Maximum of a Regression Function},
  Ann. Math. Statist. 23 (1952), no. 3, 462--466.

  \bibitem{Adagrad} J. Duchi, E. Hazan, Y. Singer:
  \emph{Adaptive Subgradient Methods for Online Learning and Stochastic Optimization},
  Journal of Machine Learning Research 12 (2011) 2121-2159

  \bibitem{Adadelta} M. D. Zeiler:
  \emph{AdaDelta: An Adaptive Learning Rate Method},
  arXiv 2012

  \bibitem{Adam} D. P. Kingma, J. L. Ba:
  \emph{Adam: A Method for Stochastic Optimization},
  arXiv 2014

  \bibitem{sobel} I. Sobel:
  \emph{An Isotropic 3x3 Image Gradient Operator},
  ResearchGate, 2014, s.4.

  \bibitem{highpass_filter} A. Makandar, B. Halalli:
  \emph{Image Enhancement Techniques using Highpass and Lowpass Filters},
  International Journal of Computer Applications, 2015, s.13.

  \bibitem{BatchNorm} S. Ioffe, C. Szegedy:
  \emph{Batch Normalization: Accelerating Deep Network Training by Reducing
  Internal Covariate Shift}, arXiv ('2015)

  \bibitem{exploding_vanishing_grad} Y. Bengio, P. Simard, P. Frasconi:
  \emph{Learning long-term dependencies with gradient descent is difficult},
  IEEE ('1994)

  \bibitem{dropout} N. Srivastava, G. Hinton, A. Krizhevsky,
  I. Sutskever, R. Salakhutdinov:
  \emph{Dropout: A Simple Way to Prevent Neural Networks from Overfitting},
  Journal of Machine Learning Research ('2014)

  \bibitem{cifar-10} A. Krizhevsky:
  \emph{Learning Multiple Layers of Features from Tiny Images},
  Master’s thesis, Department of Computer Science, University of Toronto ('2009)

  \bibitem{SmoothL1Loss} R. Girshick:
  \emph{Fast R-CNN},
  arXiv ('2015)

  \bibitem{evaluatiuon_of_relu} B. Xu, N. Wang, T. Chen, M. Li:
  \emph{Empirical Evaluation of Rectied Activations in Convolution Network},
   arXiv ('2015)

  \bibitem{deconvolution_colorization} L. Melas-Kyriazi, G. Han:
  \emph{Combining Deep Convolutional Neural Networks with Markov Random
  Fields for Image Colorization}, arXiv ('2016)

  \bibitem{ResNet} K. He, X. Zhang, S. Ren, J. Sun:
  \emph{Deep Residual Learning for Image Recognition}, arXiv ('2015)

  \bibitem{ImageNet}
  J. Deng, W. Dong, R. Socher, L.-J. Li, K. Li, L. Fei-Fei.:
  \emph{ImageNet: A large-scale hierarchical image database}, Conference on Computer Vision and Pattern Recognition, nieefektywna ('2009)

\end{thebibliography}


\section*{Załączniki}
\addcontentsline{toc}{section}{Załączniki}  % Dodanie to TOC
  \tab Załączniki i dodatki\:


\end{document}
