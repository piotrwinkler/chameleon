\documentclass[12pt]{article}
\usepackage[utf8]{inputenc}
\usepackage[MeX]{polski}
\usepackage{graphicx}
\usepackage{amsmath,amssymb,amsfonts}

%\renewcommand{\chaptername}{Rozdział}
%\renewcommand{\contentsname}{Spis treści}
%\renewcommand{\figurename}{Rys.}
%\renewcommand{\tablename}{Tab.}
%\renewcommand{\listfigurename}{Spis rysunków}
%\renewcommand{\listtablename}{Spis tabel}
%\renewcommand{\bibname}{Bibliografia}
\newcommand\tab[1][1cm]{\hspace*{#1}}

\pagestyle{headings}


\begin{document}
\begin{titlepage}
  \begin{center}

    \vspace*{1cm}
    \Huge
    \textbf{POLITECHNIKA GDAŃSKA}
    \vspace{0.5cm}
    \LARGE
    Katedra Systemów Decyzyjnych i Robotyki

    \vspace{1.5cm}
    \textbf{PRACA INŻYNIERSKA}
    \\[0.5cm]
    \textbf{Zastosowanie sieci neuronowych do edycji obrazów}

    \vspace{2.5cm}
    \Large
    \textbf{Autorzy}\\
    Piotr Winkler\\
    Bartosz Bieliński

    \vspace{3.5cm}
    Gdańsk 2019

  \end{center}
\end{titlepage}

\newpage
  \tableofcontents

\newpage
  \section{Wstęp}
  \tab Sztuczne sieci neuronowe sięgają swym początkiem lat 40. XX wieku. Historia ich rozwoju odnotowała trzy okresy, w których rozwiązania te odbijały się szerokim echem w środowisku naukowym.
  \newline \tab Pierwszy model neuronu, a potem perceptron zapoczątkowały rozwój tej dziedziny nauki, jednak pierwsze sieci jednowarstwowe nie były w stanie rozwiązywać złożonych problemów. Przeszkodę nie do pokonania stanowiła dla nich nawet prosta funkcja logiczna XOR. Z tego powodu badania seci neuronowych zostały na długi czas porzucone.
  \newline \tab Pojawienie się algorytmu wstecznej propagacji błędów pozwalającego skutecznie uczyć wielowarstwowe sieci neuronowe ponownie wzmogło zainteresowanie tematem, jednak tym razem na drodze postępowi stanęły ograniczenia technologiczne ówczesnych czasów.
  \newline \tab Wreszcie wraz z nadejściem XXI wieku postępujący rozwój komputerów oraz internetu umożliwił sztucznym sieciom neuronowym rozwinięcie skrzydeł. Wejście w erę "big data" otworzyło dostęp do olbrzymich zbiorów danych niezbędnych do treningu, a pojawienie się wysokowydajnych jednostek obliczeniowych pozwoliło znacznie ten proces skrócić.
  \newline \tab Zapoczątkowany w ten sposób rozwój trwa do dnia dzisiejszego. Sztuczne sieci neuronowe odnajdują zastosowanie w wielu dziedzinach życia i nauki. Grają w gry, przeprowadzają symulacje, przewidują i prognozują zachowanie rynku, czy pogody, analizują i przetwarzają obrazy cyfrowe.
  \newline \tab Z punktu widzenia niniejszej pracy największe znaczenie ma oczywiście ostatni z wymienionych punktów. Zdefiniowanie sieci nauronowych, jako matematycznych modeli obliczeniowych ujawnia ich naturalne predyspozycje do pracy na obrazach cyfrowych. W praktyce stanowią one bowiem zbiór liczb, wartości poszczególnych pikseli, które sieć neuronowa jest w stanie analizować, przetwarzać i zmieniać.

\newpage
  \section{Przegląd rozwiązań}
  \tab Lata rozwoju sztucznych sieci neuronowych zaowocowały powstaniem wielu technik służących do analizy i edycji obrazów. W poniższym rozdziale zaprezentowane zostaną, oraz pokrótce opisane, najważniejsze i najciekawsze przykłady, z których część znajdzie rozwinięcie w dalszej części tej pracy.

  \subsection{Sieci splotowe}
  \tab

  \subsection{Modele generatywne}
  \tab Koncepcja modeli generatywnych, w skrócie GANów, przedstawiona została w 2014 roku przez Iana Goodfellow oraz jego współpracowników na uniwersytecie w Montrealu \cite{gan}. Modele te stanowią połączenie dwóch głębokich sieci neuronowych działających przeciwstawnie do siebie nawzajem.
  \newline \tab Pierwsza sieć to tak zwany generator. W odniesieniu do tematu pracy, jego działanie polega na generowaniu nowych obrazów, lub ich fragmentów na podstawie wektora szumów.
  \newline \tab Obrazy te przekazywane są, równolegle z zestawem obrazów prawdziwych, do dyskryminatora stanowiącego drugą część modelu GAN. Działanie tej sieci neuronowej polega na określeniu (w skali 0 do 1), w jakim stopniu produkty wyjściowe generatora odpowiadają obrazom rzeczywistym.
  \newline \tab W opisanym modelu występuje zatem podwójna pętla sprzężenia zwrotnego. Dyskryminator określa autentyczność obrazów porównująć je ze zdefiniowaną odgórnie bazą danych. Z kolei generator otrzymuje informację o skuteczności swojego działania ze strony dyskryminatora.
  \newline \tab Model generatywny znajduje się w stanie ciągłego konfliktu. Generator dąży do jak najdokładniejszego fałszowania obrazów w celu oszukania dyskryminatora, którego celem jest z kolei jak najdokładniejsze wykrywanie podróbek. Obie sieci neuronowe nieustannie dążą do osiągnięcia przewagi nad rywalem w procesie treningu. Ciągła rywalizacja sprawia, że zarówno generator, jak i dyskryminator zyskują coraz wyższą skuteczność działania.
  \newline \tab W praktyce modele generatywne są w stanie naśladować dowolną dystrybucję danych. Są w stanie kreować światy podobne do naszego w zakresie obrazu, dźwięku czy mowy. Można powiedzieć, że są to prawdziwi syntetyczni artyści.

\newpage
\begin{thebibliography}{99} %{99} numery co najwyżej dwucyfrowe
  \bibitem{gan} Ian J.~Goodfellow, Jean~Pouget-Abadie, Mehdi~Mirza, Bing~Xu, David~Warde-Farley, Sherjil~Ozair, Aaron~Courville, Yoshua~Bengio:
  \emph{Generative Adversarial Networks}, ('2014)
\end{thebibliography}

\end{document}
