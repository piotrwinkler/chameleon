\section*{Streszczenie}

  Tematem pracy jest zbadanie możliwości zastosowania sieci neuronowych do
  edycji obrazów. Głównym celem było stworzenie narzędzi opartych na wyuczonych
  sieciach neuronowych, służących do odpowiedniego przetwarzania i
  modyfikowania obrazu. Następnie skuteczność tych narzędzi została oceniona i
  porównana z rozwiązaniami opartymi na klasycznych metodach przetwarzania obrazu.

  Zbadane zostały dwie problematyki, automatyczne kolorowanie czarno-białych obrazów
  oraz nakładanie na obraz prostych filtrów takich jak wykrywający
  krawędzie filtr Sobela-Feldmana. Opisy zaimplementowanych rozwiązań zostały
  umieszczone w odpowiednich podrozdziałach
  rozdziału \ref{zaimplementowane rozwiazania}.

  Uzyskane wyniki pozwoliły dojść do wielu konstruktywnych obserwacji. W przypadku obu
  problematyk udało się osiągnąć rozwiązania oparte na sztucznych sieciach
  neuronowych.

  W celu ułatwienia przeprowadzania badań został opracowany autorski framework
  o nazwie \textit{TorchFrame}. Ma on za zadanie kontrolować
  przepływ danych w procesie uczenia i testowania badanych modeli, a co za tym
  idzie, ograniczyć konieczność ingerencji ze strony użytkownika, przy jednoczesnym
  zapewnieniu swobody w prowadzonych eksperymentach i implementowanych modyfikacjach danych. Przekłada
  się to na znacznie wydajniejszy oraz mniej złożony proces trenowania
  różnorodnych architektur sieci neuronowych.

  W przypadku prostych filtrów obrazu rozwiązania
  oparte na sieciach splotowych okazały się wolniejsze i bardziej pracochłonne w
  implementacji niż metody klasyczne przy porównywalnych rezultatach. Jednakże
  z ich pomocą udało się udowodnić olbrzymią uniwersalność sieci splotowych
  mogących znaleźć zastosowanie w zagadnieniach związanych z prostym
  przetwarzaniem obrazu. Ukazane zostało ponadto podobieństwo między tradycyjnymi
  metodami filtracji, a sposobem działania neuronów tworzących sieci konwolucyjne.

  Zagadnienie automatycznego kolorowania czarno-białych
  obrazów zostało rozwiązane z użyciem dwóch różnych modeli, autorskiego modelu
  prostego oraz bardziej zaawansowanego modelu zaimplementowanego z użyciem technik
  przeniesienia uczenia. Dla modelu autorskiego zostały przeprowadzone szczegółowe
  badania dotyczące zależności wyników od poszczególnych parametrów architekury
  sieci jak i konfiguracji procesu uczenia. Rezultaty końcowe uzyskane z użyciem
  modelu autorskiego były zadowalające, ale mocno zależne od wybranych parametrów.
  Dowodzi to, że odpowiednio skonfigurowane sieci splotowe mogą być skutecznie
  zastosowane w tym zagadnieniu, aczkolwiek to model złożony pozwolił osiągnąć
  największy sukces.
  Idea tego rozwiązania opiera się na integracji cech obrazu średniego oraz wysokiego
  poziomu uzyskiwanych z użyciem złożonej sieci splotowej wytrenowanej pierwotnie do
  zadania klasyfikacji. Cechy te, po przejściu przez proces fuzji, są następnie
  wykorzystywane przez sieć dekonwolucyjną do predykcji prawdopodobnych barw dla
  obrazu wejściowego. Dzięki tym dodatkowym informacjom o obrazie udało się
  znacznie zwiększyć efektywność procesu kolorowania, a co za tym idzie,
  wiarygodność generowanych barw.

  Badania przeprowadzone w ramach tej pracy dyplomowej są dowodem na to, że
  sieci neuronowe mogą być skutecznie zastosowane jako narzędzia do wszechstronnej
  edycji obrazu. Często są one jedynym dostępnym rozwiązaniem jeśli problematyka
  jest nadzwyczaj złożona. Jednakże, pomimo niezwykłych możliwości sieci
  neuronowych, ich sukces zależy w dużej mierze od dobrze przemyślanego
  wyboru stosowanej architektury oraz poprawnie przeprowadzonego
  procesu uczenia.

  \bigskip

  \noindent\textbf{Słowa kluczowe:} sieć neuronowa, przetwarzanie obrazu,
  konwolucyjna sieć neuronowa, splotowa sieć neuronowa,
  głęboka sieć neuronowa, filtry obrazu, automatyczne kolorowanie czarno-białych
  obrazów, platforma programistyczna

  \bigskip

  \noindent\textbf{Dziedzina nauki i techniki zgodna z OECD:} Nauki
  inżynieryjne i techniczne, Elektrotechnika, elektronika, inżynieria
  informatyczna, Sprzęt komputerowy i architektura komputerów
