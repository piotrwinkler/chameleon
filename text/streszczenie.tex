\section*{Streszczenie}

  Tematem projektu jest zbadanie możliwości zastosowania sieci neuronowych do
  edycji obrazów. Głównym celem było stworzenie narzędzi, opartych na wyuczonych
  sieciach neuronowych, służących do odpowiedniego przetwarzania i
  modyfikowania obrazu. Następnie skuteczność tych narzędzi została oceniona i
  porównana z rozwiązaniami opartymi na klasycznych metodach przetwarzania obrazu.

  Zbadane zostały dwie problematyki, automatyczne kolorowanie czarno-białych obrazów
  oraz nakładanie na obraz prostych filtrów takich jak wykrywający
  krawędzie filtr Sobela-Feldmana.

  W celu ułatwienia przeprowadzania badań został opracowany autorski framework
  \textit{TorchFrame}. Ma on za zadanie kontrolować
  przepływ danych w procesie uczenia i testowania badanych modeli, a co za tym
  idzie, ograniczyć konieczność ingerencji ze strony użytkownika, przy jednoczesnym
  zapewnieniu swobody w prowadzonych eksperymentach i implementowanych modyfikacjach danych.

  W przypadku prostych filtrów obrazu rozwiązania
  oparte na sieciach splotowych okazały się wolniejsze i bardziej pracochłonne w
  implementacji niż metody klasyczne przy porównywalnych rezultatach. Jednakże
  z ich pomocą udało się udowodnić olbrzymią uniwersalność sieci splotowych
  mogących znaleźć zastosowanie w zagadnieniach związanych z prostym
  przetwarzaniem obrazu. Ukazane zostało ponadto podobieństwo między tradycyjnymi
  metodami filtracji, a sposobem działania neuronów tworzących sieci konwolucyjne.

  Do rozwiązania problematyki automatycznego kolorowania czarno-białych
  obrazów zostały przyjęte dwa różne podejścia. Pierwsze obejmowało
  zastosowanie prostego modelu autorskiego trenowanego z użyciem wielu różnych
  hiperparametrów oraz metod przygotowania danych treningowych.
  Otrzymane wyniki były zadowalające,
  jednakże są one w dużej mierze zależne od wybranych hiperparametrów oraz
  konfiguracji, w których uczona była sieć.

  Drugie podejście obejmowało zaimplementowanie bardziej złożonego modelu
  z użyciem techniki przeniesienia uczenia. Podejście to oparte
  było na integracji cech obrazu średniego oraz wysokiego poziomu.
  Tak uzyskane informacje były następnie wykorzystywane do predykcji
  prawdopodobnych barw obrazu.
  W rezultacie przełożyło się to na większą realistyczność otrzymanych kolorów,
  a co za tym idzie, wysoce satysfakcjonujące wyniki.

  Badania przeprowadzone w ramach tego projektu dyplomowego dowodzą, że
  sieci neuronowe mogą być skutecznie zastosowane jako narzędzia do wszechstronnej
  edycji obrazu. Jednakże, pomimo ich niezwykłych możliwości, sukces
  zależy w dużej mierze od dobrze przemyślanego
  wyboru stosowanej architektury oraz poprawnie przeprowadzonego
  procesu uczenia.

  \bigskip

  \noindent\textbf{Słowa kluczowe:} sieć neuronowa, przetwarzanie obrazu,
  konwolucyjna sieć neuronowa, splotowa sieć neuronowa,
  głęboka sieć neuronowa, filtry obrazu, automatyczne kolorowanie czarno-białych
  obrazów, platforma programistyczna

  \bigskip

  \noindent\textbf{Dziedzina nauki i techniki zgodna z OECD:} Nauki
  inżynieryjne i techniczne, Elektrotechnika, elektronika, inżynieria
  informatyczna, Sprzęt komputerowy i architektura komputerów
