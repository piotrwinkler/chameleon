\section*{Streszczenie}

  Tematem pracy jest zbadanie możliwości zastosowania sieci neuronowych do
  edycji obrazów. Głównym celem było stworzenie narzędzi opartych na wyuczonych
  sieciach neuronowych, służących do odpowiedniego przetwarzania i
  modyfikowania obrazu. Następnie skuteczność tych narzędzi została oceniona i
  porównana z rozwiązaniami opartymi na klasycznych metodach przetwarzania obrazu.

  Zbadane zostały dwie problematyki, automatyczne kolorowanie czarno-białych obrazów
  oraz nakładanie na obraz prostych filtrów takich jak filtr Sobela. Opisy
  zaimplementowanych rozwiązań zostały umieszczone w odpowiednych podrozdziałach
  rozdziału \ref{zaimplementowane rozwiazania}.

  Uzyskane wyniki pozwoliły wyciągnąć wiele pouczających wniosków. W przypadku obu
  problematyk udało się je rozwiązać z użyciem rozwiązań opartych na konwolucyjnych
  sieciach neuronowych.

  W przypadku prostych filtrów obrazu rozwiązania
  oparte na sieciach splotowych okazały się wolniejsze i bardziej pracochłonne w
  implementacji niż metody klasyczne przy porównywalnych rezultatach. Jednakże
  z ich pomocą udało się udowodnić olbrzymią uniwersalność sieci splotowych
  mogących być efektywnie zastosowane w zagadnieniach związanych z prostym
  przetwarzaniem obrazu.

  Zagadnienie automatycznego kolorowania czarno-białych
  obrazów zostało rozwiązane z użyciem dwóch różnych modeli, autorskiego modelu
  prostego oraz bardziej zaawansowanego modelu zaimplementowanego z użyciem technik
  przeniesienia uczenia. Dla modelu autorskiego zostały przeprowadzone szczegółowe
  badania dotyczące zależności wyników od poszczególnych parametrów architekury
  sieci jak i konfiguracji procesu uczenia. Rezultaty końcowe uzyskane z użyciem
  modelu autorskiego były zadowalajace, ale mocno zależne od wybranych parametrów.
  Dowodzi to, że odpowiednio skonfigurowane sieci splotowe mogą być skutecznie
  zastosowane w tym zagadnienu, aczkolwiek to model złożony pozwolił osiągnąć
  największy sukces.
%  Prezentuje on znacznie lepsze wyniki mogące być uznane za wybitne osiągnięcia
%  w zagadnieniu automatycznego kolorowania.
  Idea tego rozwiązania opiera się na integracji cech obrazu średniego oraz wysokiego
  poziomu uzyskiwanych z użyciem złozonej sieci splotowej wytrenowanej pierwotnie do
  zadania klasyfikacji. Cechy te, po przejściu przez proces fuzji, są następnie
  wykorzystywane przez sieć dekonwolucyjną do predykcji prawdopodobnych barw dla
  obrazu wejściowego. Dzięki tym dodatkowym informacjom o obrazie udało się
  znacznie zwiększyć efektywność procesu kolorowania, a co za tym idzie,
  wiarygodność generowanych barw.

  Badania przeprowadzone w ramach tej pracy dyplomowej są dowodem na to, że
  sieci neuronowe mogą być skutecznie zastosowane jako narzędzia do wszechstronnej
  edycji obrazu. Często są one jedynym dostępnym rozwiązaniem dla złożonej
  problematyki.

  \textless Zastosowane metody badań \textgreater \textless wyniki \textgreater
  \textless najważniejsze wnioski \textgreater

  \bigskip

  \noindent\textbf{Słowa kluczowe:} sieć neuronowa, przetwarzanie obrazu,
  konwolucyjna sieć neuronowa, splotowa sieć neuronowa,
  głęboka sieć neuronowa, filtry obrazu, automatyczne kolorowanie czarno-białych
  obrazów, platforma programistyczna

  \bigskip

  \noindent\textbf{Dziedzina nauki i techniki zgodna z OECD:} Nauki
  inżynieryjne i techniczne, Elektrotechnika, elektronika, inżynieria
  informatyczna, Sprzęt komputerowy i architektura komputerów
