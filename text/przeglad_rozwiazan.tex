\chapter{}

  \section{Przegląd rozwiązań}
  \tab Lata rozwoju sztucznych sieci neuronowych zaowocowały powstaniem wielu technik służących do analizy i edycji obrazów. W poniższym rozdziale zaprezentowane zostaną, oraz pokrótce opisane, najważniejsze i najciekawsze przykłady, z których część znajdzie rozwinięcie w dalszej części tej pracy.

  \subsection{Sieci splotowe}
  \tab

  \subsection{Modele generatywne}
  \tab Koncepcja modeli generatywnych, w skrócie GANów, przedstawiona została w 2014 roku przez Iana Goodfellow oraz jego współpracowników na uniwersytecie w Montrealu \cite{gan}. Modele te stanowią połączenie dwóch głębokich sieci neuronowych działających przeciwstawnie do siebie nawzajem.
  \newline \tab Pierwsza sieć to tak zwany generator. W odniesieniu do tematu pracy, jego działanie polega na generowaniu nowych obrazów, lub ich fragmentów na podstawie wektora szumów.
  \newline \tab Obrazy te przekazywane są, równolegle z zestawem obrazów prawdziwych, do dyskryminatora stanowiącego drugą część modelu GAN. Działanie tej sieci neuronowej polega na określeniu (w skali 0 do 1), w jakim stopniu produkty wyjściowe generatora odpowiadają obrazom rzeczywistym.
  \newline \tab W opisanym modelu występuje zatem podwójna pętla sprzężenia zwrotnego. Dyskryminator określa autentyczność obrazów porównująć je ze zdefiniowaną odgórnie bazą danych. Z kolei generator otrzymuje informację o skuteczności swojego działania ze strony dyskryminatora.
  \newline \tab Model generatywny znajduje się w stanie ciągłego konfliktu. Generator dąży do jak najdokładniejszego fałszowania obrazów w celu oszukania dyskryminatora, którego celem jest z kolei jak najdokładniejsze wykrywanie podróbek. Obie sieci neuronowe nieustannie dążą do osiągnięcia przewagi nad rywalem w procesie treningu. Ciągła rywalizacja sprawia, że zarówno generator, jak i dyskryminator zyskują coraz wyższą skuteczność działania.
  \newline \tab W praktyce modele generatywne są w stanie naśladować dowolną dystrybucję danych. Są w stanie kreować światy podobne do naszego w zakresie obrazu, dźwięku czy mowy. Można powiedzieć, że są to prawdziwi syntetyczni artyści.
