\section{Wstęp i cel pracy}

  Sztuczne sieci neuronowe sięgają swym początkiem lat 40. XX wieku.
  Historia ich rozwoju odnotowała trzy okresy, w których rozwiązania te
  odbijały się szerokim echem w środowisku naukowym.

  Pierwszy model neuronu, a potem perceptron zapoczątkowały
  rozwój tej dziedziny nauki, jednak pierwsze sieci jednowarstwowe nie były w
  stanie rozwiązywać złożonych problemów. Przeszkodę nie do pokonania stanowiła
  dla nich nawet prosta funkcja logiczna XOR. Z tego powodu badania sieci
  neuronowych zostały na długi czas porzucone.

  Pojawienie się algorytmu wstecznej propagacji błędów
  pozwalającego skutecznie uczyć wielowarstwowe sieci neuronowe ponownie
  wzmogło zainteresowanie tematem, jednak tym razem na drodze postępowi stanęły
  ograniczenia technologiczne ówczesnych czasów.

  Wreszcie wraz z nadejściem XXI wieku postępujący rozwój
  komputerów oraz internetu umożliwił sztucznym sieciom neuronowym rozwinięcie
  skrzydeł. Wejście w erę \textit{"big data"} otworzyło dostęp do olbrzymich zbiorów
  danych niezbędnych do treningu sieci, a pojawienie się wysokowydajnych
  jednostek obliczeniowych pozwoliło znacznie ten proces przyspieszyć.

  Zapoczątkowany w ten sposób rozwój trwa do dnia dzisiejszego.
  Sztuczne sieci neuronowe odnajdują zastosowanie w wielu dziedzinach życia i
  nauki. Grają w gry, przeprowadzają symulacje, przewidują i prognozują
  zachowania rynku czy pogody, a także analizują i przetwarzają obrazy cyfrowe.

  Z punktu widzenia niniejszej pracy największe znaczenie ma
  oczywiście ostatni z wymienionych punktów. Zdefiniowanie sieci neuronowych,
  jako matematycznych modeli obliczeniowych ujawnia ich naturalne predyspozycje
  do pracy na obrazach cyfrowych. W praktyce stanowią one bowiem zbiór liczb,
  wartości poszczególnych pikseli, który sieć neuronowa jest w stanie
  analizować, przetwarzać i modyfikować.

  \subsection{Cel pracy}

    Celem niniejszej pracy jest zbadanie możliwości zastosowania sieci neuronowych
    do edycji obrazu. Wiąże się to ze stworzenie narzędzi programistycznych
    umożliwiających implementowanie, trenowanie oraz testowanie sztucznych sieci
    neuronowych przeznaczonych do przetwarzania
    obrazów cyfrowych. W ramach podjętej tematyki szczególny nacisk położony
    zostanie na przetestowanie rozwiązań dedykowanych do pracy z grafiką, takich jak
    sieci splotowe.

    Po opracowaniu tychże narzędzi, opisane zostaną osiągnięte efekty pracy oraz
    zbadana zostanie skuteczność sieci neuronowych jako rozwiązania nakreślonej
    problematyki. Otrzymane rezultaty zostaną także porównane z rezultatami
    rozwiązań opartych na klasycznych metodach przetwarzania obrazów nie
    wykorzystujących technologii sztucznych sieci neuronowych.
    Omówione zostaną również wykorzystane architektury zaimplementowanych
    modeli oraz przetestowane konfiguracje procesu uczenia, zwłaszcza wykorzystane
    hiperparametry oraz sposoby przygotowania danych treningowych.

  \subsection{Założenia projektowe}

    Głównym założeniem pracy jest zaprojektowanie i zaimplementowanie
    służących do edycji obrazu narzędzi programistycznych, które powinny
    wykorzystywać do swoich celów odpowiednio wytrenowane sieci neuronowe.
    Na podstawie przeprowadzonych testów oceniona zostanie skuteczność
    wykorzystanych modeli oraz poprawność obranego podejścia do
    omawianej problematyki.

    Wykonana zostanie również analiza słuszności zastosowania sieci
    neuronowych jako rozwiązania przedstawionej problematyki, a uzyskane
    rezultaty zestawione zostaną ze znanymi metodami edycji
    obrazu nie opierającymi się na technologii sieci neuronowych.

  \subsection{Układ pracy}

    W pierwszym rozdziale opisane zostały podstawy teoretyczne, obejmujące najważniejsze
    zagadnienia związane z omawianą w pracy tematyką edycji obrazów z
    wykorzystaniem sieci neuronowych. Znaczący fragment tego rozdziału dedykowany
    jest sieciom splotowym stanowiącym most pomiędzy sztuczną
    inteligencją, a zagadnieniami związanymi z szeroko pojętym przetwarzaniem
    grafiki.

    Następnie zaprezentowane zostaną dotychczasowe dokonania i rezultaty pracy
    grup badawczych specjalizujących się w graficznych zastosowaniach sztucznej
    inteligencji.
    Przeanalizowana zostanie idea kryjąca się za każdym z analizowanych
    rozwiązań oraz uzyskane z ich pomocą wyniki.
    Przedstawione w tym rozdziale badania stanowią inspirację dla przygotowanych
    rozwiązań autorskich.

    Kolejny rozdział pracy stanowić będzie szczegółowy przegląd przygotowanego
    oprogramowania oraz zaimplementowanych za jego pomocą sieci neuronowych
    przeznaczonych do edycji obrazów. Zawarty zostanie opis prostych filtrów,
    demonstrujących ideę zastosowania sieci splotowych, oraz zaawansowanego
    modelu przeznaczonego do automatycznego kolorowania czarno-białych obrazów.

    W ostatnim rozdziale zawarte zostanie podsumowanie uzyskanych rezultatów wraz
    z krytyczną analizą zasadności zastosowania sztucznej inteligencji w
    zakresie grafiki cyfrowej.
