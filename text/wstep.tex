\chapter{}
  \section{Wstęp i cel pracy}

    \subsection{Wstęp teoretyczny}
    \tab Sztuczne sieci neuronowe sięgają swym początkiem lat 40. XX wieku. Historia ich rozwoju odnotowała trzy okresy, w których rozwiązania te odbijały się szerokim echem w środowisku naukowym.
    \newline \tab Pierwszy model neuronu, a potem perceptron zapoczątkowały rozwój tej dziedziny nauki, jednak pierwsze sieci jednowarstwowe nie były w stanie rozwiązywać złożonych problemów. Przeszkodę nie do pokonania stanowiła dla nich nawet prosta funkcja logiczna XOR. Z tego powodu badania seci neuronowych zostały na długi czas porzucone.
    \newline \tab Pojawienie się algorytmu wstecznej propagacji błędów pozwalającego skutecznie uczyć wielowarstwowe sieci neuronowe ponownie wzmogło zainteresowanie tematem, jednak tym razem na drodze postępowi stanęły ograniczenia technologiczne ówczesnych czasów.
    \newline \tab Wreszcie wraz z nadejściem XXI wieku postępujący rozwój komputerów oraz internetu umożliwił sztucznym sieciom neuronowym rozwinięcie skrzydeł. Wejście w erę "big data" otworzyło dostęp do olbrzymich zbiorów danych niezbędnych do treningu, a pojawienie się wysokowydajnych jednostek obliczeniowych pozwoliło znacznie ten proces skrócić.
    \newline \tab Zapoczątkowany w ten sposób rozwój trwa do dnia dzisiejszego. Sztuczne sieci neuronowe odnajdują zastosowanie w wielu dziedzinach życia i nauki. Grają w gry, przeprowadzają symulacje, przewidują i prognozują zachowanie rynku, czy pogody, analizują i przetwarzają obrazy cyfrowe.
    \newline \tab Z punktu widzenia niniejszej pracy największe znaczenie ma oczywiście ostatni z wymienionych punktów. Zdefiniowanie sieci nauronowych, jako matematycznych modeli obliczeniowych ujawnia ich naturalne predyspozycje do pracy na obrazach cyfrowych. W praktyce stanowią one bowiem zbiór liczb, wartości poszczególnych pikseli, które sieć neuronowa jest w stanie analizować, przetwarzać i zmieniać.

    \subsection{Cel pracy}

    \subsection{Układ pracy}
